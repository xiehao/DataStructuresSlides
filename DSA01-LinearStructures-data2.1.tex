\subsection{顺序存储与运算实现}

\begin{fragile}
    \frametitle{\insertsubsectionhead}
    \begin{block}{顺序列表(Sequence List)}
        \begin{itemize}
            \item 又名\textbf{顺序表},用一段地址\alert{连续}的存储单
                  元,\alert{依次}存储序列中的数据元素
        \end{itemize}
    \end{block}
    \pause
    \begin{exampleblock}{例}
        \begin{itemize}
            \item 考察序列$s = (34, 23, 67, 43)$采用顺序表形式存储的过程
                  \vspace{2ex}
                \begin{figure}
                    \centering
                    \begin{bytefield}{2}
                        \bitbox[]{2}{}
                        \bitboxes[]{3}{{\only<3|handout:0>{\mintinline[fontsize=\smaller]{c}{last}}}
                            {\only<4|handout:0>{\mintinline[fontsize=\smaller]{c}{last}}}
                            {\only<5|handout:0>{\mintinline[fontsize=\smaller]{c}{last}}}
                            {\only<6>{\mintinline[fontsize=\smaller]{c}{last}}}} \\
                        \bitbox[]{2}{}
                        \bitboxes[]{3}{{\only<3>{\tikzmark{sba}}}{\only<4>{\tikzmark{sba}}}
                            {\only<5>{\tikzmark{sba}}}{\only<6>{\tikzmark{sba}}}} \\
                        \bitbox[]{2}{$s$:}
                        \bitbox{3}{\only<3->{$34$}}\bitbox{3}{\only<4->{$23$}}\bitbox{3}{\only<5->{$67$}}\bitbox{3}{\only<6->{$43$}}
                        \bitboxes{3}{{}{}{}{}}
                    \end{bytefield}
                    \begin{tikzpicture}[ >=stealth, thick, black!50, overlay, remember picture, %
                            list item/.style={draw=gray, rounded corners, thick} %
                        ]
                        \only<3->{
                            \draw[->] ($({pic cs:{sba}})+(0,0.3)$) to [out=-90,in=90] ($({pic
                                cs:{sba}})+(0,-0.3)$);
                            % \fill ($({pic cs:{saa}})+(0,-0.03)$) circle [radius=0.05];
                        }
                    \end{tikzpicture}
                    \caption{顺序表存储过程演示}
                    \label{fig:demo_sequence_list_construction}
                \end{figure}
        \end{itemize}
    \end{exampleblock}
\end{fragile}

\begin{fragile}
    \frametitle{\insertsubsectionhead}
    \begin{alertblock}{思考:用何种属性描述顺序表?}
        \begin{itemize}
            \item<2-> 存储空间的起始位置
            \item<3-> \textbf{容量(capacity)}:最多可容纳的元素个数
            \item<4-> \textbf{长度(length)}:实际容纳的元素个数
        \end{itemize}
    \end{alertblock}
    \begin{figure}
        \centering
        \begin{bytefield}[bitwidth=1em]{2}
            \bitboxes[]{6}{{\only<2->{\mintinline[fontsize=\smaller]{c}{first}}}}
            \bitboxes[]{3}{{}{}{}{}{}{}{}{}} \\
            \bitboxes[]{6}{{\only<2->{\tikzmark{sca}}}}
            \bitboxes[]{3}{{}{}{}{}{}{}{}{}} \\
            \bitbox[]{3}{$s$:}
            \bitbox{3}{$34$}\bitbox{3}{$23$}\bitbox{3}{$67$}\bitbox{3}{$43$}
            \bitboxes{3}{{}{}{}{}}
            \bitbox{2}{$4$} \\
            \bitbox[]{3}{}
            \bitbox[t]{24}{\only<3->{$\underbrace{\hspace{30em}}_{\text{容量}}$}}
            \bitbox[t]{2}{\only<4->{$\underbrace{\hspace{2em}}_{\text{长度}}$}}
        \end{bytefield}
        \begin{tikzpicture}[ >=stealth, thick, black!50, overlay, remember picture, %
                list item/.style={draw=gray, rounded corners, thick} %
            ]
            \only<2->{
                \draw[->] ($({pic cs:{sca}})+(0,0.3)$) to [out=-90,in=90] ($({pic
                    cs:{sca}})+(0,-0.3)$);
            }
        \end{tikzpicture}
        \caption{顺序表的属性描述}
        \label{fig:demo_sequence_list_properties}
    \end{figure}
\end{fragile}

\begin{fragile}
    \frametitle{\insertsubsectionhead}
    \begin{alertblock}{思考:如何为顺序表分配内存?}
        \begin{itemize}
            \item<2-> 一维数组的静态分配
        \end{itemize}
    \end{alertblock}
    \begin{figure}
        \centering
        \begin{bytefield}[bitwidth=1em]{2}
            \bitbox[]{3}{$s$:}
            \bitbox{3}{$34$}\bitbox{3}{$23$}\bitbox{3}{$67$}\bitbox{3}{$43$}
            \bitboxes{3}{{}{}{}{}}
            \bitbox{2}{$4$}
        \end{bytefield}
        \caption{顺序表的内存分配}
        \label{fig:demo_sequence_list_memory_allocation}
    \end{figure}
\end{fragile}

\begin{fragile}
    \frametitle{\insertsubsectionhead}
    \begin{alertblock}{思考:如何获取任意元素的存储地址?}
        \begin{itemize}
            \item<5-> 令$p_{k}$为元素$a_{k}$的起始地址,且每个元素$a_{k}$所占空
                  间为$m$,则有
                  \[
                        p_{k} = p_{1} + (k - 1)m, \qquad k\in\mathbb{Z}\cap[1, n]
                  \]
            \item<6> 顺序列表中的元素可被\textbf{随机访问(Random Access)},时间
                  复杂度为$O(1)$
        \end{itemize}
    \end{alertblock}
    \begin{figure}
        \centering
        \begin{bytefield}[bitwidth=1em]{2}
            \bitboxes[]{6}{{\only<2->{$p_{1}$}}{}{\only<3->{$p_{k}$}}}\bitboxes[]{17}{} \\
            \bitboxes[]{6}{{\only<2->{\tikzmark{sda}}}{}{\only<3->{\tikzmark{sdb}}}}\bitboxes[]{17}{} \\
            \bitbox[]{3}{\baselinealign{$s:$}}
            \bitbox{3}{\baselinealign{$a_{1}$}}\bitbox{6}{\baselinealign{$\cdots$}}\bitbox{3}{\baselinealign{$a_{k-1}$}}
            \bitbox{3}{\baselinealign{$a_{k}$}}\bitbox{6}{\baselinealign{$\cdots$}}\bitbox{3}{\baselinealign{$a_{n}$}}
            \bitbox{6}{\baselinealign{$\cdots$}}\bitbox{2}{\baselinealign{$n$}} \\
            \bitbox[]{3}{}
            \bitbox[t]{12}{\only<4->{$\underbrace{\hspace{15em}}_{k-1}$}}\bitbox[t]{9}{}
            \bitbox[t]{3}{\only<5->{$\underbrace{\hspace{3.5em}}_{m}$}}
        \end{bytefield}
        \begin{tikzpicture}[ >=stealth, thick, black!50, overlay, remember picture, %
                list item/.style={draw=gray, rounded corners, thick} %
            ]
            \only<2->{
                \draw[->] ($({pic cs:{sda}})+(0,0.3)$) to [out=-90,in=90] ($({pic
                    cs:{sda}})+(0,-0.3)$);
            }
            \only<3->{
                \draw[->] ($({pic cs:{sdb}})+(0,0.3)$) to [out=-90,in=90] ($({pic
                    cs:{sdb}})+(0,-0.3)$);
            }
        \end{tikzpicture}
        \caption{顺序表的随机访问原理}
        \label{fig:demo_sequence_list_random_access}
    \end{figure}
\end{fragile}

\begin{fragile}
    \frametitle{\insertsubsectionhead}
    \bicolumns[0.5]{
        \begin{block}{顺序表的类型说明}
            \begin{itemize}
                \item 以整型常量表示\alert{容量}
                \item 以静态数组存储表中各元素
                \item 以尾元素的序号加$1$表示\alert{长度}
                \item 特别地,空表尾元素序号为$-1$
            \end{itemize}
        \end{block}
    }{
        \begin{minted}{c}
            #define CAPACITY 256
            
            typedef int DataType; // 元素类型

            typedef struct {
                DataType data[CAPACITY]; // 元素
                int last; // 尾元素序号
            } SequenceList;
        \end{minted}
    }
\end{fragile}

\begin{fragile}
    \frametitle{\insertsubsectionhead}
    \bicolumns[0.45]{
        \begin{block}{顺序表的初始化}
            \begin{itemize}
                \item 为空表动态分配空间
                \item 将尾元素序号置为$-1$
                \item 返回指向空表的指针
            \end{itemize}
        \end{block}
    }{
        \begin{minted}{c}
            SequenceList *create_sequence_list() {
                SequenceList *s = malloc(sizeof(SequenceList));
                if (s) {
                    s->last = -1;
                }
                return s;
            }
        \end{minted}
    }
\end{fragile}

\begin{fragile}
    \frametitle{\insertsubsectionhead}
    \bicolumns[0.45]{
        \begin{block}{向顺序表中插入元素}
            \begin{itemize}
                \item 检测并处理非法输入\footnotemark
                \item 依次\alert{向后}移动目标位置后元素
                \item 在目标位置处插入新元素
                \item 更新尾元素序号
            \end{itemize}
        \end{block}
    }{
        \begin{minted}[highlightlines={3, 4, 8}]{c}
            bool insert_sequence_list(
                    SequenceList *s, int k, DataType d) {
                if (full_sequence_list(s)
                        || wrong_insert_index(s, k)) {
                    return false; // 检测并处理各种非法插入情况
                }
                for (int i = s->last; i >= k; --i) {
                    s->data[i + 1] = s->data[i]; // 注意移动顺序
                }
                s->data[k] = d; // 插入新元素
                ++s->last; // 更新尾元素序号
                return true;
            }
        \end{minted}
    }

    \footnotetext{为表示布尔型变量,须包含
            \mintinline[fontsize=\smaller]{c}{<stdbool.h>}头,下同}
\end{fragile}

\begin{fragile}
    \frametitle{\insertsubsectionhead}
    \begin{exampleblock}{例}
        \begin{itemize}
            \item 在顺序表形式的序列$s=(35, 12, 24, 42)$中第$1$个位置处插入$33$
        \end{itemize}
    \end{exampleblock}
    \begin{figure}
        \centering
        \begin{bytefield}[bitwidth=1em]{2}
            \bitbox[]{3}{}
            \bitboxes[]{3}{{\tiny$0$}{\tiny$1$}{\tiny$2$}{\tiny$3$}{\tiny$4$}{\tiny$5$}{\tiny$6$}{\tiny$\dots$}} \\
            \bitbox[]{3}{\baselinealign{$s:$}}
            \bitboxes{3}{
                {$35$}
                {\only<1-4|handout:0>{$12$}\only<5->{$33$}}
                {\only<1-3|handout:0>{$24$}\only<4->{$12$}}
                {\only<1-2|handout:0>{$42$}\only<3->{$24$}}
                {\only<2->{$42$}}{}{}{}} \\
            \bitboxes[]{3}{{}{}{\tikzmark{sea}}{}{\only<1-5>{\tikzmark{seb}}}{\only<6->{\tikzmark{seb}}}{}{}{}} \\
            \bitboxes[]{3}{{}{}{\only<1-4>{$33$}}{}{\only<1-5|handout:0>{\mintinline[fontsize=\smaller]{c}{last}}}{\only<6->{\mintinline[fontsize=\smaller]{c}{last}}}{}{}{}}
        \end{bytefield}
        \begin{tikzpicture}[ >=stealth, thick, black!50, overlay, remember picture, %
                list item/.style={draw=gray, rounded corners, thick} %
            ]
            \draw[->] ($({pic cs:{sea}})+(0,-0.35)$) to [out=90,in=-90] ($({pic
                cs:{sea}})+(0,0.25)$);
            \draw[->] ($({pic cs:{seb}})+(0,-0.35)$) to [out=90,in=-90] ($({pic
                cs:{seb}})+(0,0.25)$);
        \end{tikzpicture}
        \caption{在顺序表中插入元素过程演示}
        \label{fig:demo_insert_sequence_list}
    \end{figure}
\end{fragile}

\begin{frame}
    \frametitle{\insertsubsectionhead}
    \begin{block}{顺序表插入算法的复杂度分析}
        \begin{itemize}
            \item<2-> 插入运算主要耗时于\alert{依次移动数据}
                \begin{itemize}
                    \item 在序号为$k$的位置插入时须移动$n-k$
                          次,$k\in\mathbb{Z}\cap[0, n]$
                \end{itemize}
            \item<3-> 若在序号为$k$的位置插入的概率为$p_{k}$,则移动次数的期望为
                \[
                    \mathbb{E}_{\text{insert}} = \sum_{k=0}^{n}{(n - k)p_{k}}
                \]
            \item<4-> 当在所有合法位置插入的概率均等时,即对所有合法的$k$均有
                $p_{k}=(n + 1)^{-1}$,则有
                \[
                    \mathbb{E}_{\text{insert}} = \frac{1}{n + 1}\sum_{k=0}^{n}{(n - k)} = \frac{n}{2}
                \]
            \item<5-> 综上,顺序表插入操作的时间复杂度为$O(n)$
        \end{itemize}
    \end{block}
\end{frame}

\begin{fragile}
    \frametitle{\insertsubsectionhead}
    \bicolumns[0.45]{
        \begin{block}{从顺序表中删除元素}
            \begin{itemize}
                \item 检测并处理非法输入
                \item 依次\alert{向前}移动目标位置后元素
                \item \alert{覆盖}目标位置处的元素
                \item 更新尾元素序号
            \end{itemize}
        \end{block}
    }{
        \begin{minted}[highlightlines={3, 7, 10}]{c}
            bool remove_sequence_list(
                    SequenceList *s, int k, DataType *d) {
                if (wrong_remove_index(s, k)) {
                    return false; // 检测并处理各种非法插入情况
                }
                if (d) {
                    *d = s->data[k]; // 记录并返回被删的值
                }
                for (int i = k; i < s->last; ++i) {
                    s->data[i] = s->data[i + 1]; // 注意移动顺序
                }
                --s->last; // 更新尾元素序号
                return true;
            }
        \end{minted}
    }
\end{fragile}

\begin{fragile}
    \frametitle{\insertsubsectionhead}
    \begin{exampleblock}{例}
        \begin{itemize}
            \item 删除顺序表形式的序列$s=(35, 33, 12, 24, 42)$中第$1$个位置元素
                  $33$
        \end{itemize}
    \end{exampleblock}
    \begin{figure}
        \centering
        \begin{bytefield}[bitwidth=1em]{2}
            \bitbox[]{3}{}
            \bitboxes[]{3}{{\tiny$0$}{\tiny$1$}{\tiny$2$}{\tiny$3$}{\tiny$4$}{\tiny$5$}{\tiny$6$}{\tiny$\dots$}} \\
            \bitbox[]{3}{\baselinealign{$s:$}}
            \bitboxes{3}{
                {$35$}
                {\only<2->{$12$}\only<1|handout:0>{$33$}}
                {\only<3->{$24$}\only<1-2|handout:0>{$12$}}
                {\only<4->{$42$}\only<1-3|handout:0>{$24$}}
                {\only<1->{$42$}}{}{}{}} \\
            \bitboxes[]{3}{{}{}{\tikzmark{sfa}}{}{\only<5->{\tikzmark{sfb}}}{\only<1-4|handout:0>{\tikzmark{sfb}}}{}{}{}} \\
            \bitboxes[]{3}{{}{}{\only<2->{$33$}}{}{\only<5->{\mintinline[fontsize=\smaller]{c}{last}}}{\only<1-4|handout:0>{\mintinline[fontsize=\smaller]{c}{last}}}{}{}{}}
        \end{bytefield}
        \begin{tikzpicture}[ >=stealth, thick, black!50, overlay, remember picture, %
                list item/.style={draw=gray, rounded corners, thick} %
            ]
            \draw[->] ($({pic cs:{sfa}})+(0,-0.35)$) to [out=90,in=-90] ($({pic
                cs:{sfa}})+(0,0.25)$);
            \draw[->] ($({pic cs:{sfb}})+(0,-0.35)$) to [out=90,in=-90] ($({pic
                cs:{sfb}})+(0,0.25)$);
        \end{tikzpicture}
        \caption{从顺序表中删除元素过程演示}
        \label{fig:demo_remove_sequence_list}
    \end{figure}
\end{fragile}

\begin{frame}
    \frametitle{\insertsubsectionhead}
    \begin{block}{顺序表删除算法的复杂度分析}
        \begin{itemize}
            \item<2-> 删除与插入运算互为逆运算,故仍主要耗时于\alert{依次移动数据}
            \item<3-> 同理可得顺序表删除操作的时间复杂度亦为$O(n)$
        \end{itemize}
    \end{block}
\end{frame}

\begin{fragile}
    \frametitle{\insertsubsectionhead}
    \bicolumns[0.45]{
        \begin{block}{在顺序表中按值查找元素}
            \begin{itemize}
                \item 从首至尾遍历表中所有元素
                \item 若发现匹配元素则返回其序号
                \item 否则返回$-1$表示查找失败
            \end{itemize}
        \end{block}
    }{
        \begin{minted}{c}
            int search_sequence_list(SequenceList *s, DataType d) {
                int k = 0;
                for (; k <= s->last && s->data[k] != d; ++k)
                    ; // 空语句,不执行任何操作
                return s->last < k ? -1 : k;
            }
        \end{minted}
    }
\end{fragile}

\begin{fragile}
    \frametitle{\insertsubsectionhead}
    \begin{exampleblock}{例}
        \begin{itemize}
            \item 在顺序表形式的序列$s=(35, 33, 12, 24, 42)$查找值为$12$的元素的
                  位置
        \end{itemize}
    \end{exampleblock}
    \begin{figure}
        \centering
        \begin{bytefield}[bitwidth=1em]{2}
            \bitbox[]{3}{}
            \bitboxes[]{3}{{\tiny$0$}{\tiny$1$}{\tiny$2$}{\tiny$3$}{\tiny$4$}{\tiny$5$}{\tiny$6$}{\tiny$\dots$}} \\
            \bitbox[]{3}{\baselinealign{$s:$}}
            \bitboxes{3}{{$35$}{$33$}{\alert<5>{$12$}}{$24$}{$42$}{}{}{}} \\
            \bitboxes[]{3}{{}{\only<2>{\tikzmark{sga}}}{\only<3>{\tikzmark{sga}}}{\only<4->{\tikzmark{sga}}}{}{}{}}
        \end{bytefield}
        \begin{tikzpicture}[ >=stealth, thick, black!50, overlay, remember picture, %
                list item/.style={draw=gray, rounded corners, thick} %
            ]
            \only<2->{
                \draw[->] ($({pic cs:{sga}})+(0,-0.35)$) to [out=90,in=-90] ($({pic
                    cs:{sga}})+(0,0.25)$);
            }
        \end{tikzpicture}
        \caption{在顺序表中查找元素过程演示}
        \label{fig:demo_search_sequence_list}
    \end{figure}
\end{fragile}

\begin{frame}
    \frametitle{\insertsubsectionhead}
    \begin{alertblock}{课堂思考练习}
        \begin{itemize}
            \item<2-> 分析顺序表删除算法的时间复杂度
            \item<3-> 实现逆序遍历搜索并讨论其与顺序遍历的异同
        \end{itemize}
    \end{alertblock}
\end{frame}

\begin{frame}
    \frametitle{\insertsubsectionhead}
    \begin{block}{顺序表的优势}
        \begin{itemize}
            \item 节省空间:无需为表示元素间逻辑关系而增加额外存储空间
            \item 随机访问:可快速访问表中任意位置元素,$T(n) = O(1)$
        \end{itemize}
    \end{block}
    \pause
    \begin{block}{顺序表的不足}
        \begin{itemize}
            \item 容量固定:表容量事先难以确定且难以扩充
            \item 增减困难:插入删除元素需移动大量元素,$T(n) = O(n)$
        \end{itemize}
    \end{block}
\end{frame}

\begin{frame}
    \frametitle{\insertsubsectionhead}
    \begin{alertblock}{课堂编程练习}
        \begin{itemize}
            \item 输入:顺序表$s_{a}$与$s_{b}$,其元素均已按\alert{升序}排列
            \item 输出:顺序表$s_{c}$,其元素为$s_{a}$与$s_{b}$中元素的合并,并
                  以\alert{降序}排列
        \end{itemize}
    \end{alertblock}
\end{frame}