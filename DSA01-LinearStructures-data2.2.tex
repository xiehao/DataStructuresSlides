\subsection{链式存储与运算实现}

\begin{frame}
    \frametitle{\insertsubsectionhead}
    \begin{block}{链式列表(Linked List)}
        \begin{itemize}
            \item 简称\textbf{链表},用一组在内存中\alert{零散}分布的存储单元存
                  储序列中的数据元素
            \item 按链接关系分为\textbf{单向链表}\footnote{简称\textbf{单链
                  表}}、\textbf{双向链表}与\textbf{循环链表}等
        \end{itemize}
    \end{block}
    \pause
    \begin{exampleblock}{例}
        \begin{itemize}
            \item 考察序列$s = (34, 23, 67, 43)$采用\alert{单链表}形式存储的过程
        \end{itemize}
        \begin{figure}
            \centering
            \begin{bytefield}{2}
                \bitbox[]{3}{}
                \bitbox[]{3}{\mintinline[fontsize=\smaller]{c}{head}}
                \bitbox[]{30}{} \\
                \bitbox[]{3}{}
                \bitbox[]{3}{\only<2->{\tikzmark{shf}}}
                \bitbox[]{30}{} \\
                \bitbox[]{3}{$s:$}
                \bitbox{3}{}\bitbox{2}{\tikzmark{sha}}\bitbox[]{2}{\tikzmark{tha}}
                \uncover<3->{\bitbox{3}{$34$}\bitbox{2}{\tikzmark{shb}}\bitbox[]{2}{\tikzmark{thb}}}
                \uncover<4->{\bitbox{3}{$23$}\bitbox{2}{\tikzmark{shc}}\bitbox[]{2}{\tikzmark{thc}}}
                \uncover<5->{\bitbox{3}{$67$}\bitbox{2}{\tikzmark{shd}}\bitbox[]{2}{\tikzmark{thd}}}
                \uncover<6->{\bitbox{3}{$43$}\bitbox{2}{\tikzmark{she}}}
            \end{bytefield}
            \begin{tikzpicture}[ >=stealth, thick, black!50, overlay, remember picture, %
                    list item/.style={draw=gray, rounded corners, thick} %
                ]
                \only<3->{
                    \draw[->] ($({pic cs:{sha}})+(0,-0.03)$) to [out=0,in=180] ($({pic
                        cs:{tha}})+(0.26,-0.03)$);
                    \fill ($({pic cs:{sha}})+(0,-0.03)$) circle [radius=0.05];
                }
                \only<4->{
                    \draw[->] ($({pic cs:{shb}})+(0,-0.03)$) to [out=0,in=180] ($({pic
                        cs:{thb}})+(0.26,-0.03)$);
                    \fill ($({pic cs:{shb}})+(0,-0.03)$) circle [radius=0.05];
                }
                \only<5->{
                    \draw[->] ($({pic cs:{shc}})+(0,-0.03)$) to [out=0,in=180] ($({pic
                        cs:{thc}})+(0.26,-0.03)$);
                    \fill ($({pic cs:{shc}})+(0,-0.03)$) circle [radius=0.05];
                }
                \only<6->{
                    \draw[->] ($({pic cs:{shd}})+(0,-0.03)$) to [out=0,in=180] ($({pic
                        cs:{thd}})+(0.26,-0.03)$);
                    \fill ($({pic cs:{shd}})+(0,-0.03)$) circle [radius=0.05];
                }
                \only<7->{
                    \draw ($({pic cs:{she}})+(-0.22,-0.3)$) -- ($({pic cs:{she}})+(0.23,0.23)$);
                }
                \only<2->{
                    \draw[->] ($({pic cs:{shf}})+(0,0.3)$) to [out=-90,in=90] ($({pic
                        cs:{shf}})+(0,-0.3)$);
                }
            \end{tikzpicture}
            \caption{单链表的存储过程演示}
            \label{fig:demo_linked_list_construction}
        \end{figure}
    \end{exampleblock}
\end{frame}

\begin{frame}
    \frametitle{\insertsubsectionhead}
    \begin{alertblock}{思考}
        \begin{itemize}
            \item 用何种属性描述单链表?
            \item 如何为单链表分配内存?
            \item 如何获取任意元素的存储地址?
        \end{itemize}
    \end{alertblock}
\end{frame}

\begin{fragile}
    \frametitle{\insertsubsectionhead}
    \bicolumns[0.5]{
        \begin{block}{单链表的类型说明}
            \begin{itemize}
                \item 以\alert{结点}为基本单位
                \item 为结点\alert{动态}分配内存
                \item 结点中包含下一个结点\alert{地址}
                \item 单链表只需记录\alert{首结点}即可
            \end{itemize}
        \end{block}
    }{
        \begin{minted}{c}            
            typedef int DataType; // 元素类型

            typedef struct LinkedNode {
                DataType data; // 数据元素
                struct LinkedNode *next; // 下一个结点
            } LinkedNode;

            typedef struct {
                LinkedNode *head; // 首结点
            } LinkedList;
        \end{minted}
    }
\end{fragile}

\begin{fragile}
    \frametitle{\insertsubsectionhead}
    \bicolumns[0.5]{
        \begin{block}{单链表结点的初始化}
            \begin{itemize}
                \item 为结点\alert{动态}分配内存
                \item 更新结点中各种数据
            \end{itemize}
        \end{block}
    }{
        \begin{minted}{c}
            LinkedNode *create_linked_node(DataType d) {
                LinkedNode *n = malloc(sizeof(LinkedNode));
                if (n) {
                    n->data = d;
                    n->next = NULL;
                }
                return n;
            }
        \end{minted}
    }
\end{fragile}

\begin{fragile}
    \frametitle{\insertsubsectionhead}
    \bicolumns[0.5]{
        \begin{block}{单链表的初始化}
            \begin{itemize}
                \item 为单链表\alert{动态}分配内存
                \item 以默认值\footnotemark{}初始化首结点
            \end{itemize}
        \end{block}
    }{
        \begin{minted}{c}
            LinkedList *create_linked_list() {
                LinkedList *s = malloc(sizeof(LinkedList));
                if (s) {
                    s->head = create_linked_node(0);
                }
                return s;
            }
        \end{minted}
    }
    
    \footnotetext{与\mintinline[fontsize=\smaller]{c}{DataType}有关,此处暂时为
        \mintinline[fontsize=\smaller]{c}{0}}
\end{fragile}

\begin{fragile}
    \frametitle{\insertsubsectionhead}
    \bicolumns[0.5]{
        \begin{block}{求单链表的长度}
            \begin{itemize}
                \item 从首结点开始遍历至尾结点
                \item 记录已遍历的结点数
                \item 计数时不包括首结点
            \end{itemize}
        \end{block}
        \uncover<2>{
            \begin{alertblock}{说明}
                \begin{itemize}
                    \item $T(n) = O(n)$
                    \item 注意与顺序表比较
                \end{itemize}
            \end{alertblock}
        }
    }{
        \begin{minted}{c}
            int get_linked_list_length(LinkedList *s) {
                int length = -1; // 计数不包括首结点
                for (LinkedNode *p = s->head; p != NULL;
                        p = p->next, ++length)
                    ; // 空语句
                return length;
            }
        \end{minted}
    }
\end{fragile}

\begin{fragile}
    \frametitle{\insertsubsectionhead}
    \bicolumns[0.5]{
        \begin{block}{在单链表中\alert{按序号}查找指定元素}
            \begin{itemize}
                \item 从首结点遍历指定次数即可
                \item 注意边界条件
            \end{itemize}
        \end{block}
        \only<2->{
            \begin{block}{在单链表中\alert{按值}查找指定元素}
                \begin{itemize}
                    \item 从首结点遍历至找到匹配值的结点
                    \item 遍历范围不应包括首结点
                \end{itemize}
            \end{block}
        }
        \only<3>{
            \begin{alertblock}{说明}
                \begin{itemize}
                    \item 二者均有:$T(n) = O(n)$
                    \item 注意与顺序表比较
                \end{itemize}
            \end{alertblock}
        }
    }{
        \begin{minted}[highlightlines={}]{c}
            LinkedNode *search_linked_by_index(
                    LinkedList *s, int k) {
                LinkedNode *p = s->head;
                for (int i = 0; p != NULL && i < k;
                        ++i, p = p->next)
                    ; // 空语句
                return p;
            }
        \end{minted}
        \pause
        \begin{minted}[highlightlines={}]{c}
            LinkedNode *search_linked_by_data(
                    LinkedList *s, DataType d) {
                LinkedNode *p = s->head->next;
                for (; p != NULL && p->data != d; p = p->next)
                    ; // 空语句
                return p;
            }
        \end{minted}
    }
\end{fragile}

\begin{fragile}
    \frametitle{\insertsubsectionhead}
    \bicolumns[0.5]{
        \begin{block}{在单链表中指定序号的结点\alert{后}插入元素}
            \begin{itemize}
                \item 按序号查找出目标结点
                \item 处理可能的非法查询结果
                \item 按指定值创建新结点
                \item 在目标结点后插入新结点
            \end{itemize}
        \end{block}
        \uncover<2>{
            \begin{alertblock}{说明}
                \begin{itemize}
                    \item 注意高亮代码的执行顺序
                    \item 插入本身:$T(n) = O(1)$
                    \item 连同查找:$T(n) = O(n)$
                    \item 注意与顺序表比较
                \end{itemize}
            \end{alertblock}
        }
    }{
        \begin{minted}[highlightlines={4, 5}]{c}
            LinkedNode *attach_after_node(
                    LinkedNode *p, LinkedNode *n) {
                assert(p && n);
                n->next = p->next;
                p->next = n;
                return n;
            }
        \end{minted}

        \begin{minted}[highlightlines={}]{c}
            bool insert_after_linked_by_index(
                    LinkedList *s, int k, DataType d) {
                LinkedNode *p = search_linked_by_index(s, k);
                if (!p) {
                    printf("Wrong insert index!\n");
                    return false;
                }
                LinkedNode *n = create_linked_node(d);
                return n && attach_after_node(p, n);
            }
        \end{minted}
    }
\end{fragile}

\begin{frame}
    \frametitle{\insertsubsectionhead}
    \begin{exampleblock}{例}
        \begin{itemize}
            \item 在单链表形式的序列$s = (34, 23, 67, 43)$中第$2$个结点$23$后插
                  入$54$
        \end{itemize}
        \begin{figure}
            \centering
            \begin{bytefield}{2}
                \bitbox[]{19}{}
                \uncover<5->{\bitbox[]{3}{\mintinline[fontsize=\smaller]{c}{n}}} \\
                \bitbox[]{19}{}
                \uncover<5->{\bitbox[]{3}{\tikzmark{sih}}} \\
                \bitbox[]{3}{}
                \bitbox[]{3}{\mintinline[fontsize=\smaller]{c}{head}}
                \bitbox[]{11}{}\bitbox[]{2}{\tikzmark{tif}}
                \uncover<5->{\bitbox{3}{$54$}\bitbox{2}{\tikzmark{sig}}} \\
                \bitbox[]{3}{}
                \bitbox[]{3}{\tikzmark{sif}}
                \bitbox[]{30}{} \\
                \bitbox[]{3}{$s:$}
                \bitbox{3}{}\bitbox{2}{\tikzmark{sia}}\bitbox[]{2}{\tikzmark{tia}}
                \bitbox{3}{$34$}\bitbox{2}{\tikzmark{sib}}\bitbox[]{2}{\tikzmark{tib}}
                \bitbox{3}{$23$}\bitbox{2}{\tikzmark{sic}}\bitbox[]{2}{\tikzmark{tic}}
                \bitbox{3}{$67$}\bitbox{2}{\tikzmark{sid}}\bitbox[]{2}{\tikzmark{tid}}
                \bitbox{3}{$43$}\bitbox{2}{\tikzmark{sie}} \\
                \bitbox[]{3}{}
                \bitbox[]{3}{\only<2>{\tikzmark{sii}}}\bitbox[]{4}{}
                \bitbox[]{3}{\only<3>{\tikzmark{sii}}}\bitbox[]{4}{}
                \bitbox[]{3}{\only<4->{\tikzmark{sii}}} \\
                \bitbox[]{3}{}
                \bitbox[]{3}{\only<2|handout:0>{\mintinline[fontsize=\smaller]{c}{p}}}\bitbox[]{4}{}
                \bitbox[]{3}{\only<3|handout:0>{\mintinline[fontsize=\smaller]{c}{p}}}\bitbox[]{4}{}
                \bitbox[]{3}{\only<4->{\mintinline[fontsize=\smaller]{c}{p}}}
            \end{bytefield}
            \begin{tikzpicture}[ >=stealth, thick, black!50, overlay, remember picture, %
                    list item/.style={draw=gray, rounded corners, thick} %
                ]
                \draw[->] ($({pic cs:{sia}})+(0,-0.03)$) to [out=0,in=180] ($({pic
                    cs:{tia}})+(0.26,-0.03)$);
                \fill ($({pic cs:{sia}})+(0,-0.03)$) circle [radius=0.05];
                \draw[->] ($({pic cs:{sib}})+(0,-0.03)$) to [out=0,in=180] ($({pic
                    cs:{tib}})+(0.26,-0.03)$);
                \fill ($({pic cs:{sib}})+(0,-0.03)$) circle [radius=0.05];
                \only<1-6|handout:0>{
                    \draw[->] ($({pic cs:{sic}})+(0,-0.03)$) to [out=0,in=180] ($({pic
                        cs:{tic}})+(0.26,-0.03)$);
                }
                \fill ($({pic cs:{sic}})+(0,-0.03)$) circle [radius=0.05];
                \draw[->] ($({pic cs:{sid}})+(0,-0.03)$) to [out=0,in=180] ($({pic
                    cs:{tid}})+(0.26,-0.03)$);
                \fill ($({pic cs:{sid}})+(0,-0.03)$) circle [radius=0.05];
                \draw ($({pic cs:{sie}})+(-0.22,-0.3)$) -- ($({pic cs:{sie}})+(0.23,0.23)$);
                \draw[->] ($({pic cs:{sif}})+(0,0.3)$) to [out=-90,in=90] ($({pic
                    cs:{sif}})+(0,-0.3)$);
                \only<5->{
                    \draw[->] ($({pic cs:{sih}})+(0,0.3)$) to [out=-90,in=90] ($({pic
                        cs:{sih}})+(0,-0.3)$);
                }
                \only<6->{
                    \draw[->] ($({pic cs:{sig}})+(0,-0.03)$) to [out=0,in=180] ($({pic
                        cs:{tic}})+(0.26,-0.03)$);
                    \fill ($({pic cs:{sig}})+(0,-0.03)$) circle [radius=0.05];
                }
                \only<7->{
                    \draw[->] ($({pic cs:{sic}})+(0,-0.03)$) to [out=0,in=180] ($({pic
                        cs:{tif}})+(0.26,-0.03)$);
                }
                \only<2->{
                    \draw[->] ($({pic cs:{sii}})+(0,-0.35)$) to [out=90,in=-90] ($({pic
                        cs:{sii}})+(0,0.25)$);
                }
            \end{tikzpicture}
            \caption{在单链表结点后插入新结点的过程演示}
            \label{fig:demo_insert_after_linked_node}
        \end{figure}
    \end{exampleblock}
\end{frame}

\begin{frame}
    \frametitle{\insertsubsectionhead}
    \begin{alertblock}{思考}
        \begin{itemize}
            \item 如何在单链表中指定序号的结点\alert{前}插入元素?
        \end{itemize}
    \end{alertblock}
\end{frame}

\begin{fragile}
    \frametitle{\insertsubsectionhead}
    \bicolumns[0.5]{
        \begin{block}{在单链表中\alert{删除}指定序号的结点}
            \begin{itemize}
                \item 按序号查找出目标结点的\alert{直接前驱}
                \item 处理可能的非法查询结果
                \item 更新结点的顺序关系
                \item 删除目标结点
            \end{itemize}
        \end{block}
        \uncover<2>{
            \begin{alertblock}{说明}
                \begin{itemize}
                    \item 注意高亮代码的执行顺序
                    \item 删除本身:$T(n) = O(1)$
                    \item 连同查找:$T(n) = O(n)$
                    \item 注意与顺序表比较
                \end{itemize}
            \end{alertblock}
        }
    }{
        \begin{minted}[highlightlines={3,4}]{c}
            LinkedNode *detach_after_node(LinkedNode *q) {
                assert(q && q->next);
                LinkedNode *p = q->next;
                q->next = p->next;
                return p;
            }
        \end{minted}
        \begin{minted}[highlightlines={}]{c}
            bool remove_linked_by_index(
                    LinkedList *s, int k, DataType *d) {
                LinkedNode *q = search_linked_by_index(s, k - 1);
                if (!q || !q->next) {
                    printf("Wrong remove index!\n");
                    return false;
                }
                LinkedNode *p = detach_after_node(q);
                if (d) { *d = p->data; }
                free(p);
                return true;
            }
        \end{minted}
    }
\end{fragile}

\begin{frame}
    \frametitle{\insertsubsectionhead}
    \begin{exampleblock}{例}
        \begin{itemize}
            \item 在单链表形式的序列$s = (34, 23, 67, 43)$中删除第$2$个结点$23$
        \end{itemize}
        \begin{figure}
            \centering
            \begin{bytefield}{2}
                \bitbox[]{3}{}
                \bitbox[]{3}{\mintinline[fontsize=\smaller]{c}{head}} \\
                \bitbox[]{3}{}
                \bitbox[]{3}{\tikzmark{sjf}}
                \bitbox[]{30}{} \\
                \bitbox[]{3}{$s:$}
                \bitbox{3}{}\bitbox{2}{\tikzmark{sja}}\bitbox[]{2}{\tikzmark{tja}}
                \bitbox{3}{$34$}\bitbox{2}{\tikzmark{sjb}}\bitbox[]{2}{\tikzmark{tjb}}
                \uncover<1-5|handout:0>{\bitbox{3}{$23$}\bitbox{2}{\tikzmark{sjc}}}\bitbox[]{2}{\tikzmark{tjc}}
                \bitbox{3}{$67$}\bitbox{2}{\tikzmark{sjd}}\bitbox[]{2}{\tikzmark{tjd}}
                \bitbox{3}{$43$}\bitbox{2}{\tikzmark{sje}} \\
                \bitbox[]{3}{}
                \bitbox[]{3}{\only<2>{\tikzmark{sji}}}\bitbox[]{4}{}
                \bitbox[]{3}{\only<3->{\tikzmark{sji}}}\bitbox[]{4}{}
                \bitbox[]{3}{\only<4->{\tikzmark{sjj}}} \\
                \bitbox[]{3}{}
                \bitbox[]{3}{\only<2|handout:0>{\mintinline[fontsize=\smaller]{c}{q}}}\bitbox[]{4}{}
                \bitbox[]{3}{\only<3->{\mintinline[fontsize=\smaller]{c}{q}}}\bitbox[]{4}{}
                \bitbox[]{3}{\only<4->{\mintinline[fontsize=\smaller]{c}{p}}}
            \end{bytefield}
            \begin{tikzpicture}[ >=stealth, thick, black!50, overlay, remember picture, %
                    list item/.style={draw=gray, rounded corners, thick} %
                ]
                \draw[->] ($({pic cs:{sja}})+(0,-0.03)$) to [out=0,in=180] ($({pic
                    cs:{tja}})+(0.26,-0.03)$);
                \fill ($({pic cs:{sja}})+(0,-0.03)$) circle [radius=0.05];
                \only<1-4|handout:0>{
                    \draw[->] ($({pic cs:{sjb}})+(0,-0.03)$) to [out=0,in=180] ($({pic
                        cs:{tjb}})+(0.26,-0.03)$);
                }
                \fill ($({pic cs:{sjb}})+(0,-0.03)$) circle [radius=0.05];
                \only<1-5|handout:0>{
                    \draw[->] ($({pic cs:{sjc}})+(0,-0.03)$) to [out=0,in=180] ($({pic
                        cs:{tjc}})+(0.26,-0.03)$);
                    \fill ($({pic cs:{sjc}})+(0,-0.03)$) circle [radius=0.05];
                }
                \draw[->] ($({pic cs:{sjd}})+(0,-0.03)$) to [out=0,in=180] ($({pic
                    cs:{tjd}})+(0.26,-0.03)$);
                \fill ($({pic cs:{sjd}})+(0,-0.03)$) circle [radius=0.05];
                \draw ($({pic cs:{sje}})+(-0.22,-0.3)$) -- ($({pic cs:{sje}})+(0.23,0.23)$);
                \draw[->] ($({pic cs:{sjf}})+(0,0.3)$) to [out=-90,in=90] ($({pic
                    cs:{sjf}})+(0,-0.3)$);
                \only<5->{
                    \draw[->] ($({pic cs:{sjb}})+(0,-0.03)$) to [out=60,in=120] ($({pic
                        cs:{tjc}})+(0.26,-0.03)$);
                }
                \only<2->{
                    \draw[->] ($({pic cs:{sji}})+(0,-0.35)$) to [out=90,in=-90] ($({pic
                        cs:{sji}})+(0,0.25)$);
                }
                \only<4-6|handout:0>{
                    \draw[->] ($({pic cs:{sjj}})+(0,-0.35)$) to [out=90,in=-90] ($({pic
                        cs:{sjj}})+(0,0.25)$);
                }
            \end{tikzpicture}
            \caption{删除单链表结点过程演示}
            \label{fig:demo_remove_linked_node}
        \end{figure}
    \end{exampleblock}
\end{frame}

\begin{frame}
    \frametitle{\insertsubsectionhead}
    \begin{block}{链表的优势}
        \begin{itemize}
            \item 容量可变:无需提前确定容量,可自动扩充
            \item 增删便捷:插入删除元素只影响局部,$T(n) = O(1)$\footnote{仅限
                  插入删除操作本身,不包括查找}
        \end{itemize}
    \end{block}
    \pause
    \begin{block}{链表的不足}
        \begin{itemize}
            \item 略占空间:每个结点均需额外存储下一结点的位置信息
            \item 访问不便:查找指定元素须从首结点开始遍历,$T(n) = O(n)$
        \end{itemize}
    \end{block}
\end{frame}

\begin{frame}
    \frametitle{\insertsectionhead}
    \begin{alertblock}{序列存储结构的选择}
        \begin{itemize}
            \item 存储:当存储规模事先难以估计时,宜选用链表
            \item 运算:当查找操作为主时,宜选用顺序表
            \item 实现:当需实现简单时,宜选用顺序表
        \end{itemize}
    \end{alertblock}
\end{frame}