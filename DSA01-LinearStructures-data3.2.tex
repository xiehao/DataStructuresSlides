\subsection{队列}

\begin{fragile}
    \frametitle{\insertsubsectionhead}
    \bicolumns[0.6]{
        \begin{block}{队列(Queue)}
            \begin{itemize}
                \item 另一种操作\alert{受限}的序列
                \item \alert{仅}允许在一端插入且在另一端删除元素
                \item 可插入的一端为\textbf{队尾(rear)},另一端为\textbf{队首(front)}
                \item \textbf{先进先出(First In First Out, FIFO)}
            \end{itemize}
        \end{block}
        \uncover<2>{
            \begin{exampleblock}{日常生活中的队列}
                \begin{itemize}
                    \item 排队的人(不允许插队)
                    \item 羽毛球桶
                    \item $\dots$
                \end{itemize}
            \end{exampleblock}
        }
    }{
        \begin{figure}
            \centering
            \begin{bytefield}{6}
                \stackbox{3.5}{6}{$\vdots$\\\vspace*{1ex}$\downarrow$} \\
                \begin{leftwordgroup}{$n$}
                    \begin{rightwordgroup}{\mintinline[fontsize=\smaller]{c}{队尾}}
                        \stackbox{1.5}{6}{$a_{n}$}
                    \end{rightwordgroup} \\
                    \stackbox{2.5}{6}{$\vdots$\vspace{1.5ex}} \\
                    \stackbox{1.5}{6}{$a_{2}$} \\
                    \begin{rightwordgroup}{\mintinline[fontsize=\smaller]{c}{队首}}
                        \stackbox{1.5}{6}{$a_{1}$}
                    \end{rightwordgroup}
                \end{leftwordgroup} \\
                \stackbox{3.5}{6}{$\downarrow$\vspace*{1ex}\\$\vdots$}
            \end{bytefield}
            \caption{队列的示意图}
            \label{fig:demo_queue}
        \end{figure}
    }
\end{fragile}

\begin{fragile}
    \frametitle{\insertsubsectionhead}
    \begin{block}{队列的抽象数据类型}
        \begin{minted}[linenos=false,escapeinside=@@]{c}
            ADT Queue {
            数据:
                数据对象: @$\mathcal{D} = \{a_{k} | a_{k}\in\text{数据元素集合}, k\in\mathbb{Z}\cap[1,n]\}$@
                逻辑关系: @$\mathcal{R} = \{\langle{}a_{k-1},a_{k}\rangle | k\in\mathbb{Z}\cap[2,n]\}$@
            操作:
                create_queue()
                    构造并初始化一个空队列@$s$@
                is_empty_queue()
                    判断队列是否为空
                in_queue(s, x)
                    若队列@$s$@存在, 则在其队尾插入值为@$x$@的新元素
                out_queue(s)
                    若队列@$s$@存在且非空, 则记录并删除其队首元素
            }
        \end{minted}
    \end{block}
\end{fragile}

\begin{fragile}
    \frametitle{\insertsubsectionhead}
    \bicolumns[0.5]{
        \begin{block}{顺序队列(版本甲)}
            \begin{itemize}
                \item 可采用已有序列结构的封装
                \item 序列第一个元素为\alert{队首}
                \item 序列最后一个元素为\alert{队尾}
                \item 入队时可直接在末尾插入新元素
                \item 出队时可删除第一个元素
            \end{itemize}
        \end{block}
    }{
        \begin{minted}[highlightlines={}]{c}
            typedef struct {
                SequenceList *_;
            } SequenceQueue;
        \end{minted}
        \begin{minted}[highlightlines={}]{c}
            bool push_sequence_queue(
                    SequenceQueue *s, DataType d) {
                return s && insert_sequence_list(
                        s->_, s->_->last + 1, d);
            }
        \end{minted}
        \begin{minted}[highlightlines={}]{c}
            bool pop_sequence_queue(
                    SequenceQueue *s, DataType *d) {
                return s && remove_sequence_list(s->_, 0, d);
            }
        \end{minted}
    }
\end{fragile}

\begin{frame}
    \frametitle{\insertsubsectionhead}
    \begin{block}{时间复杂度}
        \begin{itemize}
            \item 入队操作:无需移动任何元素,$O(1)$
            \item 出队操作:需移动所有元素,\alert{$O(n)$,如何改进?}
        \end{itemize}
    \end{block}
    \pause
    \begin{alertblock}{思路}
        \begin{itemize}
            \item \alert{与其移动元素,不如移动队首,正如队尾一样}
        \end{itemize}
    \end{alertblock}
\end{frame}

\begin{fragile}
    \frametitle{\insertsubsectionhead}
    \bicolumns[0.5]{
        \begin{block}{顺序队列(版本乙)}
            \begin{itemize}
                \item 总体同版本甲
                \item 新增队首序号成员
                \item 出队时仅需增加队首序号
            \end{itemize}
        \end{block}
    }{
        \begin{minted}[highlightlines={}]{c}
            typedef struct {
                SequenceList *_;
                int front;
            } SequenceQueue;
        \end{minted}
        \begin{minted}[highlightlines={}]{c}
            bool pop_sequence_queue(
                    SequenceQueue *s, DataType *d) {
                if (!s) {
                    return false;
                }
                if (d) {
                    *d = s->_->data[front];
                }
                return ++front;
            }
        \end{minted}
    }
\end{fragile}

\begin{fragile}
    \frametitle{\insertsubsectionhead}
    \begin{block}{假上溢现象}
        \begin{itemize}
            \item 队首前尚有可用空间,但队尾已至序列尽头
        \end{itemize}
    \end{block}
    \begin{figure}
        \centering
        \begin{bytefield}{2}
            \bitboxes[]{3}{{}{}{\mintinline[fontsize=\smaller]{c}{front}}} \\
            \bitboxes[]{3}{{}{}{\tikzmark{ska}}} \\
            \bitbox[]{3}{$s:$}
            \bitboxes{3}{{\only<5->{$54$}}{}{$34$}{\only<2->{$23$}}{\only<3->{$67$}}{\only<4->{$43$}}} \\
            \bitboxes[]{3}{{}{\only<5>{\tikzmark{skb}}}{}{\only<1>{\tikzmark{skb}}}{\only<2>{\tikzmark{skb}}}{\only<3>{\tikzmark{skb}}}{\only<4>{\tikzmark{skb}}}} \\
            \bitboxes[]{3}{{}{\only<5>{\mintinline[fontsize=\smaller]{c}{rear}}}{}{\only<1>{\mintinline[fontsize=\smaller]{c}{rear}}}{\only<2>{\mintinline[fontsize=\smaller]{c}{rear}}}{\only<3>{\mintinline[fontsize=\smaller]{c}{rear}}}{\only<4>{\mintinline[fontsize=\smaller]{c}{rear}}}}
        \end{bytefield}
        \begin{tikzpicture}[ >=stealth, thick, black!50, overlay, remember picture, %
                list item/.style={draw=gray, rounded corners, thick} %
            ]
            \draw[->] ($({pic cs:{ska}})+(0,0.3)$) to [out=-90,in=90] ($({pic
                cs:{ska}})+(0,-0.3)$);
            \draw[<-] ($({pic cs:{skb}})+(0,0.25)$) to [out=-90,in=90] ($({pic
                cs:{skb}})+(0,-0.35)$);
        \end{tikzpicture}
        \caption{队列的假上溢现象及其解决思路}
        \label{fig:queue_fake_overflow}
    \end{figure}
    \pause
    \uncover<5>{
        \begin{alertblock}{思路}
            \begin{itemize}
                \item \alert{采用\textbf{循环队列},假上溢时可将队尾移至内置数组
                          首元素,以充分利用空间}
            \end{itemize}
        \end{alertblock}
    }
\end{fragile}

\begin{fragile}
    \frametitle{\insertsubsectionhead}
    \begin{block}{新问题:如何判断队列已空或已满?}
        \vspace*{2ex}
        \bicolumns[0.5]{
            \begin{figure}
                \centering
                \begin{bytefield}{2}
                    \bitboxes[]{3}{{}{}{\mintinline[fontsize=\smaller]{c}{front}}} \\
                    \bitboxes[]{3}{{}{}{\tikzmark{skc}}} \\
                    \bitbox[]{3}{$s:$}
                    \bitboxes{3}{{}{}{}{}{}} \\
                    \bitboxes[]{3}{{}{}{\tikzmark{skd}}} \\
                    \bitboxes[]{3}{{}{}{\mintinline[fontsize=\smaller]{c}{rear}}}
                \end{bytefield}
                \begin{tikzpicture}[ >=stealth, thick, black!50, overlay, remember picture, %
                        list item/.style={draw=gray, rounded corners, thick} %
                    ]
                    \draw[->] ($({pic cs:{skc}})+(0,0.3)$) to [out=-90,in=90] ($({pic
                        cs:{skc}})+(0,-0.3)$);
                    \draw[<-] ($({pic cs:{skd}})+(0,0.25)$) to [out=-90,in=90] ($({pic
                        cs:{skd}})+(0,-0.35)$);
                \end{tikzpicture}
                \caption{队列已空}
                \label{fig:queue_empty}
            \end{figure}
        }{
            \begin{figure}
                \centering
                \begin{bytefield}{2}
                    \bitboxes[]{3}{{}{}{\mintinline[fontsize=\smaller]{c}{front}}} \\
                    \bitboxes[]{3}{{}{}{\tikzmark{ske}}} \\
                    \bitbox[]{3}{$s:$}
                    \bitboxes{3}{{$54$}{$34$}{$23$}{$67$}{$43$}} \\
                    \bitboxes[]{3}{{}{}{\tikzmark{skf}}} \\
                    \bitboxes[]{3}{{}{}{\mintinline[fontsize=\smaller]{c}{rear}}}
                \end{bytefield}
                \begin{tikzpicture}[ >=stealth, thick, black!50, overlay, remember picture, %
                        list item/.style={draw=gray, rounded corners, thick} %
                    ]
                    \draw[->] ($({pic cs:{ske}})+(0,0.3)$) to [out=-90,in=90] ($({pic
                        cs:{ske}})+(0,-0.3)$);
                    \draw[<-] ($({pic cs:{skf}})+(0,0.25)$) to [out=-90,in=90] ($({pic
                        cs:{skf}})+(0,-0.35)$);
                \end{tikzpicture}
                \caption{队列已满}
                \label{fig:queue_full}
            \end{figure}
        }
    \end{block}
\end{fragile}

\begin{frame}
    \frametitle{\insertsubsectionhead}
    \begin{block}{思路甲:设立标志位\mintinline[fontsize=\smaller]{c}{flag}辅助判断}
        \begin{itemize}
            \item 除检查队首与队尾是否重合外,尚需判断标志位(可约定:0为队空,1
                  为队满)
        \end{itemize}
    \end{block}
    \pause
    \begin{block}{思路乙:在队首处空出一个元素}
        \begin{itemize}
            \item 队空时队首队尾重合,队满时二者不重合
        \end{itemize}
    \end{block}
    \pause
    \begin{block}{思路丙:新增\mintinline[fontsize=\smaller]{c}{size}成员记录队列长度}
        \begin{itemize}
            \item 队空时长度为0,队满时长度为容量
        \end{itemize}
    \end{block}
\end{frame}

\begin{fragile}
    \frametitle{\insertsubsectionhead}
    \bicolumns[0.5]{
        \begin{block}{顺序队列(版本丙)}
            \begin{itemize}
                \item 总体同版本乙
                \item 新增队列长度成员
                \item 因出入较大,故已不借用序列结构
            \end{itemize}
        \end{block}
    }{
        \begin{minted}[highlightlines={}]{c}
            typedef struct {
                DataType data[CAPACITY];
                int front; // 队首
                int rear; // 队尾
                int size; // 队列长度
            } SequenceQueue;
        \end{minted}
    }
\end{fragile}

\begin{fragile}
    \frametitle{\insertsubsectionhead}
    \bicolumns[0.5]{
        \begin{block}{顺序队列的初始化}
            \begin{itemize}
                \item 为队列分配空间
                \item 将队首与队尾均置于末尾
                \item 将队列长度置0
            \end{itemize}
        \end{block}
    }{
        \begin{minted}[highlightlines={}]{c}
            SequenceQueue *create_sequence_queue() {
                SequenceQueue *s = malloc(sizeof(SequenceQueue));
                if (s) {
                    s->front = s->rear = CAPACITY - 1;
                    s->size = 0;
                }
                return s;
            }
        \end{minted}
    }
\end{fragile}

\begin{fragile}
    \frametitle{\insertsubsectionhead}
    \bicolumns[0.5]{
        \begin{block}{入队}
            \begin{itemize}
                \item 确保队列未满
                \item 队尾后移
                      \begin{itemize}
                          \item 若已至绝境则从头再来
                      \end{itemize}
                \item 加入新元素
                \item 更新队列长度
            \end{itemize}
        \end{block}
    }{
        \begin{minted}[highlightlines={}]{c}
            bool push_sequence_queue(
                    SequenceQueue *s, DataType d) {
                if (full_sequence_queue(s)) {
                    return false;
                }
                s->rear = (s->rear + 1) % CAPACITY;
                s->data[s->rear] = d;
                return ++s->size;
            }
        \end{minted}
    }
\end{fragile}

\begin{fragile}
    \frametitle{\insertsubsectionhead}
    \bicolumns[0.5]{
        \begin{block}{出队}
            \begin{itemize}
                \item 确保队列非空
                \item 队首后移\footnotemark
                      \begin{itemize}
                          \item 若已至绝境则从头再来
                      \end{itemize}
                \item 记录已删除元素
                \item 更新队列长度
            \end{itemize}
        \end{block}
    }{
        \begin{minted}[highlightlines={}]{c}
            bool pop_sequence_queue(
                    SequenceQueue *s, DataType *d) {
                if (empty_sequence_queue(s)) {
                    return false;
                }
                s->front = (s->front + 1) % CAPACITY;
                if (d) {
                    *d = s->data[s->front];
                }
                --s->size;
                return true;
            }
        \end{minted}
    }

    \footnotetext{注:队首一般指向首元素\alert{前一个}位置}
\end{fragile}

\begin{fragile}
    \frametitle{\insertsubsectionhead}
    \bicolumns[0.5]{
        \begin{block}{链式队列}
            \begin{itemize}
                \item 可采用已有链表结构的封装
                \item 新增队首与队尾指针
            \end{itemize}
        \end{block}
    }{
        \begin{minted}[highlightlines={}]{c}
            typedef struct {
                LinkedList *_;
                LinkedNode *front; // 队首
                LinkedNode *rear; // 队尾
            } LinkedQueue;
        \end{minted}
    }
\end{fragile}

\begin{fragile}
    \frametitle{\insertsubsectionhead}
    \begin{exampleblock}{链式队列}
        \vspace*{2ex}
        \bicolumns[0.3]{
            \begin{figure}
                \centering
                \begin{bytefield}{2}
                    \bitboxes[]{3}{{}{\mintinline[fontsize=\smaller]{c}{front}}} \\
                    \bitboxes[]{3}{{}{\tikzmark{smf}}} \\
                    \bitbox[]{3}{$s:$}
                    \bitbox{3}{}\bitbox{2}{\tikzmark{sma}} \\
                    \bitbox[]{3}{}
                    \bitbox[]{3}{\tikzmark{smg}} \\
                    \bitbox[]{3}{}
                    \bitbox[]{3}{\mintinline[fontsize=\smaller]{c}{rear}}
                \end{bytefield}
                \begin{tikzpicture}[ >=stealth, thick, black!50, overlay, remember picture, %
                        list item/.style={draw=gray, rounded corners, thick} %
                    ]
                    \draw ($({pic cs:{sma}})+(-0.22,-0.3)$) -- ($({pic cs:{sma}})+(0.23,0.23)$);
                    \draw[->] ($({pic cs:{smf}})+(0,0.3)$) to [out=-90,in=90] ($({pic
                        cs:{smf}})+(0,-0.3)$);
                    \draw[<-] ($({pic cs:{smg}})+(0,0.25)$) to [out=-90,in=90] ($({pic
                        cs:{smg}})+(0,-0.35)$);
                \end{tikzpicture}
                \caption{空链式队列}
                \label{fig:demo_empty_linked_queue}
            \end{figure}
        }{
            \begin{figure}
                \centering
                \begin{bytefield}{2}
                    \bitboxes[]{3}{{}{\mintinline[fontsize=\smaller]{c}{front}}} \\
                    \bitboxes[]{3}{{}{\tikzmark{slf}}} \\
                    \bitbox[]{3}{$s:$}
                    \bitbox{3}{}\bitbox{2}{\tikzmark{sla}}\bitbox[]{2}{\tikzmark{tla}}
                    \bitbox{3}{$34$}\bitbox{2}{\tikzmark{slb}}\bitbox[]{2}{\tikzmark{tlb}}
                    \bitbox{3}{$23$}\bitbox{2}{\tikzmark{slc}}\bitbox[]{2}{\tikzmark{tlc}}
                    \bitbox{3}{$43$}\bitbox{2}{\tikzmark{sld}} \\
                    \bitbox[]{24}{}
                    \bitbox[]{3}{\tikzmark{slg}} \\
                    \bitbox[]{24}{}
                    \bitbox[]{3}{\mintinline[fontsize=\smaller]{c}{rear}}
                \end{bytefield}
                \begin{tikzpicture}[ >=stealth, thick, black!50, overlay, remember picture, %
                        list item/.style={draw=gray, rounded corners, thick} %
                    ]
                    \draw[->] ($({pic cs:{sla}})+(0,-0.03)$) to [out=0,in=180] ($({pic
                        cs:{tla}})+(0.26,-0.03)$);
                    \fill ($({pic cs:{sla}})+(0,-0.03)$) circle [radius=0.05];
                    \draw[->] ($({pic cs:{slb}})+(0,-0.03)$) to [out=0,in=180] ($({pic
                        cs:{tlb}})+(0.26,-0.03)$);
                    \fill ($({pic cs:{slb}})+(0,-0.03)$) circle [radius=0.05];
                    \draw[->] ($({pic cs:{slc}})+(0,-0.03)$) to [out=0,in=180] ($({pic
                        cs:{tlc}})+(0.26,-0.03)$);
                    \fill ($({pic cs:{slc}})+(0,-0.03)$) circle [radius=0.05];
                    \draw ($({pic cs:{sld}})+(-0.22,-0.3)$) -- ($({pic cs:{sld}})+(0.23,0.23)$);
                    \draw[->] ($({pic cs:{slf}})+(0,0.3)$) to [out=-90,in=90] ($({pic
                        cs:{slf}})+(0,-0.3)$);
                    \draw[<-] ($({pic cs:{slg}})+(0,0.25)$) to [out=-90,in=90] ($({pic
                        cs:{slg}})+(0,-0.35)$);
                \end{tikzpicture}
                \caption{非空链式队列}
                \label{fig:demo_non_empty_linked_queue}
            \end{figure}
        }
    \end{exampleblock}
\end{fragile}

\begin{fragile}
    \frametitle{\insertsubsectionhead}
    \bicolumns[0.5]{
        \begin{block}{链式队列的初始化}
            \begin{itemize}
                \item 调用链表初始化方法
                \item 将队首与队尾均指向首结点
            \end{itemize}
        \end{block}
    }{
        \begin{minted}[highlightlines={}]{c}
            LinkedQueue *create_linked_queue() {
                LinkedQueue *s = malloc(sizeof(LinkedQueue));
                if (s) {
                    s->_ = create_linked_list();
                    s->front = s->rear = s->_->head;
                }
                return s;
            }
        \end{minted}
    }
\end{fragile}

\begin{fragile}
    \frametitle{\insertsubsectionhead}
    \bicolumns[0.5]{
        \begin{block}{链式队列的判空}
            \begin{itemize}
                \item 队首与队尾是否重合
            \end{itemize}
        \end{block}
    }{
        \begin{minted}[highlightlines={}]{c}
            bool empty_linked_queue(LinkedQueue *s) {
                return !s || s->front == s->rear;
            }
        \end{minted}
    }
\end{fragile}

\begin{fragile}
    \frametitle{\insertsubsectionhead}
    \bicolumns[0.5]{
        \begin{block}{入队}
            \begin{itemize}
                \item 创建新结点
                \item 在链表末端插入新结点\footnotemark
                \item 队尾指针后移
            \end{itemize}
        \end{block}
    }{
        \begin{minted}[highlightlines={}]{c}
            bool push_linked_queue(
                    LinkedQueue *s, DataType d) {
                if (!s) {
                    return false;
                }
                LinkedNode *n = create_linked_node(d);
                return n && (s->rear =
                    attach_after_node(s->rear, n));
            }
        \end{minted}
    }

    \footnotetext{注:此时已有队尾指针记录队尾结点,故无需每次查找之}
\end{fragile}

\begin{fragile}
    \frametitle{\insertsubsectionhead}
    \bicolumns[0.5]{
        \begin{block}{出队}
            \begin{itemize}
                \item 记录链表第一结点的值
                \item 删除该首结点
                \item 当队列空时队尾指针前移
            \end{itemize}
        \end{block}
    }{
        \begin{minted}[highlightlines={}]{c}
            bool pop_linked_queue(
                    LinkedQueue *s, DataType *d) {
                if (empty_linked_queue(s)) {
                    return false;
                }
                LinkedNode *p = detach_after_node(s->front);
                if (d) {
                    *d = p->data;
                }
                free(p);
                if (empty_linked_queue(s)) {
                    s->rear = s->front;
                }
                return true;
            }
        \end{minted}
    }
\end{fragile}
