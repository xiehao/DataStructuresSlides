\subsection{二叉树的遍历}

\begin{frame}
    \frametitle{\insertsubsectionhead}
    \begin{block}{遍历(traversal)}
        \begin{itemize}
            \item 按某种约定顺序访问半线性结构中的所有结点
            \item 每个结点均\alert{被且仅被}访问$1$次
            \item \alert{意义}:使半线性结构转化为线性结构
            \item 两类常见遍历方式:\alert{深度}优先与\alert{广度}优先
            \item 前者可按访问\alert{根}结点的次序区分
                \begin{itemize}
                    \item \textbf{先序(preorder)遍历}:\alert{根}结点
                        $\Rightarrow$子树序列\footnote{按顺序遍历每个子树,遍历
                        方式亦为递归相同遍历方式,其余类同}
                    \item \textbf{中序(inorder)遍历}\footnote{仅针对二叉树}:左
                        子树$\Rightarrow$\alert{根}结点$\Rightarrow$右子树
                    \item \textbf{后序(postorder)遍历}:子树序列
                        $\Rightarrow$\alert{根}结点
                \end{itemize}
            \item 后者包括\textbf{层序(level order)遍历}
        \end{itemize}
    \end{block}
\end{frame}

\begin{fragile}
    \frametitle{\insertsubsectionhead}
    \begin{figure}
        \centering
        \begin{tikzpicture}[ >=stealth, thick, black!50, %
                list item/.style={draw=gray, circle, thick}, %
                level distance=5ex, %
                level/.style={sibling distance=24ex/#1}
            ]
            \node [terminal] (a) {$A$}
                child {node [terminal] (b) {$B$}
                    child {node [terminal] (d) {$D$}}
                    child {node [terminal] (e) {$E$}
                        child {node [terminal] (g) {$G$}}
                        child [missing]
                    }
                }
                child {node [terminal] (c) {$C$}
                    child [missing]
                    child {node [terminal] (f) {$F$}}
                };
            \draw [dashed,->] (a) to [in=90,out=-90] (b);
            \draw [dashed,->] (b) to [in=90,out=-90] (d);
            \draw [dashed,->] (d) to [in=90,out=-90] (e);
            \draw [dashed,->] (e) to [in=90,out=-90] (g);
            \draw [dashed,->] (g) to [in=90,out=-90] (c);
            \draw [dashed,->] (c) to [in=90,out=-90] (f);
        \end{tikzpicture}
        \vspace{-8ex}
        \caption{二叉树的\alert{先}序遍历示
        例:$A\rightarrow{}B\rightarrow{}D\rightarrow{}E\rightarrow{}G\rightarrow{}C\rightarrow{}F$}
        \label{fig:demo_traverse_binary_tree_preorder}
    \end{figure}
\end{fragile}

\begin{fragile}
    \frametitle{\insertsubsectionhead}
    \begin{figure}
        \centering
        \begin{tikzpicture}[ >=stealth, thick, black!50, %
                list item/.style={draw=gray, circle, thick}, %
                level distance=5ex, %
                level/.style={sibling distance=24ex/#1}
            ]
            \node [terminal] (a) {$A$}
                child {node [terminal] (b) {$B$}
                    child {node [terminal] (d) {$D$}}
                    child {node [terminal] (e) {$E$}
                        child {node [terminal] (g) {$G$}}
                        child [missing]
                    }
                }
                child {node [terminal] (c) {$C$}
                    child [missing]
                    child {node [terminal] (f) {$F$}}
                };
            \draw [dashed,->] (d) to [in=180,out=0] (b);
            \draw [dashed,->] (b) to [in=180,out=0] (g);
            \draw [dashed,->] (g) to [in=180,out=0] (e);
            \draw [dashed,->] (e) to [in=180,out=0] (a);
            \draw [dashed,->] (a) to [in=180,out=0] (c);
            \draw [dashed,->] (c) to [in=180,out=0] (f);
        \end{tikzpicture}
        % \vspace{0ex}
        \caption{二叉树的\alert{中}序遍历示
        例:$D\rightarrow{}B\rightarrow{}G\rightarrow{}E\rightarrow{}A\rightarrow{}C\rightarrow{}F$}
        \label{fig:demo_traverse_binary_tree_inorder}
    \end{figure}
\end{fragile}

\begin{fragile}
    \frametitle{\insertsubsectionhead}
    \begin{figure}
        \centering
        \begin{tikzpicture}[ >=stealth, thick, black!50, %
                list item/.style={draw=gray, circle, thick}, %
                level distance=5ex, %
                level/.style={sibling distance=24ex/#1}
            ]
            \node [terminal] (a) {$A$}
                child {node [terminal] (b) {$B$}
                    child {node [terminal] (d) {$D$}}
                    child {node [terminal] (e) {$E$}
                        child {node [terminal] (g) {$G$}}
                        child [missing]
                    }
                }
                child {node [terminal] (c) {$C$}
                    child [missing]
                    child {node [terminal] (f) {$F$}}
                };
            \draw [dashed,->] (d) to [in=-90,out=90] (g);
            \draw [dashed,->] (g) to [in=-90,out=90] (e);
            \draw [dashed,->] (e) to [in=-90,out=90] (b);
            \draw [dashed,->] (b) to [in=-90,out=90] (f);
            \draw [dashed,->] (f) to [in=-90,out=90] (c);
            \draw [dashed,->] (c) to [in=-90,out=90] (a);
        \end{tikzpicture}
        \vspace{-6ex}
        \caption{二叉树的\alert{后}序遍历示
        例:$D\rightarrow{}G\rightarrow{}E\rightarrow{}B\rightarrow{}F\rightarrow{}C\rightarrow{}A$}
        \label{fig:demo_traverse_binary_tree_postorder}
    \end{figure}
\end{fragile}

\begin{fragile}
    \frametitle{\insertsubsectionhead}
    \begin{figure}
        \centering
        \begin{tikzpicture}[ >=stealth, thick, black!50, %
                list item/.style={draw=gray, circle, thick}, %
                level distance=5ex, %
                level/.style={sibling distance=24ex/#1}
            ]
            \node [terminal] (a) {$A$}
                child {node [terminal] (b) {$B$}
                    child {node [terminal] (d) {$D$}}
                    child {node [terminal] (e) {$E$}
                        child {node [terminal] (g) {$G$}}
                        child [missing]
                    }
                }
                child {node [terminal] (c) {$C$}
                    child [missing]
                    child {node [terminal] (f) {$F$}}
                };
            \draw [dashed,->] (a) to [in=180,out=0] (b);
            \draw [dashed,->] (b) to [in=180,out=0] (c);
            \draw [dashed,->] (c) to [in=180,out=0] (d);
            \draw [dashed,->] (d) to [in=180,out=0] (e);
            \draw [dashed,->] (e) to [in=180,out=0] (f);
            \draw [dashed,->] (f) to [in=180,out=0] (g);
        \end{tikzpicture}
        % \vspace{0ex}
        \caption{二叉树的\alert{层}序遍历示
        例:$A\rightarrow{}B\rightarrow{}C\rightarrow{}D\rightarrow{}E\rightarrow{}F\rightarrow{}G$}
        \label{fig:demo_traverse_binary_tree_levelorder}
    \end{figure}
\end{fragile}

\begin{fragile}
    \frametitle{\insertsubsectionhead}
    \begin{block}{二叉树遍历性质甲}
        \begin{itemize}
            \item 由\alert{先}序遍历与\alert{中}序遍历可推出\alert{后}序遍历
        \end{itemize}
    \end{block}
    \pause
    \bicolumns[0.5]{
        \begin{alertblock}{证明}
            \begin{enumerate}
                \item 由先序遍历性质可找出根结点
                \item 由中序遍历性质可找出左右子树
                \item 对左右子树分别递归应用上述步骤直至无左右子树\hfill$\qed$
            \end{enumerate}
        \end{alertblock}
    }{
        \begin{figure}
            \centering
            \begin{bytefield}{2}
                \bitbox[]{6}{先序:}\bitbox{2}{根}\bitbox{6}{左子树}\bitbox{6}{右子树} \\
                \\
                \bitbox[]{6}{中序:}\bitbox{6}{左子树}\bitbox{2}{根}\bitbox{6}{右子树}
            \end{bytefield}
            \caption{先序$+$中序$\Rightarrow$后序}
            \label{fig:preorder_inorder__postorder}
        \end{figure}
    }
\end{fragile}

\begin{frame}
    \frametitle{\insertsubsectionhead}
    \begin{exampleblock}{例}
        \begin{itemize}
            \item 已知先序:$A\rightarrow{}B\rightarrow{}D\rightarrow{}E\rightarrow{}G\rightarrow{}C\rightarrow{}F$
            \item 已知中序:$D\rightarrow{}B\rightarrow{}G\rightarrow{}E\rightarrow{}A\rightarrow{}C\rightarrow{}F$
            \item 求后序
        \end{itemize}
    \end{exampleblock}
    \begin{figure}
        \centering
        \begin{bytefield}{2}
            \bitbox[]{6}{先序:}\bitboxes{3}{{\alert<2->{$A$}}{\alert<3->{$B$}}{\alert<4->{$D$}}{\alert<5->{$E$}}{\alert<6->{$G$}}{\alert<7->{$C$}}{\alert<8->{$F$}}} \\
            \\
            \bitbox[]{6}{中序:}\bitboxes{3}{{\alert<4->{$D$}}{\alert<3->{$B$}}{\alert<6->{$G$}}{\alert<5->{$E$}}{\alert<2->{$A$}}{\alert<7->{$C$}}{\alert<8->{$F$}}} \\
            \\
            \bitbox[]{6}{后序:}\bitboxes{3}{{\only<4->{$D$}}{\only<6->{$G$}}{\only<5->{$E$}}{\only<3->{$B$}}{\only<8->{$F$}}{\only<7->{$C$}}{\only<2->{$A$}}}
        \end{bytefield}
        \caption{示例过程:先序$+$中序$\Rightarrow$后序}
        \label{fig:demo_preorder_inorder__postorder}
    \end{figure}
\end{frame}

\begin{fragile}
    \frametitle{\insertsubsectionhead}
    \begin{block}{二叉树遍历性质乙}
        \begin{itemize}
            \item 由\alert{后}序遍历与\alert{中}序遍历可推出\alert{先}序遍历
        \end{itemize}
    \end{block}
    \pause
    \bicolumns[0.5]{
        \begin{alertblock}{证明}
            \begin{enumerate}
                \item 与性质甲类似,略
            \end{enumerate}
        \end{alertblock}
    }{
        \begin{figure}
            \centering
            \begin{bytefield}{2}
                \bitbox[]{6}{后序:}\bitbox{6}{左子树}\bitbox{6}{右子树}\bitbox{2}{根} \\
                \\
                \bitbox[]{6}{中序:}\bitbox{6}{左子树}\bitbox{2}{根}\bitbox{6}{右子树}
            \end{bytefield}
            \caption{后序$+$中序$\Rightarrow$先序}
            \label{fig:postorder_inorder__preorder}
        \end{figure}
    }
\end{fragile}

\begin{frame}
    \frametitle{\insertsubsectionhead}
    \begin{exampleblock}{例}
        \begin{itemize}
            \item 已知后序:$D\rightarrow{}G\rightarrow{}E\rightarrow{}B\rightarrow{}F\rightarrow{}C\rightarrow{}A$
            \item 已知中序:$D\rightarrow{}B\rightarrow{}G\rightarrow{}E\rightarrow{}A\rightarrow{}C\rightarrow{}F$
            \item 求先序
        \end{itemize}
    \end{exampleblock}
    \begin{figure}
        \centering
        \begin{bytefield}{2}
            \bitbox[]{6}{后序:}\bitboxes{3}{{\alert<4->{$D$}}{\alert<6->{$G$}}{\alert<5->{$E$}}{\alert<3->{$B$}}{\alert<8->{$F$}}{\alert<7->{$C$}}{\alert<2->{$A$}}} \\
            \\
            \bitbox[]{6}{中序:}\bitboxes{3}{{\alert<4->{$D$}}{\alert<3->{$B$}}{\alert<6->{$G$}}{\alert<5->{$E$}}{\alert<2->{$A$}}{\alert<7->{$C$}}{\alert<8->{$F$}}} \\
            \\
            \bitbox[]{6}{先序:}\bitboxes{3}{{\only<2->{$A$}}{\only<3->{$B$}}{\only<4->{$D$}}{\only<5->{$E$}}{\only<6->{$G$}}{\only<7->{$C$}}{\only<8->{$F$}}}
        \end{bytefield}
        \caption{示例过程:后序$+$中序$\Rightarrow$先序}
        \label{fig:demo_postorder_inorder__preorder}
    \end{figure}
\end{frame}

\begin{fragile}
    \frametitle{\insertsubsectionhead}
    \begin{block}{二叉树遍历性质丙}
        \begin{itemize}
            \item 由\alert{先}序遍历与\alert{后}序遍历\alert{不可}推出\alert{中}序遍历
        \end{itemize}
    \end{block}
    \pause
    \bicolumns[0.5]{
        \begin{alertblock}{证明}
            \begin{enumerate}
                \item 当根结点度为$1$时无法区分左右子树
            \end{enumerate}
        \end{alertblock}
    }{
        \begin{figure}
            \centering
            \begin{bytefield}{2}
                \bitbox[]{6}{先序:}\bitbox{2}{根}\bitbox{12}{左子树$\;?\;$右子树} \\
                \\
                \bitbox[]{6}{后序:}\bitbox{12}{左子树$\;?\;$右子树}\bitbox{2}{根}
            \end{bytefield}
            \caption{先序$+$后序$\nRightarrow$中序}
            \label{fig:preorder_postorder__inorder}
        \end{figure}
    }
\end{fragile}

\begin{fragile}
    \frametitle{\insertsubsectionhead}
    \begin{block}{二叉树深度优先遍历的递归实现\footnote{须事先定
                  义:\mintinline[fontsize=\smaller]{c}{typedef void
                  (*Visit)(DataType);}}}
        \begin{columns}
            \column{0.33\textwidth}
            \begin{minted}[linenos=false]{c}
                void traverse_preorder(
                        BinaryTreeNode *p,
                        Visit v) {
                    if (!p) {
                        return; // 递归出口
                    }
                    v(p->data);
                    traverse_preorder(p->left, v);
                    traverse_preorder(p->right, v);
                }
            \end{minted}
            \column{0.33\textwidth}
            \begin{minted}[linenos=false]{c}
                void traverse_inorder(
                        BinaryTreeNode *p,
                        Visit v) {
                    if (!p) {
                        return; // 递归出口
                    }
                    traverse_inorder(p->left, v);
                    v(p->data);
                    traverse_inorder(p->right, v);
                }
            \end{minted}
            \column{0.33\textwidth}
            \begin{minted}[linenos=false]{c}
                void traverse_postorder(
                        BinaryTreeNode *p,
                        Visit v) {
                    if (!p) {
                        return; // 递归出口
                    }
                    traverse_postorder(p->left, v);
                    traverse_postorder(p->right, v);
                    v(p->data);
                }
            \end{minted}
        \end{columns}
    \end{block}
\end{fragile}

\begin{fragile}
    \frametitle{\insertsubsectionhead}
    \bicolumns[0.5]{
        \begin{block}{二叉树广度优先遍历的非递归实现}
            \begin{itemize}
                \item 用队列对每层结点按顺序缓存
                \item 根结点首先入队
                \item 当结点出队时均将其子树按顺序入队
                \item 当队列空时结束
            \end{itemize}
        \end{block}
    }{
        \begin{minted}{c}
            void traverse_binary_tree_level_order(
                    BinaryTreeNode *p, Visit v) {
                LinkedQueue *buffer = create_linked_queue();
                if (!buffer) { return; }
                push_linked_queue(buffer, p);
                while (!empty_linked_queue(buffer)) {
                    pop_linked_queue(buffer, (DataType *)(&p));
                    v(p->data);
                    if (p->left) {
                        push_linked_queue(buffer, p->left);
                    }
                    if (p->right) {
                        push_linked_queue(buffer, p->right);
                    }
                }
                destroy_linked_queue(buffer);
            }
        \end{minted}
    }
\end{fragile}

\begin{frame}
    \frametitle{\insertsubsectionhead}
    \begin{exampleblock}{应用:表达式树}
        \begin{itemize}
            \item 考虑仅包括二元运算的表达式
            \item 可将运算符作为根结点,左右操作数分别为左右子结点
            \item 操作数为叶结点,运算符为非叶结点
            \item 如此可将表达式转换为二叉树
            \item 前/中/后缀表达式分别对应二叉树的先/中/后序遍历
        \end{itemize}
    \end{exampleblock}
\end{frame}

\begin{fragile}
    \frametitle{\insertsubsectionhead}
    \bicolumns[0.5]{
        \begin{exampleblock}{例:表达式树}
            \begin{itemize}
                \item 前缀表达:\mintinline[fontsize=\smaller]{console}{+ - A / B C * D E}
                \item 中缀表达:\mintinline[fontsize=\smaller]{console}{A - B / C + D * E}
                \item 后缀表达:\mintinline[fontsize=\smaller]{console}{A B C / - D E * +}
            \end{itemize}
        \end{exampleblock}
    }{
        \begin{figure}
            \centering
            \begin{tikzpicture}[ >=stealth, thick, black!50, %
                    list item/.style={draw=gray, circle, thick}, %
                    level distance=4ex, %
                    level/.style={sibling distance=16ex/#1}
                ]
                \node [terminal] {$+$}
                    child {node [terminal] {$-$}
                        child {node [terminal] {$A$}}
                        child {node [terminal] {$/$}
                            child {node [terminal] {$B$}}
                            child {node [terminal] {$C$}}
                        }
                    }
                    child {node [terminal] {$*$}
                        child {node [terminal] {$D$}}
                        child {node [terminal] {$E$}}
                    };
            \end{tikzpicture}
            \caption{表达式树示例}
            \label{fig:demo_expression_tree}
        \end{figure}
    }
\end{fragile}