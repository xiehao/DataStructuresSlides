\section{二叉树}

\begin{fragile}
    \frametitle{\insertsectionhead}
    \begin{block}{二叉树(binary tree)}
        \begin{itemize}
            \item 度不大于$2$的有序树
            \item 子结点可按左右区分
        \end{itemize}
    \end{block}
    \begin{alertblock}{将树转化为二叉树}
        \begin{itemize}
            \item 令长子为左子结点、首个兄弟为右子结点
            \item 任何树均可按此法转化为二叉树
            \item 因二叉树的表示与运算相对方便,故树的问题均可转化为二叉树形式进
                  行研究
        \end{itemize}
    \end{alertblock}
\end{fragile}

\begin{fragile}
    \frametitle{\insertsectionhead}
    \begin{figure}
        \centering
        \begin{subfigure}[b]{0.35\textwidth}
            \centering
            \begin{tikzpicture}[ >=stealth, thick, black!50, %
                    list item/.style={draw=gray, circle, thick}, %
                    level distance=5ex, % sibling distance=5ex, %
                    level/.style={sibling distance=12ex/#1}, %
                    % edge from parent path={
                    %     (\tikzparentnode.south) .. controls +(0,-0.2) and +(0,0.2) .. (\tikzchildnode.north)
                    % }, %
                ]
                \node [terminal] {$A$}
                    child {node [terminal] {$B$}
                        child {node [terminal] {$D$}
                            child {node [terminal] {$G$}}
                            child {node [terminal] {$H$}}
                        }
                        child {node [terminal] {$E$}}
                        child {node [terminal] {$F$}
                            child {node [terminal] {$I$}}
                        }
                    }
                    child {node [terminal] {$C$}};
            \end{tikzpicture}
            \caption{树表示}
            \label{subfig:tree_representation}
        \end{subfigure}
        ~
        \begin{subfigure}[b]{0.25\textwidth}
            \centering
            \begin{tikzpicture}[ >=stealth, thick, black!50, %
                    list item/.style={draw=gray, circle, thick}, %
                    % level distance=3ex, % sibling distance=5ex, %
                ]
                \matrix [row sep=2.5ex,column sep=3ex] {
                    \node[terminal](a){$A$}; & & \\
                    \node[terminal](b){$B$}; & \node[terminal](c){$C$}; & \\
                    \node[terminal](d){$D$}; & \node[terminal](e){$E$}; & \node[terminal](f){$F$}; \\
                    \node[terminal](g){$G$}; & \node[terminal](h){$H$}; & \node[terminal](i){$I$}; \\
                };
                \draw (a) -- (b) -- (d) -- (g);
                \draw (f) -- (i);
                \draw [dashed] (b) -- (c);
                \draw [dashed] (d) -- (e) -- (f);
                \draw [dashed] (g) -- (h);
            \end{tikzpicture}
            \vspace{-1ex}
            \caption{中间表示}
            \label{subfig:middle_representation}
        \end{subfigure}
        ~
        \begin{subfigure}[b]{0.35\textwidth}
            \centering
            \begin{tikzpicture}[ >=stealth, thick, black!50, %
                    list item/.style={draw=gray, circle, thick}, %
                    level distance=3.5ex, %
                    level/.style={sibling distance=16ex/#1}, %
                    % edge from parent path={
                    %     (\tikzparentnode.south) .. controls +(0,-0.1) and +(0,0.1) .. (\tikzchildnode.north)
                    % }, %
                ]
                \node [terminal] {$A$}
                    child {node [terminal] {$B$}
                        child {node [terminal] {$D$}
                            child {node [terminal] {$G$}
                                child [missing]
                                child {node [terminal] {$H$}}
                            }
                            child {node [terminal] {$E$}
                                child [missing]
                                child {node [terminal] {$F$}
                                    child {node [terminal] {$I$}}
                                    child [missing]
                                }
                            }
                        }
                        child {node [terminal] {$C$}}
                    }
                    child [missing];
            \end{tikzpicture}
            \caption{二叉树表示}
            \label{subfig:binary_tree_representation}
        \end{subfigure}
        \caption{将树转化为二叉树}
        \label{fig:demo_tree_to_binary_tree}
    \end{figure}
\end{fragile}

\subsection{几种特殊的二叉树}

\begin{fragile}
    \frametitle{\insertsubsectionhead}
    \bicolumns[0.5]{
        \begin{block}{左斜树\footnotemark}
            \begin{itemize}
                \item 所有非叶结点均有且仅有$1$个左子树
                \item 每层均有且仅有$1$个结点
                \item 已退化为线性结构
            \end{itemize}
        \end{block}
    }{
        \begin{figure}
            \centering
            \begin{subfigure}[b]{0.3\textwidth}
                \centering
                \begin{tikzpicture}[ >=stealth, thick, black!50, %
                        list item/.style={draw=gray, circle, thick}, %
                        level distance=4ex, sibling distance=4ex, %
                    ]
                    \node [terminal] {$A$}
                        child {node [terminal] {$B$}
                            child {node [terminal] {$C$}
                                child {node [terminal] {$D$}}
                                child [missing]
                            }
                            child [missing]
                        }
                        child [missing];
                \end{tikzpicture}
                \caption{左斜树}
                \label{subfig:left_slanted_tree}
            \end{subfigure}
            ~
            \begin{subfigure}[b]{0.3\textwidth}
                \centering
                \begin{tikzpicture}[ >=stealth, thick, black!50, %
                        list item/.style={draw=gray, circle, thick}, %
                        level distance=4ex, sibling distance=4ex, %
                    ]
                    \node [terminal] {$A$}
                        child [missing]
                        child {node [terminal] {$B$}
                            child [missing]
                            child {node [terminal] {$C$}
                                child [missing]
                                child {node [terminal] {$D$}}
                            }
                        };
                \end{tikzpicture}
                \caption{右斜树}
                \label{subfig:right_slanted_tree}
            \end{subfigure}
            \caption{斜树}
            \label{fig:demo_slanted_tree}
        \end{figure}
    }

    \footnotetext{右斜树与之类似,只需将左改作右}
\end{fragile}

\begin{fragile}
    \frametitle{\insertsubsectionhead}
    \bicolumns[0.5]{
        \begin{block}{满二叉树(full binary tree)}
            \begin{itemize}
                \item 所有叶结点均在最后一层
                \item 所有非叶结点度均为$2$
            \end{itemize}
        \end{block}
        \begin{exampleblock}{性质}
            \begin{itemize}
                \item 同深度的二叉树中满二叉树结点最多
                \item 同深度的二叉树中满二叉树叶结点最多
            \end{itemize}
        \end{exampleblock}
    }{
        \begin{figure}
            \centering
                \begin{tikzpicture}[ >=stealth, thick, black!50, %
                        list item/.style={draw=gray, circle, thick}, %
                        level distance=4ex, %
                        level 1/.style={sibling distance=16ex}, %
                        level 2/.style={sibling distance=8ex}, %
                        level 3/.style={sibling distance=4ex}, %
                    ]
                    \node [terminal] {$A$}
                        child {node [terminal] {$B$}
                            child {node [terminal] {$D$}
                                child {node [terminal] {$H$}}
                                child {node [terminal] {$I$}}
                            }
                            child {node [terminal] {$E$}
                                child {node [terminal] {$J$}}
                                child {node [terminal] {$K$}}
                            }
                        }
                        child {node [terminal] {$C$}
                            child {node [terminal] {$F$}
                                child {node [terminal] {$L$}}
                                child {node [terminal] {$M$}}
                            }
                            child {node [terminal] {$G$}
                                child {node [terminal] {$N$}}
                                child {node [terminal] {$O$}}
                            }
                        };
                \end{tikzpicture}
            \caption{满二叉树}
            \label{fig:demo_full_binary_tree}
        \end{figure}
    }
\end{fragile}

\begin{fragile}
    \frametitle{\insertsubsectionhead}
        \begin{figure}
            \centering
            \begin{subfigure}[b]{0.4\textwidth}
                \centering
                \begin{tikzpicture}[ >=stealth, thick, black!50, %
                        list item/.style={draw=gray, circle, thick}, %
                        level distance=4ex, %
                        level 1/.style={sibling distance=16ex}, %
                        level 2/.style={sibling distance=8ex}, %
                        level 3/.style={sibling distance=4ex}, %
                    ]
                    \node [terminal] {$A$}
                        child {node [terminal] {$B$}
                            child {node [terminal] {$D$}
                                child {node [terminal] {$H$}}
                                child {node [terminal] {$I$}}
                            }
                            child {node [terminal] {$E$}
                                child {node [terminal] {$J$}}
                                child {node [terminal] {$K$}}
                            }
                        }
                        child {node [terminal] {$C$}
                            child {node [terminal] {$F$}
                                child {node [terminal] {$L$}}
                                child {node [terminal] {$M$}}
                            }
                            child {node [terminal] {\alert{$G$}}}
                        };
                \end{tikzpicture}
                \caption{叶结点不在同层}
                \label{subfig:non_full_binary_tree_a}
            \end{subfigure}
            ~
            \begin{subfigure}[b]{0.4\textwidth}
                \centering
                \begin{tikzpicture}[ >=stealth, thick, black!50, %
                        list item/.style={draw=gray, circle, thick}, %
                        level distance=4ex, %
                        level 1/.style={sibling distance=16ex}, %
                        level 2/.style={sibling distance=8ex}, %
                        level 3/.style={sibling distance=4ex}, %
                    ]
                    \node [terminal] {$A$}
                        child {node [terminal] {$B$}
                            child {node [terminal] {$D$}
                                child {node [terminal] {$H$}}
                                child {node [terminal] {$I$}}
                            }
                            child {node [terminal] {$E$}
                                child {node [terminal] {$J$}}
                                child {node [terminal] {$K$}}
                            }
                        }
                        child {node [terminal] {$C$}
                            child {node [terminal] {$F$}
                                child {node [terminal] {$L$}}
                                child {node [terminal] {$M$}}
                            }
                            child {node [terminal] {\alert{$G$}}
                                child {node [terminal] {$N$}}
                                child [missing]
                            }
                        };
                \end{tikzpicture}
                \caption{个别非叶结点度不为$2$}
                \label{subfig:non_full_binary_tree_b}
            \end{subfigure}
            \caption{非满二叉树}
            \label{fig:demo_non_full_binary_tree}
        \end{figure}
\end{fragile}

\begin{fragile}
    \frametitle{\insertsubsectionhead}
    \bicolumns[0.5]{
        \begin{block}{完全二叉树(proper binary tree)}
            \begin{itemize}
                \item 若去除最后一层结点,则为满二叉树
                \item 最后一层结点自左至右连续\footnotemark{}排列
            \end{itemize}
        \end{block}
        \begin{exampleblock}{性质}
            \begin{itemize}
                \item 同结点数的二叉树中完全二叉树最矮
                \item 满二叉树亦为完全二叉树的一种
            \end{itemize}
        \end{exampleblock}
    }{
        \begin{figure}
            \centering
                \begin{tikzpicture}[ >=stealth, thick, black!50, %
                        list item/.style={draw=gray, circle, thick}, %
                        level distance=4ex, %
                        level 1/.style={sibling distance=16ex}, %
                        level 2/.style={sibling distance=8ex}, %
                        level 3/.style={sibling distance=4ex}, %
                    ]
                    \node [terminal] {$A$}
                        child {node [terminal] {$B$}
                            child {node [terminal] {$D$}
                                child {node [terminal] {$H$}}
                                child {node [terminal] {$I$}}
                            }
                            child {node [terminal] {$E$}
                                child {node [terminal] {$J$}}
                                child [missing]
                            }
                        }
                        child {node [terminal] {$C$}
                            child {node [terminal] {$F$}}
                            child {node [terminal] {$G$}}
                        };
                \end{tikzpicture}
            \caption{完全二叉树}
            \label{fig:demo_proper_binary_tree}
        \end{figure}
    }

    \footnotetext{指中间无空结点}
\end{fragile}
\subsection{二叉树的性质}

\begin{frame}
    \frametitle{\insertsubsectionhead}
    \begin{block}{性质甲}
        \begin{itemize}
            \item 令$N$层二叉树第$k$层结点数为$n_{l}(k)$,则有
            \[
                1\leq{}n_{l}(k)\leq2^{k},\;k\in\mathbb{Z}\cap[0,N)
            \]
        \end{itemize}
    \end{block}
    \pause
    \begin{exampleblock}{证明}
        \begin{enumerate}
            \item 左侧不等式显然成立
            \item 右侧不等式当$k=0$时,第$0$层仅有根结点,故$n_{l}(0)=1=2^{0}$显
                  然成立
            \item 假设当$k=n<N-1$时成立,即$n_{l}(n)\leq2^{n}$,因所有结点的度均
                  不大于$2$,故
                  \[
                        n_{l}(n+1)\leq2\cdot{}n_{l}(n)\leq2\cdot2^{n}=2^{n+1}
                  \]
                  于是当$k=n+1$时,归纳假设成立\hfill$\qed$
        \end{enumerate}
    \end{exampleblock}
\end{frame}

\begin{frame}
    \frametitle{\insertsubsectionhead}
    \begin{block}{性质乙}
        \begin{itemize}
            \item 令$k$层二叉树的结点总数为$n(k)$,则有
                  \[
                        k\leq{}n(k)\leq2^{k}-1
                  \]
        \end{itemize}
    \end{block}
    \pause
    \begin{exampleblock}{证明}
        \begin{enumerate}
            \item 由性质甲,$1\leq{}n_{l}(i)\leq2^{i},\;i\in\mathbb{Z}\cap[0,k)$
            \item 经累加后,有
                  \[
                        k=\sum_{i=0}^{k-1}{1}\leq{}n(k)=\sum_{i=0}^{k-1}{n_{l}(i)}\leq\sum_{i=0}^{k-1}{2^{i}}=2^{k}-1
                  \]
                  \hfill$\qed$
        \end{enumerate}
    \end{exampleblock}
\end{frame}

\begin{frame}
    \frametitle{\insertsubsectionhead}
    \begin{alertblock}{思考}
        \begin{itemize}
            \item 满足$n(k)=k$的二叉树一定是斜树么?
            \item 满足$n(k)=2^{k}-1$的二叉树一定是满二叉树么?
        \end{itemize}
    \end{alertblock}
\end{frame}

\begin{frame}
    \frametitle{\insertsubsectionhead}
    \begin{block}{性质丙}
        \begin{itemize}
            \item 令二叉树中度为$d$的结点数为$n_{d},\;(d\in\{0,1,2\})$,则有
                  \[
                        n_{0}=n_{2}+1
                  \]
        \end{itemize}
    \end{block}
    \pause
    \begin{exampleblock}{证明}
        \begin{enumerate}
            \item 一方面,结点总数由各种度的结点构成,故总结点数$n$满足
                  \[
                        n=\sum_{d=0}^{2}{n_{d}}=n_{0}+n_{1}+n_{2}
                  \]
            \item 另一方面,每个非根结点均由$1$个结点生成,故度为$d$的结点可生成
                  $d$个结点,即
                  \[
                        n=\sum_{d=0}^{2}{d\cdot{}n_{d}}+1=n_{1}+2n_{2}+1
                  \]
            \item 综上得证\hfill$\qed$
        \end{enumerate}
    \end{exampleblock}
\end{frame}

\begin{frame}
    \frametitle{\insertsubsectionhead}
    \begin{alertblock}{思考}
        \begin{itemize}
            \item 有$n$个结点的完全二叉树有多少叶结点?
        \end{itemize}
    \end{alertblock}
    \pause
    \begin{exampleblock}{提示}
        \begin{itemize}
            \item 在完全二叉树中,度为$1$的结点数不多于$1$
            \item 当$n$为偶数时,$n_{1}=1$,$n_{0}=n/2$,$n_{2}=n/2-1$
            \item 当$n$为奇数时,$n_{1}=0$,$n_{0}=(n+1)/2$,$n_{2}=(n-1)/2$
            \item 故$n_{0}=\lceil{n/2}\rceil,\;n_{2}=\lfloor{(n-1)/2}\rfloor$
        \end{itemize}
    \end{exampleblock}
\end{frame}

\begin{frame}
    \frametitle{\insertsubsectionhead}
    \begin{block}{性质丁}
        \begin{itemize}
            \item 具有$n$个结点的完全二叉树层数$k=\lfloor\log_{2}{n}\rfloor+1$
        \end{itemize}
    \end{block}
    \pause
    \begin{exampleblock}{证明}
        \begin{enumerate}
            \item 由完全二叉树性质与性质乙可得,$k$层完全二叉树结点数$n$满足
                \[
                    2^{k-1}-1<n\leq2^{k}-1\;\text{或}\;2^{k-1}\leq{}n<2^{k}
                \]
            \item 取对数
                \[
                    k-1\leq\log_{2}{n}<k\;\text{或}\;\log_{2}{n}<k\leq\log_{2}{n}+1
                \]
            \item 注意到$k\in\mathbb{Z}$,于是得证\hfill$\qed$
        \end{enumerate}
    \end{exampleblock}
\end{frame}

\begin{fragile}
    \frametitle{\insertsubsectionhead}
    \bicolumns[0.5]{
        \begin{exampleblock}{完全二叉树按层序编号}
            \begin{itemize}
                \item 可为完全二叉树结点按层序依次编号
                \item 约定根编号为$1$,依次递增
                \item 称如此编号为$k$的结点为\textbf{结点$k$}
            \end{itemize}
        \end{exampleblock}
    }{
        \begin{figure}
            \centering
                \begin{tikzpicture}[ >=stealth, thick, black!50, %
                        list item/.style={draw=gray, circle, thick}, %
                        level distance=4ex, %
                        level 1/.style={sibling distance=16ex}, %
                        level 2/.style={sibling distance=8ex}, %
                        level 3/.style={sibling distance=4ex}, %
                    ]
                    \node [terminal] {$1$}
                        child {node [terminal] {$2$}
                            child {node [terminal] {$4$}
                                child {node [terminal] {$8$}}
                                child {node [terminal] {$9$}}
                            }
                            child {node [terminal] {$5$}
                                child {node [terminal] {$10$}}
                                child [missing]
                            }
                        }
                        child {node [terminal] {$3$}
                            child {node [terminal] {$6$}}
                            child {node [terminal] {$7$}}
                        };
                \end{tikzpicture}
            \caption{为完全二叉树按层序编号}
            \label{fig:demo_number_proper_binary_tree_by_layer_order}
        \end{figure}
    }
\end{fragile}

\begin{frame}
    \frametitle{\insertsubsectionhead}
    \begin{block}{性质戊}
        \begin{itemize}
            \item 为含有$n$个结点的完全二叉树按层序对结点编号\footnote{根结点为
                  $1$},则有
                \begin{enumerate}
                    \item 结点$k$的左右子结点序号分别为$2k$与
                        $2k+1,\;k\in\mathbb{Z}\cap[1,\lfloor{}n/2\rfloor]$
                    \item 结点$k$的父结点序号为
                        $\lfloor{}k/2\rfloor,\;k\in\mathbb{Z}\cap(1,n]$
                \end{enumerate}
        \end{itemize}
    \end{block}
    \pause
    \begin{exampleblock}{证明}
        \begin{enumerate}
            \item 考察结论$1$,当$k=1$时,显然其左右子结点序号分别为$2$与$3$,成
                  立
            \item 假设当$k=m$时成立,即结点$m$的左右子结点序号分别为$2m$与
                  $2m+1$,
                  \begin{itemize}
                    \item 因结点$m+1$的左子结点必为结点$m$的右子结点的后继
                    \item 故结点$m+1$的左子结点序号为$(2m+1)+1=2(m+1)$
                    \item 且结点$m+1$的右子结点序号为$2(m+1)+1$
                  \end{itemize}
                  则当$k=m+1$时,假设成立,故结论$1$成立
            \item 由结论$1$知结论$2$成立\hfill\qed
        \end{enumerate}
    \end{exampleblock}
\end{frame}
\subsection{二叉树的存储结构}

\begin{frame}
    \frametitle{\insertsubsectionhead}
    \begin{block}{顺序存储}
        \begin{itemize}
            \item 按完全二叉树层序编号方式为二叉树编号,跳过不存在结点的编号
            \item 以静态数组方式存储,留空不存在的结点
        \end{itemize}
    \end{block}
\end{frame}

\begin{frame}
    \frametitle{\insertsubsectionhead}
    \begin{figure}
        \centering
        \begin{subfigure}[b]{0.45\textwidth}
            \centering
            \begin{tikzpicture}[ >=stealth, thick, black!50, %
                    list item/.style={draw=gray, circle, thick}, %
                    level distance=4ex, %
                    level 1/.style={sibling distance=16ex}, %
                    level 2/.style={sibling distance=8ex}, %
                    level 3/.style={sibling distance=4ex}, %
                ]
                \node [terminal] {$A$}
                    child {node [terminal] {$B$}
                        child {node [terminal] {$D$}}
                        child {node [terminal] {$E$}
                            child {node [terminal] {$G$}}
                            child [missing]
                        }
                    }
                    child {node [terminal] {$C$}
                        child [missing]
                        child {node [terminal] {$F$}}
                    };
            \end{tikzpicture}
            \caption{逻辑结构}
            \label{subfig:demo_logistic_structure_binary_tree_sequence_storage}
        \end{subfigure}
        ~
        \begin{subfigure}[b]{0.45\textwidth}
            \centering
            \begin{bytefield}{2}
                % \foreach \i in {0,...,10} {\bitbox[]{2}{\tiny{\i}}} \\
                \bitboxes[]{2}{{\tiny{$0$}}{\tiny{$1$}}{\tiny{$2$}}{\tiny{$3$}}{\tiny{$4$}}{\tiny{$5$}}{\tiny{$6$}}{\tiny{$7$}}{\tiny{$8$}}{\tiny{$9$}}{\tiny{$10$}}} \\
                \bitboxes{2}{{}{$A$}{$B$}{$C$}{$D$}{$E$}{}{$F$}{}{}{$G$}}
            \end{bytefield}
            \caption{物理结构}
            \label{subfig:demo_physical_structure_binary_tree_sequence_storage}
        \end{subfigure}
        \caption{二叉树的顺序存储示例}
        \label{fig:demo_sequence_storage_binary_tree}
    \end{figure}
\end{frame}

\begin{frame}
    \frametitle{\insertsubsectionhead}
    \begin{figure}
        \centering
        \begin{subfigure}[b]{0.35\textwidth}
            \centering
            \begin{tikzpicture}[ >=stealth, thick, black!50, %
                    list item/.style={draw=gray, circle, thick}, %
                    level distance=4ex, %
                    level 1/.style={sibling distance=16ex}, %
                    level 2/.style={sibling distance=8ex}, %
                    level 3/.style={sibling distance=4ex}, %
                ]
                \node [terminal] {$A$}
                    child [missing]
                    child {node [terminal] {$B$}
                        child [missing]
                        child {node [terminal] {$C$}
                            child [missing]
                            child {node [terminal] {$D$}}}
                    };
            \end{tikzpicture}
            \caption{逻辑结构}
            \label{subfig:demo_logistic_structure_slanted_tree_sequence_storage}
        \end{subfigure}
        ~
        \begin{subfigure}[b]{0.55\textwidth}
            \centering
            \begin{bytefield}{2}
                % \foreach \i in {0,...,10} {\bitbox[]{2}{\tiny{\i}}} \\
                \bitboxes[]{2}{{\tiny{$0$}}{\tiny{$1$}}{\tiny{$2$}}{\tiny{$3$}}{\tiny{$4$}}{\tiny{$5$}}{\tiny{$6$}}{\tiny{$7$}}{\tiny{$8$}}{\tiny{$9$}}{\tiny{$10$}}{\tiny{$11$}}{\tiny{$12$}}{\tiny{$13$}}{\tiny{$14$}}{\tiny{$15$}}} \\
                \bitboxes{2}{{}{$A$}{}{$B$}{}{}{}{$C$}{}{}{}{}{}{}{}{$D$}}
            \end{bytefield}
            \caption{物理结构}
            \label{subfig:demo_physical_structure_slanted_tree_sequence_storage}
        \end{subfigure}
        \caption{右斜树的顺序存储示例}
        \label{fig:demo_sequence_storage_slanted_tree}
    \end{figure}
\end{frame}

\begin{frame}
    \frametitle{\insertsubsectionhead}
    \begin{exampleblock}{顺序存储的特点}
        \begin{itemize}
            \item 可利用性质戊快速访问各结点:$O(1)$
            \item 增删结点可能需要大幅调整存储
            \item 在存储含有稀疏结点的二叉树时需耗费大量存储空间
            \item 仅适合存储含有稠密结点的完全二叉树
        \end{itemize}
    \end{exampleblock}
\end{frame}

\begin{fragile}
    \frametitle{\insertsubsectionhead}
    \bicolumns[0.5]{
        \begin{block}{链式存储}
            \begin{itemize}
                \item 采用二/三叉链表表示结点
                \item 结点除存信息包括
                    \begin{itemize}
                        \item 数据信息
                        \item 左、右子结点地址
                        \item 可选的父结点地址
                    \end{itemize}
                \item 为后续处理方便,设置虚拟首结点
            \end{itemize}
        \end{block}
        \begin{figure}
            \centering
            \begin{bytefield}[boxformatting=\baselinealign]{2}
                \bitbox{3}{\mintinline[fontsize=\smaller]{c}{data}}
                \bitbox{3}{\mintinline[fontsize=\smaller]{c}{left}}
                \bitbox{4}{\mintinline[fontsize=\smaller]{c}{right}}
                \bitbox{4}{\mintinline[fontsize=\smaller]{console}{parent}}
            \end{bytefield}
            \caption{二叉树结点的链式存储结构}
            \label{fig:linked_node_binary_tree}
        \end{figure}
    }{
        \begin{minted}{c}
            typedef struct BinaryTreeNode {
                DataType data; // 数据信息
                struct BinaryTreeNode *left; // 左子结点地址
                struct BinaryTreeNode *right; // 右子结点地址
                struct BinaryTreeNode *parent; // 可选的父结点地址
            } BinaryTreeNode;
        \end{minted}
        \begin{minted}{c}
            typedef struct {
                BinaryTreeNode *head; // 虚拟首结点
            } BinaryTree;
        \end{minted}
    }
\end{fragile}

\begin{fragile}
    \frametitle{\insertsubsectionhead}
    \begin{figure}
        \centering
        \begin{subfigure}[b]{0.3\textwidth}
            \centering
            \begin{tikzpicture}[ >=stealth, thick, black!50, %
                    list item/.style={draw=gray, circle, thick}, %
                    level distance=4ex, %
                    level/.style={sibling distance=12ex/#1}
                ]
                \node [terminal] {$A$}
                    child {node [terminal] {$B$}
                        child {node [terminal] {$D$}}
                        child {node [terminal] {$E$}
                            child {node [terminal] {$G$}}
                            child [missing]
                        }
                    }
                    child {node [terminal] {$C$}
                        child [missing]
                        child {node [terminal] {$F$}}
                    };
            \end{tikzpicture}
            \caption{逻辑结构}
            \label{subfig:demo_logistic_structure_binary_tree_linked_storage}
        \end{subfigure}
        ~
        \begin{subfigure}[b]{0.6\textwidth}
            \centering
            \begin{bytefield}{2}
                \bitbox[]{2}{\tikzmark{tchu}} \\
                \bitbox{2}{首}\bitboxes{1}{{\tikzmark{scha}}{\tikzmark{schx}}{\tikzmark{schy}}}
                \bitbox[]{9}{}\bitbox[]{1}{\tikzmark{tcal}}
                \bitbox{2}{$A$}\bitboxes{1}{{\tikzmark{scab}}{\tikzmark{scac}}{\tikzmark{scah}}} \\
                \bitbox[]{5}{}
                \bitbox[]{2}{\tikzmark{tcbu}}\bitbox[]{8}{}
                \bitbox[]{2}{\tikzmark{tcad}}\bitbox[]{8}{}
                \bitbox[]{2}{\tikzmark{tccu}} \\
                \bitbox[]{5}{}
                \bitbox{2}{$B$}\bitboxes{1}{{\tikzmark{scbd}}{\tikzmark{scbe}}{\tikzmark{scba}}}
                \bitbox[]{15}{}
                \bitbox{2}{$C$}\bitboxes{1}{{\tikzmark{sccx}}{\tikzmark{sccf}}{\tikzmark{scca}}} \\
                \bitbox[]{2}{\tikzmark{tcdu}}\bitbox[]{3}{}\bitbox[]{2}{\tikzmark{tcbd}}\bitbox[]{3}{}
                \bitbox[]{2}{\tikzmark{tceu}}\bitbox[]{13}{}\bitbox[]{2}{\tikzmark{tccd}}\bitbox[]{3}{}
                \bitbox[]{2}{\tikzmark{tcfu}} \\
                \bitbox{2}{$D$}\bitboxes{1}{{\tikzmark{scdx}}{\tikzmark{scdy}}{\tikzmark{scdb}}}
                \bitbox[]{5}{}
                \bitbox{2}{$E$}\bitboxes{1}{{\tikzmark{sceg}}{\tikzmark{scex}}{\tikzmark{sceb}}}
                \bitbox[]{15}{}
                \bitbox{2}{$F$}\bitboxes{1}{{\tikzmark{scfx}}{\tikzmark{scfy}}{\tikzmark{scfc}}} \\
                \bitbox[]{8}{}
                \bitbox[]{2}{\tikzmark{tcgu}}\bitbox[]{2}{\tikzmark{tced}} \\
                \bitbox[]{8}{}
                \bitbox{2}{$G$}\bitboxes{1}{{\tikzmark{scgx}}{\tikzmark{scgy}}{\tikzmark{scge}}}
            \end{bytefield}
            \begin{tikzpicture}[ >=stealth, thick, black!50, overlay, remember picture, %
                    list item/.style={draw=gray, rounded corners, thick} %
                ]
                \draw[->] ($(pic cs:scha)+(0,-0.03)$) to [out=90,in=180] ($(pic cs:tcal)+(0.1,-0.03)$);
                \fill ($(pic cs:scha)+(0,-0.03)$) circle [radius=0.05];
                \draw[->] ($(pic cs:scab)+(0,-0.03)$) to [out=-90,in=90] ($(pic cs:tcbu)+(0,-0.3)$);
                \fill ($(pic cs:scab)+(0,-0.03)$) circle [radius=0.05];
                \draw[->] ($(pic cs:scac)+(0,-0.03)$) to [out=-90,in=90] ($(pic cs:tccu)+(0,-0.3)$);
                \fill ($(pic cs:scac)+(0,-0.03)$) circle [radius=0.05];
                \draw[->] ($(pic cs:scbd)+(0,-0.03)$) to [out=-90,in=90] ($(pic cs:tcdu)+(0,-0.3)$);
                \fill ($(pic cs:scbd)+(0,-0.03)$) circle [radius=0.05];
                \draw[->] ($(pic cs:scbe)+(0,-0.03)$) to [out=-90,in=90] ($(pic cs:tceu)+(0,-0.3)$);
                \fill ($(pic cs:scbe)+(0,-0.03)$) circle [radius=0.05];
                \draw[->] ($(pic cs:sccf)+(0,-0.03)$) to [out=-90,in=90] ($(pic cs:tcfu)+(0,-0.3)$);
                \fill ($(pic cs:sccf)+(0,-0.03)$) circle [radius=0.05];
                \draw[->] ($(pic cs:sceg)+(0,-0.03)$) to [out=-90,in=90] ($(pic cs:tcgu)+(0,-0.3)$);
                \fill ($(pic cs:sceg)+(0,-0.03)$) circle [radius=0.05];
                \draw[dashed,->] ($(pic cs:scah)+(0,-0.03)$) to [out=90,in=90] ($(pic cs:tchu)+(0,-0.3)$);
                \fill ($(pic cs:scah)+(0,-0.03)$) circle [radius=0.05];
                \draw[dashed,->] ($(pic cs:scba)+(0,-0.03)$) to [out=90,in=-90] ($(pic cs:tcad)+(0,0.23)$);
                \fill ($(pic cs:scba)+(0,-0.03)$) circle [radius=0.05];
                \draw[dashed,->] ($(pic cs:scca)+(0,-0.03)$) to [out=90,in=-90] ($(pic cs:tcad)+(0,0.23)$);
                \fill ($(pic cs:scca)+(0,-0.03)$) circle [radius=0.05];
                \draw[dashed,->] ($(pic cs:scdb)+(0,-0.03)$) to [out=90,in=-90] ($(pic cs:tcbd)+(0,0.23)$);
                \fill ($(pic cs:scdb)+(0,-0.03)$) circle [radius=0.05];
                \draw[dashed,->] ($(pic cs:sceb)+(0,-0.03)$) to [out=90,in=-90] ($(pic cs:tcbd)+(0,0.23)$);
                \fill ($(pic cs:sceb)+(0,-0.03)$) circle [radius=0.05];
                \draw[dashed,->] ($(pic cs:scfc)+(0,-0.03)$) to [out=90,in=-90] ($(pic cs:tccd)+(0,0.23)$);
                \fill ($(pic cs:scfc)+(0,-0.03)$) circle [radius=0.05];
                \draw[dashed,->] ($(pic cs:scge)+(0,-0.03)$) to [out=90,in=-90] ($(pic cs:tced)+(0,0.23)$);
                \fill ($(pic cs:scge)+(0,-0.03)$) circle [radius=0.05];
                \draw ($(pic cs:schx)+(-0.1,-0.3)$) to ($(pic cs:schx)+(0.1,0.23)$);
                \draw ($(pic cs:schy)+(-0.1,-0.3)$) to ($(pic cs:schy)+(0.1,0.23)$);
                \draw ($(pic cs:sccx)+(-0.1,-0.3)$) to ($(pic cs:sccx)+(0.1,0.23)$);
                \draw ($(pic cs:scdx)+(-0.1,-0.3)$) to ($(pic cs:scdx)+(0.1,0.23)$);
                \draw ($(pic cs:scdy)+(-0.1,-0.3)$) to ($(pic cs:scdy)+(0.1,0.23)$);
                \draw ($(pic cs:scex)+(-0.1,-0.3)$) to ($(pic cs:scex)+(0.1,0.23)$);
                \draw ($(pic cs:scfx)+(-0.1,-0.3)$) to ($(pic cs:scfx)+(0.1,0.23)$);
                \draw ($(pic cs:scfy)+(-0.1,-0.3)$) to ($(pic cs:scfy)+(0.1,0.23)$);
                \draw ($(pic cs:scgx)+(-0.1,-0.3)$) to ($(pic cs:scgx)+(0.1,0.23)$);
                \draw ($(pic cs:scgy)+(-0.1,-0.3)$) to ($(pic cs:scgy)+(0.1,0.23)$);
            \end{tikzpicture}
            \vspace{-2ex}
            \caption{物理结构}
            \label{subfig:demo_physical_structure_binary_tree_linked_storage}
        \end{subfigure}
        \caption{二叉树的链式存储示例}
        \label{fig:demo_linked_storage_binary_tree}
    \end{figure}
\end{fragile}
\subsection{二叉树的基本操作}

\begin{fragile}
    \frametitle{\insertsubsectionhead}
    \begin{block}{二叉树的抽象数据类型}
        \begin{minted}[linenos=false,escapeinside=@@]{c}
            ADT BinaryTree {
                数据:
                    数据对象: @$\mathcal{D} = \{a_{k} | a_{k}\in\text{数据元素集合}, k\in\mathbb{Z}\cap[0,n)\}$@
                    逻辑关系: 若@$\mathcal{D}=\varnothing$@, 则@$\mathcal{R}=\varnothing$@; 否则@$\mathcal{R}=\{\langle{}a_{i},a_{j}\rangle | i<j;i,j\in\mathbb{Z}\cap[0,n);a_{i},a_{j}\in\mathcal{D}_{l}\cup\{a_{0}\}\text{或}\mathcal{D}_{r}\cup\{a_{0}\}\}$\footnotemark@
                操作:
                    create_binary_tree()
                        构造并初始化一个空二叉树@$t$@
                    insert_left_binary_tree(p, d)
                        在二叉树中结点@$p$@下插入值为@$d$@的左子结点, 右子结点可简单类比, 略
                    remove_left_binary_tree(p)
                        在二叉树中删除结点@$p$@的左子树, 右子树可简单类比, 略
                    traverse_binary_tree(t)
                        以某种方式遍历二叉树@$t$@的所有结点
            }
        \end{minted}
    \end{block}

    \footnotetext{$a_{0}$为根结点,$\mathcal{D}_{l}$与$\mathcal{D}_{r}$分别为其
                  左右子树结点集合,且$\mathcal{D}_{l}\cap\mathcal{D}_{r}=\varnothing$}
\end{fragile}

\begin{fragile}
    \frametitle{\insertsubsectionhead}
    \bicolumns[0.5]{
        \begin{block}{二叉树的初始化}
            \begin{itemize}
                \item 利用结点初始化方法
                \item 注意空指针处理
            \end{itemize}
        \end{block}
    }{
        \begin{minted}{c}
            BinaryTreeNode *create_binary_tree_node(DataType d) {
                BinaryTreeNode *n = 
                        malloc(sizeof(BinaryTreeNode));
                if (n) {
                    n->data = d;
                    n->left = n->right = n->parent = NULL;
                }
                return n;
            }
        \end{minted}
        \begin{minted}{c}
            BinaryTree *create_binary_tree() {
                BinaryTree *t = malloc(sizeof(BinaryTree));
                if (t) {
                    t->head = create_binary_tree_node(0);
                }
                return t;
            }
        \end{minted}
    }
\end{fragile}

\begin{fragile}
    \frametitle{\insertsubsectionhead}
    \bicolumns[0.5]{
        \begin{block}{为二叉树中的指定结点插入左子结点\footnotemark}
            \begin{itemize}
                \item 将原左子树作为新结点的左子树
                \item 注意更新双向链接关系的顺序
                \item 可将操作拆分为两部分
                \item 类似于双向链表的结点插入
            \end{itemize}
        \end{block}
    }{
        \begin{minted}[highlightlines={5,7}]{c}
            BinaryTreeNode *attach_left_binary_tree(
                    BinaryTreeNode *p, BinaryTreeNode *n) {
                assert(n && p);
                if (p->left) {
                    n->left = p->left, p->left->parent = n;
                }
                p->left = n, n->parent = p;
                return n;
            }
        \end{minted}
        \begin{minted}{c}
            bool insert_left_binary_tree(
                    BinaryTreeNode *p, DataType d) {
                if (!p) {
                    printf("Wrong insertion place!\n");
                    return false;
                }
                BinaryTreeNode *n = create_binary_tree_node(d);
                return n && attach_left_binary_tree(p, n);
            }
        \end{minted}
    }

    \footnotetext{右子结点可直接类比,略,下同}
\end{fragile}

\begin{fragile}
    \frametitle{\insertsubsectionhead}
    \bicolumns[0.5]{
        \begin{block}{删除二叉树中的指定结点的左子树}
            \begin{itemize}
                \item 注意更新双向链接关系的顺序
                \item 可将操作拆分为两部分
                \item 拆除的子树须通过遍历释放
            \end{itemize}
        \end{block}
    }{
        \begin{minted}[highlightlines={5}]{c}
            BinaryTreeNode *detach_left_binary_tree(
                    BinaryTreeNode *p) {
                assert(p && p->left);
                BinaryTreeNode *c = p->left;
                p->left = c->parent = NULL;
                return c;
            }
        \end{minted}
        \begin{minted}{c}
            bool remove_left_binary_tree(BinaryTreeNode *p) {
                if (!p) {
                    printf("Wrong removal place!\n");
                    return false;
                }
                BinaryTreeNode *c = detach_left_binary_tree(p);
                return cleanup_binary_tree_by_node(c);
            }
        \end{minted}
    }
\end{fragile}
\subsection{二叉树的遍历}

\begin{frame}
    \frametitle{\insertsubsectionhead}
    \begin{block}{遍历(traversal)}
        \begin{itemize}
            \item 按某种约定顺序访问半线性结构中的所有结点
            \item 每个结点均\alert{被且仅被}访问$1$次
            \item \alert{意义}:使半线性结构转化为线性结构
            \item 两类常见遍历方式:\alert{深度}优先与\alert{广度}优先
            \item 前者可按访问\alert{根}结点的次序区分
                \begin{itemize}
                    \item \textbf{先序(preorder)遍历}:\alert{根}结点
                        $\Rightarrow$子树序列\footnote{按顺序遍历每个子树,遍历
                        方式亦为递归相同遍历方式,其余类同}
                    \item \textbf{中序(inorder)遍历}\footnote{仅针对二叉树}:左
                        子树$\Rightarrow$\alert{根}结点$\Rightarrow$右子树
                    \item \textbf{后序(postorder)遍历}:子树序列
                        $\Rightarrow$\alert{根}结点
                \end{itemize}
            \item 后者包括\textbf{层序(level order)遍历}
        \end{itemize}
    \end{block}
\end{frame}

\begin{fragile}
    \frametitle{\insertsubsectionhead}
    \begin{figure}
        \centering
        \begin{tikzpicture}[ >=stealth, thick, black!50, %
                list item/.style={draw=gray, circle, thick}, %
                level distance=5ex, %
                level/.style={sibling distance=24ex/#1}
            ]
            \node [terminal] (a) {$A$}
                child {node [terminal] (b) {$B$}
                    child {node [terminal] (d) {$D$}}
                    child {node [terminal] (e) {$E$}
                        child {node [terminal] (g) {$G$}}
                        child [missing]
                    }
                }
                child {node [terminal] (c) {$C$}
                    child [missing]
                    child {node [terminal] (f) {$F$}}
                };
            \draw [dashed,->] (a) to [in=90,out=-90] (b);
            \draw [dashed,->] (b) to [in=90,out=-90] (d);
            \draw [dashed,->] (d) to [in=90,out=-90] (e);
            \draw [dashed,->] (e) to [in=90,out=-90] (g);
            \draw [dashed,->] (g) to [in=90,out=-90] (c);
            \draw [dashed,->] (c) to [in=90,out=-90] (f);
        \end{tikzpicture}
        \vspace{-8ex}
        \caption{二叉树的\alert{先}序遍历示
        例:$A\rightarrow{}B\rightarrow{}D\rightarrow{}E\rightarrow{}G\rightarrow{}C\rightarrow{}F$}
        \label{fig:demo_traverse_binary_tree_preorder}
    \end{figure}
\end{fragile}

\begin{fragile}
    \frametitle{\insertsubsectionhead}
    \begin{figure}
        \centering
        \begin{tikzpicture}[ >=stealth, thick, black!50, %
                list item/.style={draw=gray, circle, thick}, %
                level distance=5ex, %
                level/.style={sibling distance=24ex/#1}
            ]
            \node [terminal] (a) {$A$}
                child {node [terminal] (b) {$B$}
                    child {node [terminal] (d) {$D$}}
                    child {node [terminal] (e) {$E$}
                        child {node [terminal] (g) {$G$}}
                        child [missing]
                    }
                }
                child {node [terminal] (c) {$C$}
                    child [missing]
                    child {node [terminal] (f) {$F$}}
                };
            \draw [dashed,->] (d) to [in=180,out=0] (b);
            \draw [dashed,->] (b) to [in=180,out=0] (g);
            \draw [dashed,->] (g) to [in=180,out=0] (e);
            \draw [dashed,->] (e) to [in=180,out=0] (a);
            \draw [dashed,->] (a) to [in=180,out=0] (c);
            \draw [dashed,->] (c) to [in=180,out=0] (f);
        \end{tikzpicture}
        % \vspace{0ex}
        \caption{二叉树的\alert{中}序遍历示
        例:$D\rightarrow{}B\rightarrow{}G\rightarrow{}E\rightarrow{}A\rightarrow{}C\rightarrow{}F$}
        \label{fig:demo_traverse_binary_tree_inorder}
    \end{figure}
\end{fragile}

\begin{fragile}
    \frametitle{\insertsubsectionhead}
    \begin{figure}
        \centering
        \begin{tikzpicture}[ >=stealth, thick, black!50, %
                list item/.style={draw=gray, circle, thick}, %
                level distance=5ex, %
                level/.style={sibling distance=24ex/#1}
            ]
            \node [terminal] (a) {$A$}
                child {node [terminal] (b) {$B$}
                    child {node [terminal] (d) {$D$}}
                    child {node [terminal] (e) {$E$}
                        child {node [terminal] (g) {$G$}}
                        child [missing]
                    }
                }
                child {node [terminal] (c) {$C$}
                    child [missing]
                    child {node [terminal] (f) {$F$}}
                };
            \draw [dashed,->] (d) to [in=-90,out=90] (g);
            \draw [dashed,->] (g) to [in=-90,out=90] (e);
            \draw [dashed,->] (e) to [in=-90,out=90] (b);
            \draw [dashed,->] (b) to [in=-90,out=90] (f);
            \draw [dashed,->] (f) to [in=-90,out=90] (c);
            \draw [dashed,->] (c) to [in=-90,out=90] (a);
        \end{tikzpicture}
        \vspace{-6ex}
        \caption{二叉树的\alert{后}序遍历示
        例:$D\rightarrow{}G\rightarrow{}E\rightarrow{}B\rightarrow{}F\rightarrow{}C\rightarrow{}A$}
        \label{fig:demo_traverse_binary_tree_postorder}
    \end{figure}
\end{fragile}

\begin{fragile}
    \frametitle{\insertsubsectionhead}
    \begin{figure}
        \centering
        \begin{tikzpicture}[ >=stealth, thick, black!50, %
                list item/.style={draw=gray, circle, thick}, %
                level distance=5ex, %
                level/.style={sibling distance=24ex/#1}
            ]
            \node [terminal] (a) {$A$}
                child {node [terminal] (b) {$B$}
                    child {node [terminal] (d) {$D$}}
                    child {node [terminal] (e) {$E$}
                        child {node [terminal] (g) {$G$}}
                        child [missing]
                    }
                }
                child {node [terminal] (c) {$C$}
                    child [missing]
                    child {node [terminal] (f) {$F$}}
                };
            \draw [dashed,->] (a) to [in=180,out=0] (b);
            \draw [dashed,->] (b) to [in=180,out=0] (c);
            \draw [dashed,->] (c) to [in=180,out=0] (d);
            \draw [dashed,->] (d) to [in=180,out=0] (e);
            \draw [dashed,->] (e) to [in=180,out=0] (f);
            \draw [dashed,->] (f) to [in=180,out=0] (g);
        \end{tikzpicture}
        % \vspace{0ex}
        \caption{二叉树的\alert{层}序遍历示
        例:$A\rightarrow{}B\rightarrow{}C\rightarrow{}D\rightarrow{}E\rightarrow{}F\rightarrow{}G$}
        \label{fig:demo_traverse_binary_tree_levelorder}
    \end{figure}
\end{fragile}

\begin{fragile}
    \frametitle{\insertsubsectionhead}
    \begin{block}{二叉树遍历性质甲}
        \begin{itemize}
            \item 由\alert{先}序遍历与\alert{中}序遍历可推出\alert{后}序遍历
        \end{itemize}
    \end{block}
    \pause
    \bicolumns[0.5]{
        \begin{alertblock}{证明}
            \begin{enumerate}
                \item 由先序遍历性质可找出根结点
                \item 由中序遍历性质可找出左右子树
                \item 对左右子树分别递归应用上述步骤直至无左右子树\hfill$\qed$
            \end{enumerate}
        \end{alertblock}
    }{
        \begin{figure}
            \centering
            \begin{bytefield}{2}
                \bitbox[]{6}{先序:}\bitbox{2}{根}\bitbox{6}{左子树}\bitbox{6}{右子树} \\
                \\
                \bitbox[]{6}{中序:}\bitbox{6}{左子树}\bitbox{2}{根}\bitbox{6}{右子树}
            \end{bytefield}
            \caption{先序$+$中序$\Rightarrow$后序}
            \label{fig:preorder_inorder__postorder}
        \end{figure}
    }
\end{fragile}

\begin{frame}
    \frametitle{\insertsubsectionhead}
    \begin{exampleblock}{例}
        \begin{itemize}
            \item 已知先序:$A\rightarrow{}B\rightarrow{}D\rightarrow{}E\rightarrow{}G\rightarrow{}C\rightarrow{}F$
            \item 已知中序:$D\rightarrow{}B\rightarrow{}G\rightarrow{}E\rightarrow{}A\rightarrow{}C\rightarrow{}F$
            \item 求后序
        \end{itemize}
    \end{exampleblock}
    \begin{figure}
        \centering
        \begin{bytefield}{2}
            \bitbox[]{6}{先序:}\bitboxes{3}{{\alert<2->{$A$}}{\alert<3->{$B$}}{\alert<4->{$D$}}{\alert<5->{$E$}}{\alert<6->{$G$}}{\alert<7->{$C$}}{\alert<8->{$F$}}} \\
            \\
            \bitbox[]{6}{中序:}\bitboxes{3}{{\alert<4->{$D$}}{\alert<3->{$B$}}{\alert<6->{$G$}}{\alert<5->{$E$}}{\alert<2->{$A$}}{\alert<7->{$C$}}{\alert<8->{$F$}}} \\
            \\
            \bitbox[]{6}{后序:}\bitboxes{3}{{\only<4->{$D$}}{\only<6->{$G$}}{\only<5->{$E$}}{\only<3->{$B$}}{\only<8->{$F$}}{\only<7->{$C$}}{\only<2->{$A$}}}
        \end{bytefield}
        \caption{示例过程:先序$+$中序$\Rightarrow$后序}
        \label{fig:demo_preorder_inorder__postorder}
    \end{figure}
\end{frame}

\begin{fragile}
    \frametitle{\insertsubsectionhead}
    \begin{block}{二叉树遍历性质乙}
        \begin{itemize}
            \item 由\alert{后}序遍历与\alert{中}序遍历可推出\alert{先}序遍历
        \end{itemize}
    \end{block}
    \pause
    \bicolumns[0.5]{
        \begin{alertblock}{证明}
            \begin{enumerate}
                \item 与性质甲类似,略
            \end{enumerate}
        \end{alertblock}
    }{
        \begin{figure}
            \centering
            \begin{bytefield}{2}
                \bitbox[]{6}{后序:}\bitbox{6}{左子树}\bitbox{6}{右子树}\bitbox{2}{根} \\
                \\
                \bitbox[]{6}{中序:}\bitbox{6}{左子树}\bitbox{2}{根}\bitbox{6}{右子树}
            \end{bytefield}
            \caption{后序$+$中序$\Rightarrow$先序}
            \label{fig:postorder_inorder__preorder}
        \end{figure}
    }
\end{fragile}

\begin{frame}
    \frametitle{\insertsubsectionhead}
    \begin{exampleblock}{例}
        \begin{itemize}
            \item 已知后序:$D\rightarrow{}G\rightarrow{}E\rightarrow{}B\rightarrow{}F\rightarrow{}C\rightarrow{}A$
            \item 已知中序:$D\rightarrow{}B\rightarrow{}G\rightarrow{}E\rightarrow{}A\rightarrow{}C\rightarrow{}F$
            \item 求先序
        \end{itemize}
    \end{exampleblock}
    \begin{figure}
        \centering
        \begin{bytefield}{2}
            \bitbox[]{6}{后序:}\bitboxes{3}{{\alert<4->{$D$}}{\alert<6->{$G$}}{\alert<5->{$E$}}{\alert<3->{$B$}}{\alert<8->{$F$}}{\alert<7->{$C$}}{\alert<2->{$A$}}} \\
            \\
            \bitbox[]{6}{中序:}\bitboxes{3}{{\alert<4->{$D$}}{\alert<3->{$B$}}{\alert<6->{$G$}}{\alert<5->{$E$}}{\alert<2->{$A$}}{\alert<7->{$C$}}{\alert<8->{$F$}}} \\
            \\
            \bitbox[]{6}{先序:}\bitboxes{3}{{\only<2->{$A$}}{\only<3->{$B$}}{\only<4->{$D$}}{\only<5->{$E$}}{\only<6->{$G$}}{\only<7->{$C$}}{\only<8->{$F$}}}
        \end{bytefield}
        \caption{示例过程:后序$+$中序$\Rightarrow$先序}
        \label{fig:demo_postorder_inorder__preorder}
    \end{figure}
\end{frame}

\begin{fragile}
    \frametitle{\insertsubsectionhead}
    \begin{block}{二叉树遍历性质丙}
        \begin{itemize}
            \item 由\alert{先}序遍历与\alert{后}序遍历\alert{不可}推出\alert{中}序遍历
        \end{itemize}
    \end{block}
    \pause
    \bicolumns[0.5]{
        \begin{alertblock}{证明}
            \begin{enumerate}
                \item 当根结点度为$1$时无法区分左右子树
            \end{enumerate}
        \end{alertblock}
    }{
        \begin{figure}
            \centering
            \begin{bytefield}{2}
                \bitbox[]{6}{先序:}\bitbox{2}{根}\bitbox{12}{左子树$\;?\;$右子树} \\
                \\
                \bitbox[]{6}{后序:}\bitbox{12}{左子树$\;?\;$右子树}\bitbox{2}{根}
            \end{bytefield}
            \caption{先序$+$后序$\nRightarrow$中序}
            \label{fig:preorder_postorder__inorder}
        \end{figure}
    }
\end{fragile}

\begin{fragile}
    \frametitle{\insertsubsectionhead}
    \begin{block}{二叉树深度优先遍历的递归实现\footnote{须事先定
                  义:\mintinline[fontsize=\smaller]{c}{typedef void
                  (*Visit)(DataType);}}}
        \begin{columns}
            \column{0.33\textwidth}
            \begin{minted}[linenos=false]{c}
                void traverse_preorder(
                        BinaryTreeNode *p,
                        Visit v) {
                    if (!p) {
                        return; // 递归出口
                    }
                    v(p->data);
                    traverse_preorder(p->left, v);
                    traverse_preorder(p->right, v);
                }
            \end{minted}
            \column{0.33\textwidth}
            \begin{minted}[linenos=false]{c}
                void traverse_inorder(
                        BinaryTreeNode *p,
                        Visit v) {
                    if (!p) {
                        return; // 递归出口
                    }
                    traverse_inorder(p->left, v);
                    v(p->data);
                    traverse_inorder(p->right, v);
                }
            \end{minted}
            \column{0.33\textwidth}
            \begin{minted}[linenos=false]{c}
                void traverse_postorder(
                        BinaryTreeNode *p,
                        Visit v) {
                    if (!p) {
                        return; // 递归出口
                    }
                    traverse_postorder(p->left, v);
                    traverse_postorder(p->right, v);
                    v(p->data);
                }
            \end{minted}
        \end{columns}
    \end{block}
\end{fragile}

\begin{fragile}
    \frametitle{\insertsubsectionhead}
    \bicolumns[0.5]{
        \begin{block}{二叉树广度优先遍历的非递归实现}
            \begin{itemize}
                \item 用队列对每层结点按顺序缓存
                \item 根结点首先入队
                \item 当结点出队时均将其子树按顺序入队
                \item 当队列空时结束
            \end{itemize}
        \end{block}
    }{
        \begin{minted}{c}
            void traverse_binary_tree_level_order(
                    BinaryTreeNode *p, Visit v) {
                LinkedQueue *buffer = create_linked_queue();
                if (!buffer) { return; }
                push_linked_queue(buffer, p);
                while (!empty_linked_queue(buffer)) {
                    pop_linked_queue(buffer, (DataType *)(&p));
                    v(p->data);
                    if (p->left) {
                        push_linked_queue(buffer, p->left);
                    }
                    if (p->right) {
                        push_linked_queue(buffer, p->right);
                    }
                }
                destroy_linked_queue(buffer);
            }
        \end{minted}
    }
\end{fragile}

\begin{frame}
    \frametitle{\insertsubsectionhead}
    \begin{exampleblock}{应用:表达式树}
        \begin{itemize}
            \item 考虑仅包括二元运算的表达式
            \item 可将运算符作为根结点,左右操作数分别为左右子结点
            \item 操作数为叶结点,运算符为非叶结点
            \item 如此可将表达式转换为二叉树
            \item 前/中/后缀表达式分别对应二叉树的先/中/后序遍历
        \end{itemize}
    \end{exampleblock}
\end{frame}

\begin{fragile}
    \frametitle{\insertsubsectionhead}
    \bicolumns[0.5]{
        \begin{exampleblock}{例:表达式树}
            \begin{itemize}
                \item 前缀表达:\mintinline[fontsize=\smaller]{console}{+ - A / B C * D E}
                \item 中缀表达:\mintinline[fontsize=\smaller]{console}{A - B / C + D * E}
                \item 后缀表达:\mintinline[fontsize=\smaller]{console}{A B C / - D E * +}
            \end{itemize}
        \end{exampleblock}
    }{
        \begin{figure}
            \centering
            \begin{tikzpicture}[ >=stealth, thick, black!50, %
                    list item/.style={draw=gray, circle, thick}, %
                    level distance=4ex, %
                    level/.style={sibling distance=16ex/#1}
                ]
                \node [terminal] {$+$}
                    child {node [terminal] {$-$}
                        child {node [terminal] {$A$}}
                        child {node [terminal] {$/$}
                            child {node [terminal] {$B$}}
                            child {node [terminal] {$C$}}
                        }
                    }
                    child {node [terminal] {$*$}
                        child {node [terminal] {$D$}}
                        child {node [terminal] {$E$}}
                    };
            \end{tikzpicture}
            \caption{表达式树示例}
            \label{fig:demo_expression_tree}
        \end{figure}
    }
\end{fragile}
\subsection{二叉树的简单应用:Huffman编码}

\begin{frame}
    \frametitle{\insertsubsectionhead}
    \begin{exampleblock}{哈夫曼(Huffman)编码}
        \begin{itemize}
            \item 需求:信息传输中的压缩编码
                \begin{itemize}
                    \item 频率\alert{高}的信息采用\alert{短}编码
                    \item 频率\alert{低}的信息采用\alert{长}编码
                \end{itemize}
            \item 目标:用尽可能少的数据表示尽可能多的信息
            \item 应用:语音、图像、视频等流媒体数据的压缩
        \end{itemize}
    \end{exampleblock}
\end{frame}

\begin{fragile}
    \frametitle{\insertsubsectionhead}
    \bicolumns[0.5]{
        \begin{table}
            \small
            \centering
            \caption{例:两种颜色二进制编码的总数据量比较}
            \label{tab:demo_huffman_lena}
            \begin{tabular}{c|rr|r}
                \toprule
                \textbf{颜色} & \textbf{定长} & \textbf{Huffman} & \textbf{出现概率} \\
                \midrule
                $A$ & \texttt{000} & \texttt{00} & $18\%$ \\
                $B$ & \texttt{001} & \texttt{11} & $32\%$ \\
                $C$ & \texttt{010} & \texttt{010} & $10\%$ \\
                $D$ & \texttt{011} & \texttt{100} & $14\%$ \\
                $E$ & \texttt{100} & \texttt{101} & $16\%$ \\
                $F$ & \texttt{101} & \texttt{0110} & $4\%$ \\
                $G$ & \texttt{110} & \texttt{0111} & $6\%$ \\
                \midrule
                \textbf{总数据量}\footnotemark & \alert{$3n$} & \alert{$2.6n$} & \\
                \bottomrule
            \end{tabular}
        \end{table}
    }{
        \begin{figure}
            \centering
            \includegraphics[width=0.8\textwidth]{images/lena.png}
            \caption{实例图片}
            \label{fig:lena}
        \end{figure}   
    }

    \footnotetext{$n$为像素点个数}
\end{fragile}

\begin{fragile}
    \frametitle{\insertsubsectionhead}
    \bicolumns[0.4]{
        \begin{block}{问题转化与建模}
            \begin{itemize}
                \item 因二进制编码,故二叉树表示
                \item 左、右子树路径分别表示0与1
                \item 编码值与叶结点一一对应
                \item 叶结点记录编码值对应权重
                \item 其他结点表示其子结点权重和
                \item 根至叶路径表示该叶对应编码
            \end{itemize}
        \end{block}
    }{
        \begin{figure}
            \centering
            \begin{tikzpicture}[ >=stealth, thick, black!50, %
                    list item/.style={draw=gray, circle, thick}, %
                    level distance=4ex, %
                    level/.style={sibling distance=28ex/#1}
                ]
                \node [nonterminal] {$100$}
                    child [dotted] {node [nonterminal,solid] {$38$}
                        child [dotted] {node [terminal,solid] {$A|18$}}
                        child [solid] {node [nonterminal] {$20$}
                            child [dotted] {node [terminal,solid] {$C|10$}}
                            child [solid] {node [nonterminal] {$10$}
                                child [dotted] {node [terminal,solid] {$F|4$}}
                                child [solid] {node [terminal] {$G|6$}}
                            }
                        }
                    }
                    child [solid] {node [nonterminal] {$62$}
                        child [dotted] {node [nonterminal,solid] {$30$}
                            child [dotted] {node [terminal,solid] {$D|14$}}
                            child [solid] {node [terminal] {$E|16$}}
                        }
                        child [solid] {node [terminal] {$B|32$}}
                    };
            \end{tikzpicture}
            \caption{Huffman编码树示例\footnotemark}
            \label{fig:demo_huffman_tree}
        \end{figure}
    }

    \footnotetext{路径中虚线表示0,实线表示1}
\end{fragile}

\begin{frame}
    \frametitle{\insertsubsectionhead}
    \begin{block}{Huffman编码的目标}
        \begin{itemize}
            \item 最小化所有叶结点深度的加权线性组合,即
                \[
                    \min_{}{\sum_{k=0}^{n-1}{\omega_{k}d_{k}}}
                \]
                其中$\omega_{k}$与$d_{k}$分别为第$k$个叶结点的\alert{权重}与
                \alert{深度}
            \item 满足上述要求的最优二叉树被称为相应信息的Huffman编码树
        \end{itemize}
    \end{block}
\end{frame}

\begin{frame}
    \frametitle{\insertsubsectionhead}
    \begin{block}{Huffman编码树的构建过程}
        \begin{enumerate}
            \item \textbf{初始化}:由给定的权重集合$\{\omega_{k}\}_{k=0}^{n-1}$
                  构建由$n$棵根树组成的\alert{森林}$\mathcal{F}$
            \item \textbf{合并}:在$\mathcal{F}$中选取根权重\alert{最小}的两棵二
                  叉树分别作为左右子树构建新二叉树
            \item \textbf{替换}:以新二叉树替换两棵旧二叉树
            \item \textbf{重复}:重复步骤2、3直至$\mathcal{F}$中只有一棵二叉树即
                  为Huffman编码树
        \end{enumerate}
    \end{block}
\end{frame}

\begin{fragile}
    \frametitle{\insertsubsectionhead}
    \begin{figure}
        \centering
        \begin{subfigure}[T][0.45\textheight][b]{0.1\textwidth}
            \centering
            \begin{tikzpicture}[ >=stealth, thick, black!50, %
                    list item/.style={draw=gray, circle, thick}, %
                    level distance=4ex, %
                    level/.style={sibling distance=28ex/#1}
                ]
                \node [terminal] {$A|18$};
            \end{tikzpicture}
            \caption{$A$}
            \label{subfig:demo_huffman_1a}
        \end{subfigure}
        ~
        \begin{subfigure}[T][0.45\textheight][b]{0.1\textwidth}
            \centering
            \begin{tikzpicture}[ >=stealth, thick, black!50, %
                    list item/.style={draw=gray, circle, thick}, %
                    level distance=4ex, %
                    level/.style={sibling distance=28ex/#1}
                ]
                \node [terminal] {$B|32$};
            \end{tikzpicture}
            \caption{$B$}
            \label{subfig:demo_huffman_1b}
        \end{subfigure}
        ~
        \begin{subfigure}[T][0.45\textheight][b]{0.1\textwidth}
            \centering
            \begin{tikzpicture}[ >=stealth, thick, black!50, %
                    list item/.style={draw=gray, circle, thick}, %
                    level distance=4ex, %
                    level/.style={sibling distance=28ex/#1}
                ]
                \node [terminal] {$C|10$};
            \end{tikzpicture}
            \caption{$C$}
            \label{subfig:demo_huffman_1c}
        \end{subfigure}
        ~
        \begin{subfigure}[T][0.45\textheight][b]{0.1\textwidth}
            \centering
            \begin{tikzpicture}[ >=stealth, thick, black!50, %
                    list item/.style={draw=gray, circle, thick}, %
                    level distance=4ex, %
                    level/.style={sibling distance=28ex/#1}
                ]
                \node [terminal] {$D|14$};
            \end{tikzpicture}
            \caption{$D$}
            \label{subfig:demo_huffman_1d}
        \end{subfigure}
        ~
        \begin{subfigure}[T][0.45\textheight][b]{0.1\textwidth}
            \centering
            \begin{tikzpicture}[ >=stealth, thick, black!50, %
                    list item/.style={draw=gray, circle, thick}, %
                    level distance=4ex, %
                    level/.style={sibling distance=28ex/#1}
                ]
                \node [terminal] {$E|16$};
            \end{tikzpicture}
            \caption{$E$}
            \label{subfig:demo_huffman_1e}
        \end{subfigure}
        ~
        \begin{subfigure}[T][0.45\textheight][b]{0.1\textwidth}
            \centering
            \begin{tikzpicture}[ >=stealth, thick, black!50, %
                    list item/.style={draw=gray, circle, thick}, %
                    level distance=4ex, %
                    level/.style={sibling distance=28ex/#1}
                ]
                \node [terminal] {\alert{$F|4$}};
            \end{tikzpicture}
            \caption{$F$}
            \label{subfig:demo_huffman_1f}
        \end{subfigure}
        ~
        \begin{subfigure}[T][0.45\textheight][b]{0.1\textwidth}
            \centering
            \begin{tikzpicture}[ >=stealth, thick, black!50, %
                    list item/.style={draw=gray, circle, thick}, %
                    level distance=4ex, %
                    level/.style={sibling distance=28ex/#1}
                ]
                \node [terminal] {\alert{$G|6$}};
            \end{tikzpicture}
            \caption{$G$}
            \label{subfig:demo_huffman_1g}
        \end{subfigure}
        \caption{Huffman编码树构建过程:初始化}
        \label{fig:demo_huffman_construction_1}
    \end{figure}
\end{fragile}

\begin{fragile}
    \frametitle{\insertsubsectionhead}
    \begin{figure}
        \centering
        \begin{subfigure}[T][0.45\textheight][b]{0.1\textwidth}
            \centering
            \begin{tikzpicture}[ >=stealth, thick, black!50, %
                    list item/.style={draw=gray, circle, thick}, %
                    level distance=4ex, %
                    level/.style={sibling distance=28ex/#1}
                ]
                \node [terminal] {$A|18$};
            \end{tikzpicture}
            \caption{$A$}
            \label{subfig:demo_huffman_2a}
        \end{subfigure}
        ~
        \begin{subfigure}[T][0.45\textheight][b]{0.1\textwidth}
            \centering
            \begin{tikzpicture}[ >=stealth, thick, black!50, %
                    list item/.style={draw=gray, circle, thick}, %
                    level distance=4ex, %
                    level/.style={sibling distance=28ex/#1}
                ]
                \node [terminal] {$B|32$};
            \end{tikzpicture}
            \caption{$B$}
            \label{subfig:demo_huffman_2b}
        \end{subfigure}
        ~
        \begin{subfigure}[T][0.45\textheight][b]{0.1\textwidth}
            \centering
            \begin{tikzpicture}[ >=stealth, thick, black!50, %
                    list item/.style={draw=gray, circle, thick}, %
                    level distance=4ex, %
                    level/.style={sibling distance=28ex/#1}
                ]
                \node [terminal] {\alert{$C|10$}};
            \end{tikzpicture}
            \caption{$C$}
            \label{subfig:demo_huffman_2c}
        \end{subfigure}
        ~
        \begin{subfigure}[T][0.45\textheight][b]{0.1\textwidth}
            \centering
            \begin{tikzpicture}[ >=stealth, thick, black!50, %
                    list item/.style={draw=gray, circle, thick}, %
                    level distance=4ex, %
                    level/.style={sibling distance=28ex/#1}
                ]
                \node [terminal] {$D|14$};
            \end{tikzpicture}
            \caption{$D$}
            \label{subfig:demo_huffman_2d}
        \end{subfigure}
        ~
        \begin{subfigure}[T][0.45\textheight][b]{0.1\textwidth}
            \centering
            \begin{tikzpicture}[ >=stealth, thick, black!50, %
                    list item/.style={draw=gray, circle, thick}, %
                    level distance=4ex, %
                    level/.style={sibling distance=28ex/#1}
                ]
                \node [terminal] {$E|16$};
            \end{tikzpicture}
            \caption{$E$}
            \label{subfig:demo_huffman_2e}
        \end{subfigure}
        \begin{subfigure}[T][0.45\textheight][b]{0.2\textwidth}
            \centering
            \begin{tikzpicture}[ >=stealth, thick, black!50, %
                    list item/.style={draw=gray, circle, thick}, %
                    level distance=4ex, %
                    level/.style={sibling distance=6ex/#1}
                ]
                \node [nonterminal] {\alert{$10$}}
                    child [dotted] {node [terminal,solid] {$F|4$}}
                    child [solid] {node [terminal] {$G|6$}};
            \end{tikzpicture}
            \caption{$F+G$}
            \label{subfig:demo_huffman_2fg}
        \end{subfigure}
        \caption{Huffman编码树构建过程:合并与替换}
        \label{fig:demo_huffman_construction_2}
    \end{figure}
\end{fragile}

\begin{fragile}
    \frametitle{\insertsubsectionhead}
    \begin{figure}
        \centering
        \begin{subfigure}[T][0.45\textheight][b]{0.1\textwidth}
            \centering
            \begin{tikzpicture}[ >=stealth, thick, black!50, %
                    list item/.style={draw=gray, circle, thick}, %
                    level distance=4ex, %
                    level/.style={sibling distance=28ex/#1}
                ]
                \node [terminal] {$A|18$};
            \end{tikzpicture}
            \caption{$A$}
            \label{subfig:demo_huffman_3a}
        \end{subfigure}
        ~
        \begin{subfigure}[T][0.45\textheight][b]{0.1\textwidth}
            \centering
            \begin{tikzpicture}[ >=stealth, thick, black!50, %
                    list item/.style={draw=gray, circle, thick}, %
                    level distance=4ex, %
                    level/.style={sibling distance=28ex/#1}
                ]
                \node [terminal] {$B|32$};
            \end{tikzpicture}
            \caption{$B$}
            \label{subfig:demo_huffman_3b}
        \end{subfigure}
        ~
        \begin{subfigure}[T][0.45\textheight][b]{0.3\textwidth}
            \centering
            \begin{tikzpicture}[ >=stealth, thick, black!50, %
                    list item/.style={draw=gray, circle, thick}, %
                    level distance=4ex, %
                    level/.style={sibling distance=12ex/#1}
                ]
                \node [nonterminal] {$20$}
                    child [dotted] {node [terminal,solid] {$C|10$}}
                    child [solid] {node [nonterminal] {$10$}
                        child [dotted] {node [terminal,solid] {$F|4$}}
                        child [solid] {node [terminal] {$G|6$}}
                        };
            \end{tikzpicture}
            \caption{$C+F+G$}
            \label{subfig:demo_huffman_3cfg}
        \end{subfigure}
        ~
        \begin{subfigure}[T][0.45\textheight][b]{0.1\textwidth}
            \centering
            \begin{tikzpicture}[ >=stealth, thick, black!50, %
                    list item/.style={draw=gray, circle, thick}, %
                    level distance=4ex, %
                    level/.style={sibling distance=28ex/#1}
                ]
                \node [terminal] {\alert{$D|14$}};
            \end{tikzpicture}
            \caption{$D$}
            \label{subfig:demo_huffman_3d}
        \end{subfigure}
        ~
        \begin{subfigure}[T][0.45\textheight][b]{0.1\textwidth}
            \centering
            \begin{tikzpicture}[ >=stealth, thick, black!50, %
                    list item/.style={draw=gray, circle, thick}, %
                    level distance=4ex, %
                    level/.style={sibling distance=28ex/#1}
                ]
                \node [terminal] {\alert{$E|16$}};
            \end{tikzpicture}
            \caption{$E$}
            \label{subfig:demo_huffman_3e}
        \end{subfigure}
        \caption{Huffman编码树构建过程:合并与替换}
        \label{fig:demo_huffman_construction_3}
    \end{figure}
\end{fragile}

\begin{fragile}
    \frametitle{\insertsubsectionhead}
    \begin{figure}
        \centering
        \begin{subfigure}[T][0.45\textheight][b]{0.1\textwidth}
            \centering
            \begin{tikzpicture}[ >=stealth, thick, black!50, %
                    list item/.style={draw=gray, circle, thick}, %
                    level distance=4ex, %
                    level/.style={sibling distance=28ex/#1}
                ]
                \node [terminal] {\alert{$A|18$}};
            \end{tikzpicture}
            \caption{$A$}
            \label{subfig:demo_huffman_4a}
        \end{subfigure}
        ~
        \begin{subfigure}[T][0.45\textheight][b]{0.1\textwidth}
            \centering
            \begin{tikzpicture}[ >=stealth, thick, black!50, %
                    list item/.style={draw=gray, circle, thick}, %
                    level distance=4ex, %
                    level/.style={sibling distance=28ex/#1}
                ]
                \node [terminal] {$B|32$};
            \end{tikzpicture}
            \caption{$B$}
            \label{subfig:demo_huffman_4b}
        \end{subfigure}
        ~
        \begin{subfigure}[T][0.45\textheight][b]{0.3\textwidth}
            \centering
            \begin{tikzpicture}[ >=stealth, thick, black!50, %
                    list item/.style={draw=gray, circle, thick}, %
                    level distance=4ex, %
                    level/.style={sibling distance=12ex/#1}
                ]
                \node [nonterminal] {\alert{$20$}}
                    child [dotted] {node [terminal,solid] {$C|10$}}
                    child [solid] {node [nonterminal] {$10$}
                        child [dotted] {node [terminal,solid] {$F|4$}}
                        child [solid] {node [terminal] {$G|6$}}
                        };
            \end{tikzpicture}
            \caption{$C+F+G$}
            \label{subfig:demo_huffman_4cfg}
        \end{subfigure}
        ~
        \begin{subfigure}[T][0.45\textheight][b]{0.24\textwidth}
            \centering
            \begin{tikzpicture}[ >=stealth, thick, black!50, %
                    list item/.style={draw=gray, circle, thick}, %
                    level distance=4ex, %
                    level/.style={sibling distance=6ex/#1}
                ]
                \node [nonterminal] {$30$}
                    child [dotted] {node [terminal,solid] {$D|14$}}
                    child [solid] {node [terminal] {$E|16$}};
            \end{tikzpicture}
            \caption{$D+E$}
            \label{subfig:demo_huffman_4de}
        \end{subfigure}
        \caption{Huffman编码树构建过程:合并与替换}
        \label{fig:demo_huffman_construction_4}
    \end{figure}
\end{fragile}

\begin{fragile}
    \frametitle{\insertsubsectionhead}
    \begin{figure}
        \centering
        \begin{subfigure}[T][0.45\textheight][b]{0.5\textwidth}
            \centering
            \begin{tikzpicture}[ >=stealth, thick, black!50, %
                    list item/.style={draw=gray, circle, thick}, %
                    level distance=4ex, %
                    level/.style={sibling distance=18ex/#1}
                ]
                \node [nonterminal] {$38$}
                    child [dotted] {node [terminal,solid] {$A|18$}}
                    child [solid] {node [nonterminal] {$20$}
                        child [dotted] {node [terminal,solid] {$C|10$}}
                        child [solid] {node [nonterminal] {$10$}
                            child [dotted] {node [terminal,solid] {$F|4$}}
                            child [solid] {node [terminal] {$G|6$}}
                        }
                    };
            \end{tikzpicture}
            \caption{$A+C+F+G$}
            \label{subfig:demo_huffman_5acfg}
        \end{subfigure}
        ~
        \begin{subfigure}[T][0.45\textheight][b]{0.1\textwidth}
            \centering
            \begin{tikzpicture}[ >=stealth, thick, black!50, %
                    list item/.style={draw=gray, circle, thick}, %
                    level distance=4ex, %
                    level/.style={sibling distance=28ex/#1}
                ]
                \node [terminal] {\alert{$B|32$}};
            \end{tikzpicture}
            \caption{$B$}
            \label{subfig:demo_huffman_5b}
        \end{subfigure}
        ~
        \begin{subfigure}[T][0.45\textheight][b]{0.24\textwidth}
            \centering
            \begin{tikzpicture}[ >=stealth, thick, black!50, %
                    list item/.style={draw=gray, circle, thick}, %
                    level distance=4ex, %
                    level/.style={sibling distance=6ex/#1}
                ]
                \node [nonterminal] {\alert{$30$}}
                    child [dotted] {node [terminal,solid] {$D|14$}}
                    child [solid] {node [terminal] {$E|16$}};
            \end{tikzpicture}
            \caption{$D+E$}
            \label{subfig:demo_huffman_5de}
        \end{subfigure}
        \caption{Huffman编码树构建过程:合并与替换}
        \label{fig:demo_huffman_construction_5}
    \end{figure}
\end{fragile}

\begin{fragile}
    \frametitle{\insertsubsectionhead}
    \begin{figure}
        \centering
        \begin{subfigure}[T][0.45\textheight][b]{0.5\textwidth}
            \centering
            \begin{tikzpicture}[ >=stealth, thick, black!50, %
                    list item/.style={draw=gray, circle, thick}, %
                    level distance=4ex, %
                    level/.style={sibling distance=18ex/#1}
                ]
                \node [nonterminal] {\alert{$38$}}
                    child [dotted] {node [terminal,solid] {$A|18$}}
                    child [solid] {node [nonterminal] {$20$}
                        child [dotted] {node [terminal,solid] {$C|10$}}
                        child [solid] {node [nonterminal] {$10$}
                            child [dotted] {node [terminal,solid] {$F|4$}}
                            child [solid] {node [terminal] {$G|6$}}
                        }
                    };
            \end{tikzpicture}
            \caption{$A+C+F+G$}
            \label{subfig:demo_huffman_6acfg}
        \end{subfigure}
        ~
        \begin{subfigure}[T][0.45\textheight][b]{0.4\textwidth}
            \centering
            \begin{tikzpicture}[ >=stealth, thick, black!50, %
                    list item/.style={draw=gray, circle, thick}, %
                    level distance=4ex, %
                    level/.style={sibling distance=12ex/#1}
                ]
                \node [nonterminal] {\alert{$62$}}
                    child [dotted] {node [nonterminal,solid] {$30$}
                        child [dotted] {node [terminal,solid] {$D|14$}}
                        child [solid] {node [terminal] {$E|16$}}
                    }
                    child [solid] {node [terminal] {$B|32$}};
            \end{tikzpicture}
            \caption{$B+D+E$}
            \label{subfig:demo_huffman_6deb}
        \end{subfigure}
        \caption{Huffman编码树构建过程:合并与替换}
        \label{fig:demo_huffman_construction_6}
    \end{figure}
\end{fragile}

\begin{fragile}
    \frametitle{\insertsubsectionhead}
    \begin{figure}
        \centering
        \begin{subfigure}[T][0.45\textheight][b]{0.9\textwidth}
            \centering
            \begin{tikzpicture}[ >=stealth, thick, black!50, %
                    list item/.style={draw=gray, circle, thick}, %
                    level distance=4ex, %
                    level/.style={sibling distance=36ex/#1}
                ]
                \node [nonterminal] {$100$}
                    child [dotted] {node [nonterminal,solid] {$38$}
                        child [dotted] {node [terminal,solid] {$A|18$}}
                        child [solid] {node [nonterminal] {$20$}
                            child [dotted] {node [terminal,solid] {$C|10$}}
                            child [solid] {node [nonterminal] {$10$}
                                child [dotted] {node [terminal,solid] {$F|4$}}
                                child [solid] {node [terminal] {$G|6$}}
                            }
                        }
                    }
                    child [solid] {node [nonterminal] {$62$}
                        child [dotted] {node [nonterminal,solid] {$30$}
                            child [dotted] {node [terminal,solid] {$D|14$}}
                            child [solid] {node [terminal] {$E|16$}}
                        }
                        child [solid] {node [terminal] {$B|32$}}
                    };
            \end{tikzpicture}
            \caption{$A+B+C+D+E+F+G$}
            \label{subfig:demo_huffman_7abcdefg}
        \end{subfigure}
        \caption{Huffman编码树构建过程:合并与替换}
        \label{fig:demo_huffman_construction_7}
    \end{figure}
\end{fragile}

\begin{frame}
    \frametitle{\insertsubsectionhead}
    \begin{block}{Huffman编码树的特点}
        \begin{itemize}
            \item 无度为$1$的结点
                \begin{itemize}
                    \item 所有非叶结点均为两个子结点合并而成,故度为$2$
                \end{itemize}
            \item 结果不唯一
                \begin{itemize}
                    \item 可额外规定合并时左右子结点权重顺序,如左小于右
                \end{itemize}
        \end{itemize}
    \end{block}
\end{frame}

\begin{fragile}
    \frametitle{\insertsubsectionhead}
    \bicolumns[0.5]{
        \begin{block}{Huffman编码树的顺序实现}
            \begin{itemize}
                \item 结点包括权重、左右子结点与父结点信息
                \item 可采用顺序表或数组存储结点序列
            \end{itemize}
        \end{block}
        \begin{figure}
            \centering
            \begin{bytefield}[boxformatting=\baselinealign]{2}
                \bitbox{4}{\mintinline[fontsize=\smaller]{c}{weight}}
                \bitbox{4}{\mintinline[fontsize=\smaller]{c}{left}}
                \bitbox{4}{\mintinline[fontsize=\smaller]{c}{right}}
                \bitbox{4}{\mintinline[fontsize=\smaller]{c}{parent}}
            \end{bytefield}
            \caption{Huffman编码树的结点结构}
            \label{fig:huffman_tree_node_struct}
        \end{figure}
        \begin{minted}{c}
            typedef struct {
                int weight; // 结点权重
                int left; // 左子结点序号
                int right; // 右子结点序号
                int parent; // 父结点序号
            } HuffmanNode;
        \end{minted}
    }{
        \begin{figure}
            \centering
            \begin{bytefield}{2}
                \bitbox[]{2}{\alert<5->{\tiny{$0$}}}\bitbox[]{3}{\only<4>{$i_{1}$}}\bitboxes{2}{{$18$}{$-1$}{$-1$}{\only<-4>{$-1$}\only<5->{$10$}}}\bitbox[]{2}{$A$} \\
                \bitbox[]{2}{\alert<6->{\tiny{$1$}}}\bitbox[]{3}{\only<5>{$i_{2}$}}\bitboxes{2}{{$32$}{$-1$}{$-1$}{\only<-5>{$-1$}\only<6->{$11$}}}\bitbox[]{2}{$B$} \\
                \bitbox[]{2}{\alert<3->{\tiny{$2$}}}\bitbox[]{3}{\only<2>{$i_{1}$}}\bitboxes{2}{{$10$}{$-1$}{$-1$}{\only<-2>{$-1$}\only<3->{$8$}}}\bitbox[]{2}{$C$} \\
                \bitbox[]{2}{\alert<4->{\tiny{$3$}}}\bitbox[]{3}{\only<3>{$i_{1}$}}\bitboxes{2}{{$14$}{$-1$}{$-1$}{\only<-3>{$-1$}\only<4->{$9$}}}\bitbox[]{2}{$D$} \\
                \bitbox[]{2}{\alert<4->{\tiny{$4$}}}\bitbox[]{3}{\only<3>{$i_{2}$}}\bitboxes{2}{{$16$}{$-1$}{$-1$}{\only<-3>{$-1$}\only<4->{$9$}}}\bitbox[]{2}{$E$} \\
                \bitbox[]{2}{\alert<2->{\tiny{$5$}}}\bitbox[]{3}{\only<1>{$i_{1}$}}\bitboxes{2}{{$4$}{$-1$}{$-1$}{\only<1>{$-1$}\only<2->{$7$}}}\bitbox[]{2}{$F$} \\
                \bitbox[]{2}{\alert<2->{\tiny{$6$}}}\bitbox[]{3}{\only<1>{$i_{2}$}}\bitboxes{2}{{$6$}{$-1$}{$-1$}{\only<1>{$-1$}\only<2->{$7$}}}\bitbox[]{2}{$G$} \\
                \bitbox[]{2}{\alert<3->{\tiny{$7$}}}\bitbox[]{3}{\only<2>{$i_{2}$}}\bitboxes{2}{{\only<1>{$100$}\only<2->{$10$}}{\only<1>{$-1$}\only<2->{$5$}}{\only<1>{$-1$}\only<2->{$6$}}{\only<-2>{$-1$}\only<3->{$8$}}} \\
                \bitbox[]{2}{\alert<5->{\tiny{$8$}}}\bitbox[]{3}{\only<4>{$i_{2}$}}\bitboxes{2}{{\only<-2>{$100$}\only<3->{$20$}}{\only<-2>{$-1$}\only<3->{$2$}}{\only<-2>{$-1$}\only<3->{$7$}}{\only<-4>{$-1$}\only<5->{$10$}}} \\
                \bitbox[]{2}{\alert<6->{\tiny{$9$}}}\bitbox[]{3}{\only<5>{$i_{1}$}}\bitboxes{2}{{\only<-3>{$100$}\only<4->{$30$}}{\only<-3>{$-1$}\only<4->{$3$}}{\only<-3>{$-1$}\only<4->{$4$}}{\only<-5>{$-1$}\only<6->{$11$}}} \\
                \bitbox[]{2}{\alert<7->{\tiny{$10$}}}\bitbox[]{3}{\only<6>{$i_{1}$}}\bitboxes{2}{{\only<-4>{$100$}\only<5->{$38$}}{\only<-4>{$-1$}\only<5->{$0$}}{\only<-4>{$-1$}\only<5->{$8$}}{\only<-6>{$-1$}\only<7->{$12$}}} \\
                \bitbox[]{2}{\alert<7->{\tiny{$11$}}}\bitbox[]{3}{\only<6>{$i_{2}$}}\bitboxes{2}{{\only<-5>{$100$}\only<6->{$62$}}{\only<-5>{$-1$}\only<6->{$9$}}{\only<-5>{$-1$}\only<6->{$1$}}{\only<-6>{$-1$}\only<7->{$12$}}} \\
                \bitbox[]{2}{\tiny{$12$}}\bitbox[]{3}{}\bitboxes{2}{{$100$}{\only<-6>{$-1$}\only<7->{$10$}}{\only<-6>{$-1$}\only<7->{$11$}}{$-1$}}
            \end{bytefield}
            \caption{Huffman编码树的构建过程}
            \label{fig:demo_huffman_tree_construction_array}
        \end{figure}
    }
\end{fragile}