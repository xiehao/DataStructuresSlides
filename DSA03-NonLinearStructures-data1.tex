\section{基本术语}

\begin{fragile}
    \frametitle{\insertsectionhead}
    \bicolumns[0.5]{
        \begin{block}{图(Graph)}
            \begin{itemize}
                \item 可被定义为$G=(V,E)$,其中\footnotemark
                      \begin{itemize}
                          \item 集合$V$中元素$v$为\textbf{顶点(vertex)}
                          \item 集合$E$中元素$e\in{V}\times{V}$为\textbf{边
                                    (edge)}
                      \end{itemize}
                \item 只定义\alert{拓扑}关系,与几何位置无关
            \end{itemize}
        \end{block}
    }{
        \begin{figure}
            \centering
            \begin{tikzpicture}[ >=stealth, thick, black!50, %
                    list item/.style={draw=gray, circle, thick}, %
                ]
                \node (a) [terminal] at (-2,0) {$v_{0}$};
                \node (b) [terminal] at (0,-1.5) {$v_{1}$};
                \node (c) [terminal] at (2,0) {$v_{2}$};
                \node (d) [terminal] at (0,1.5) {$v_{3}$};
                \draw[->] (a) to [bend left] node [pos=0.7,above,sloped] {$e_{0}$} (b);
                \draw[<-] (a) to [bend right] node [pos=0.5,below,sloped] {$e_{1}$} (b);
                \draw[->] (a) to [bend left] node [pos=0.5,above,sloped] {$e_{2}$} (d);
                \draw[<-] (a) to [bend right] node [pos=0.7,below,sloped] {$e_{3}$} (d);
                \draw (b) to [bend right] node [pos=0.5,below,sloped] {$e_{4}$} (c);
                \draw (d) to [bend left] node [pos=0.5,above,sloped] {$e_{5}$} (c);
                \draw (a) to node [pos=0.7,above,sloped] {$e_{6}$} (c);
            \end{tikzpicture}
            \caption{图}
            \label{fig:demo-graph}
        \end{figure}
    }

    \footnotetext{顶点又名\textbf{结点(node)},边又名\textbf{弧(arc)},且$V$与
        $E$均为有限集}
\end{fragile}

\begin{fragile}
    \frametitle{\insertsectionhead}
    \bicolumns[0.5]{
        \begin{block}{有向图(Undigraph)与无向图(Digraph)}
            \begin{itemize}
                \item 按顶点是否有顺序可将边$e$分为
                      \begin{itemize}
                          \item 无顺序的\textbf{无向边},记作$(u,v)$
                          \item 有顺序的\textbf{有向边},记作
                                $\langle{u,v}\rangle$
                      \end{itemize}
                \item 只有无向边的图为\textbf{无向图}
                \item 只有有向边的图为\textbf{有向图}
                \item 二者均有的图为\textbf{混合图(mixed graph)}
                \item 无向图与混合图均可转化为有向图
                      \begin{itemize}
                          \item 将一条无向边拆成两条相反的有向边
                      \end{itemize}
            \end{itemize}
        \end{block}
    }{
        \begin{figure}
            \centering
            \begin{subfigure}[T]{0.9\textwidth}
                \centering
                \begin{tikzpicture}[ >=stealth, thick, black!50, %
                        list item/.style={draw=gray, circle, thick}, %
                    ]
                    \node (a) [terminal] at (-2,0) {$v_{0}$};
                    \node (b) [terminal] at (0,-.5) {$v_{1}$};
                    \node (c) [terminal] at (2,0) {$v_{2}$};
                    \node (d) [terminal] at (0,.5) {$v_{3}$};
                    \draw[->] (a) to (b);
                    \draw[<-] (a) to (d);
                    \draw[->] (b) to (c);
                    \draw[->] (d) to (c);
                    \draw[->] (a) to (c);
                \end{tikzpicture}
                \caption{有向图}
                \label{subfig:digraph}
            \end{subfigure}
            ~
            \begin{subfigure}[T]{0.9\textwidth}
                \centering
                \begin{tikzpicture}[ >=stealth, thick, black!50, %
                        list item/.style={draw=gray, circle, thick}, %
                    ]
                    \node (a) [terminal] at (-2,0) {$v_{0}$};
                    \node (b) [terminal] at (0,-.5) {$v_{1}$};
                    \node (c) [terminal] at (2,0) {$v_{2}$};
                    \node (d) [terminal] at (0,.5) {$v_{3}$};
                    \draw (a) to (b);
                    \draw (a) to (d);
                    \draw (b) to (c);
                    \draw (d) to (c);
                    \draw (a) to (c);
                \end{tikzpicture}
                \caption{无向图}
                \label{subfig:undigraph}
            \end{subfigure}
            \caption{有向图与无向图}
            \label{fig:digraph_undigraph}
        \end{figure}
    }
\end{fragile}

\begin{fragile}
    \frametitle{\insertsectionhead}
    \bicolumns[0.55]{
        \begin{block}{顶点的度(Degree)}
            \begin{itemize}
                \item 若边$e=\langle{}v_{a},v_{b}\rangle$,则
                      \begin{itemize}
                          \item 称$v_{a}$与$v_{b}$\textbf{邻接(adjacent)}
                          \item 称二者均与$e$彼此\textbf{关联(incident)}
                          \item 称$e$为$v_{a}$的\textbf{出边(outgoing edge)}
                          \item 称$e$为$v_{b}$的\textbf{入边(incoming edge)}
                      \end{itemize}
                \item 在无向图中称与顶点$v$关联的边数为$v$的\textbf{度}
                \item 在有向图中称与顶点$v$关联的出入边数分别为$v$的\textbf{出
                          度(out-degree)}与\textbf{入度(in-degree)}
            \end{itemize}
        \end{block}
    }{
        \begin{figure}
            \centering
            \begin{subfigure}[T]{0.9\textwidth}
                \centering
                \begin{tikzpicture}[ >=stealth, thick, black!50, %
                        list item/.style={draw=gray, circle, thick}, %
                        every node/.style={terminal, rectangle split, rectangle split horizontal, rectangle split parts=2}, %
                    ]
                    \node (a) at (-1.5,0) {$2$\nodepart{two}$1$};
                    \node (b) at (0,-.5) {$1$\nodepart{two}$1$};
                    \node (c) at (1.5,0) {$0$\nodepart{two}$3$};
                    \node (d) at (0,.5) {$1$\nodepart{two}$1$};
                    \draw[->] (a) to (b);
                    \draw[<-] (a) to (d);
                    \draw[->] (b) to (c);
                    \draw[->] (d) to (c);
                    \draw[->] (a) to (c);
                \end{tikzpicture}
                \caption{有向图的出度(左)与入度(右)}
                \label{subfig:degree_digraph}
            \end{subfigure}
            ~
            \begin{subfigure}[T]{0.9\textwidth}
                \centering
                \begin{tikzpicture}[ >=stealth, thick, black!50, %
                        list item/.style={draw=gray, circle, thick}, %
                    ]
                    \node (a) [terminal] at (-1.5,0) {$3$};
                    \node (b) [terminal] at (0,-.5) {$2$};
                    \node (c) [terminal] at (1.5,0) {$3$};
                    \node (d) [terminal] at (0,.5) {$2$};
                    \draw (a) to (b);
                    \draw (a) to (d);
                    \draw (b) to (c);
                    \draw (d) to (c);
                    \draw (a) to (c);
                \end{tikzpicture}
                \caption{无向图的度}
                \label{subfig:degree_undigraph}
            \end{subfigure}
            \caption{顶点的度}
            \label{fig:degree_digraph_undigraph}
        \end{figure}
    }
\end{fragile}

\begin{frame}
    \frametitle{\insertsectionhead}
    \begin{block}{简单图(Simple Graph)}
        \begin{itemize}
            \item 称起点与终点相同的边为\textbf{自环(self-loop)}
            \item 称\alert{不含}自环且所有边均\alert{唯一}的图为\textbf{简单
                      图}\footnote{本课程只讨论简单图}
        \end{itemize}
    \end{block}
    \begin{figure}
        \centering
        \begin{subfigure}[T]{0.3\textwidth}
            \centering
            \begin{tikzpicture}[ >=stealth, thick, black!50, %
                    list item/.style={draw=gray, circle, thick}, %
                ]
                \node (a) [terminal] at (-1.5,0) {$v_{0}$};
                \node (b) [terminal] at (0,-.5) {$v_{1}$};
                \node (c) [terminal] at (1.5,0) {$v_{2}$};
                \node (d) [terminal] at (0,.5) {$v_{3}$};
                \draw (a) to (b);
                \draw (a) to (d);
                \draw (b) to (c);
                \draw (d) to (c);
                \draw (a) to (c);
            \end{tikzpicture}
            \caption{简单图}
            \label{subfig:simple_graph}
        \end{subfigure}
        ~
        \begin{subfigure}[T]{0.3\textwidth}
            \centering
            \begin{tikzpicture}[ >=stealth, thick, black!50, %
                    list item/.style={draw=gray, circle, thick}, %
                ]
                \node (a) [terminal] at (-1.5,0) {$v_{0}$};
                \node (b) [terminal] at (0,-.5) {$v_{1}$};
                \node (c) [terminal] at (1.5,0) {$v_{2}$};
                \node (d) [terminal] at (0,.5) {$v_{3}$};
                \draw (a) to (b);
                \draw[red] (a) to [loop above] ();
                \draw (a) to (d);
                \draw (b) to (c);
                \draw (d) to (c);
                \draw (a) to (c);
            \end{tikzpicture}
            \caption{带自环的非简单图}
            \label{subfig:non_simple_graph_a}
        \end{subfigure}
        ~
        \begin{subfigure}[T]{0.3\textwidth}
            \centering
            \begin{tikzpicture}[ >=stealth, thick, black!50, %
                    list item/.style={draw=gray, circle, thick}, %
                ]
                \node (a) [terminal] at (-1.5,0) {$v_{0}$};
                \node (b) [terminal] at (0,-.5) {$v_{1}$};
                \node (c) [terminal] at (1.5,0) {$v_{2}$};
                \node (d) [terminal] at (0,.5) {$v_{3}$};
                \draw (a) to (b);
                \draw[red] (a) to [bend right] (b);
                \draw (a) to (d);
                \draw (b) to (c);
                \draw (d) to (c);
                \draw (a) to (c);
            \end{tikzpicture}
            \caption{有重复边的非简单图}
            \label{subfig:non_simple_graph_b}
        \end{subfigure}
        \caption{简单图与非简单图}
        \label{fig:simple_graph_non_simple_graph}
    \end{figure}
\end{frame}

\begin{fragile}
    \frametitle{\insertsectionhead}
    \begin{block}{路径(Path)}
        \begin{itemize}
            \item 若序列$\pi=\{v_{k}\}_{k=0}^{n}$满足$v_{k}$与$v_{k+1}$\alert{邻
                      接}\footnote{指存在边$e_{k}$满足$e_{k}=(v_{k},v_{k+1})$或
                  $e_{k}=\langle{}v_{k},v_{k+1}\rangle$,$k\in\mathbb{Z}\cap[0,n)$},
                  则称之为自$v_{0}$至$v_{n}$的一条\textbf{路径}
                  \begin{itemize}
                      \item 称经过的总边数$|\pi|$为路径\textbf{长度(length)}
                      \item 称无重复顶点的路径为\textbf{简单(simple)路径}
                  \end{itemize}
            \item 两顶点间的路径一般\alert{不唯一}
        \end{itemize}
    \end{block}
    \begin{figure}
        \centering
        \begin{subfigure}[T]{0.3\textwidth}
            \centering
            \begin{tikzpicture}[ >=stealth, thick, black!50, %
                    list item/.style={draw=gray, circle, thick}, %
                ]
                \node (a) [terminal] at (-1.5,0) {$v_{0}$};
                \node (b) [terminal] at (0,-.5) {$v_{1}$};
                \node (c) [terminal] at (1.5,0) {$v_{2}$};
                \node (d) [terminal] at (0,.5) {$v_{3}$};
                \draw[<-,red] (a) to (b);
                \draw (a) to (d);
                \draw (b) to (c);
                \draw[<-,red] (d) to (c);
                \draw[->,red] (a) to (c);
            \end{tikzpicture}
            \caption{简单路径:$(v_{1},v_{0},v_{2},v_{3})$}
            \label{subfig:simple_path}
        \end{subfigure}
        ~
        \begin{subfigure}[T]{0.3\textwidth}
            \centering
            \begin{tikzpicture}[ >=stealth, thick, black!50, %
                    list item/.style={draw=gray, circle, thick}, %
                ]
                \node (a) [terminal] at (-1.5,0) {$v_{0}$};
                \node (b) [terminal] at (0,-.5) {$v_{1}$};
                \node (c) [terminal] at (1.5,0) {$v_{2}$};
                \node (d) [terminal] at (0,.5) {$v_{3}$};
                \draw[<-,red] (a) to (b);
                \draw[<-,red] (a) to (d);
                \draw (b) to (c);
                \draw[<-,red] (d) to (c);
                \draw[->,red] (a) to (c);
            \end{tikzpicture}
            \caption{非简单路径:$(v_{1},v_{0},v_{2},v_{3},v_{0})$}
            \label{subfig:non_simple_path}
        \end{subfigure}
        \caption{图的路径}
        \label{fig:graph_paths}
    \end{figure}
\end{fragile}

\begin{fragile}
    \frametitle{\insertsectionhead}
    \begin{block}{环路(Cycle)}
        \begin{itemize}
            \item 若路径$\pi=\{v_{k}\}_{k=0}^{n}$中起止顶点相同,即
                  $v_{0}=v_{n}$,则称其为\textbf{环路}
                  \begin{itemize}
                      \item 若除起止结点相同外无任何其他结点两两相同,则称其为
                            \textbf{简单环路}
                      \item 称经过图中各\alert{边}一次且仅一次的环路为\textbf{欧
                                拉环路(Eulerian tour)}
                      \item 称经过图中各\alert{顶点}一次且仅一次的环路为
                            \textbf{哈米尔顿环路(Hamiltonian tour)}
                  \end{itemize}
        \end{itemize}
    \end{block}
    \begin{figure}
        \centering
        \begin{subfigure}[T]{0.3\textwidth}
            \centering
            \begin{tikzpicture}[ >=stealth, thick, black!50, %
                    list item/.style={draw=gray, circle, thick}, %
                ]
                \node (a) [terminal] at (-1.5,0) {$v_{0}$};
                \node (b) [terminal] at (0,-.5) {$v_{1}$};
                \node (c) [terminal] at (1.5,0) {$v_{2}$};
                \node (d) [terminal] at (0,.5) {$v_{3}$};
                \draw[->] (a) to (b);
                \draw[<-,red] (a) to (d);
                \draw[->] (b) to (c);
                \draw[<-,red] (d) to (c);
                \draw[->,red] (a) to (c);
                \draw[<-] (a) to [bend left=60] (c);
            \end{tikzpicture}
            \caption{简单环路:$(v_{0},v_{2},v_{3})$}
            \label{subfig:simple_cycle}
        \end{subfigure}
        \begin{subfigure}[T]{0.35\textwidth}
            \centering
            \begin{tikzpicture}[ >=stealth, thick, black!50, %
                    list item/.style={draw=gray, circle, thick}, %
                ]
                \node (a) [terminal] at (-1.5,0) {$v_{0}$};
                \node (b) [terminal] at (0,-.5) {$v_{1}$};
                \node (c) [terminal] at (1.5,0) {$v_{2}$};
                \node (d) [terminal] at (0,.5) {$v_{3}$};
                \draw[->,red] (a) to (b);
                \draw[<-,red] (a) to (d);
                \draw[->,red] (b) to (c);
                \draw[<-,red] (d) to (c);
                \draw[->,red] (a) to (c);
                \draw[<-,red] (a) to [bend left=60] (c);
            \end{tikzpicture}
            \caption{欧拉环路:$(v_{0},v_{1},v_{2},v_{0},v_{2},v_{3},v_{0})$}
            \label{subfig:demo_euler_tour}
        \end{subfigure}
        \begin{subfigure}[T]{0.32\textwidth}
            \centering
            \begin{tikzpicture}[ >=stealth, thick, black!50, %
                    list item/.style={draw=gray, circle, thick}, %
                ]
                \node (a) [terminal] at (-1.5,0) {$v_{0}$};
                \node (b) [terminal] at (0,-.5) {$v_{1}$};
                \node (c) [terminal] at (1.5,0) {$v_{2}$};
                \node (d) [terminal] at (0,.5) {$v_{3}$};
                \draw[->,red] (a) to (b);
                \draw[<-,red] (a) to (d);
                \draw[->,red] (b) to (c);
                \draw[<-,red] (d) to (c);
                \draw[->] (a) to (c);
                \draw[<-] (a) to [bend left=60] (c);
            \end{tikzpicture}
            \caption{哈米尔顿环路:$(v_{0},v_{1},v_{2},v_{3},v_{0})$}
            \label{subfig:demo_hamiltonian_tour}
        \end{subfigure}
        \caption{图的环路}
        \label{fig:graph_cycles}
    \end{figure}
\end{fragile}

\begin{fragile}
    \frametitle{\insertsectionhead}
    \begin{block}{完全图(Complete Graph)}
        \begin{itemize}
            \item 图中任意两顶点均邻接
            \item 若顶点数为$n$则无向与有向边数分别为$\frac{n(n-1)}{2}$与
            $n(n-1)$
        \end{itemize}
    \end{block}
    \begin{figure}
        \centering
        \begin{subfigure}[b]{0.3\textwidth}
            \centering
            \begin{tikzpicture}[ >=stealth, thick, black!50, %
                    list item/.style={draw=gray, circle, thick}, %
                ]
                \node (a) [terminal] at (-1.5,0) {$v_{0}$};
                \node (b) [terminal] at (0,-0.75) {$v_{1}$};
                \node (c) [terminal] at (1.5,0) {$v_{2}$};
                \node (d) [terminal] at (0,0.75) {$v_{3}$};
                \draw (a) to (b);
                \draw (a) to [bend right=90] (c);
                \draw (a) to (d);
                \draw (b) to (c);
                \draw (b) to (d);
                \draw (c) to (d);
            \end{tikzpicture}
            \caption{完全无向图}
            \label{subfig:complete_undigraph}
        \end{subfigure}
        ~
        \begin{subfigure}[b]{0.3\textwidth}
            \centering
            \begin{tikzpicture}[ >=stealth, thick, black!50, %
                    list item/.style={draw=gray, circle, thick}, %
                ]
                \node (a) [terminal] at (-1.5,0) {$v_{0}$};
                \node (b) [terminal] at (0,-0.75) {$v_{1}$};
                \node (c) [terminal] at (1.5,0) {$v_{2}$};
                \node (d) [terminal] at (0,0.75) {$v_{3}$};
                \draw[->] (a) to [bend right=10] (b);
                \draw[->] (a) to [bend right=90] (c);
                \draw[->] (a) to [bend right=10] (d);
                \draw[->] (b) to [bend right=10] (c);
                \draw[->] (b) to [bend right=10] (d);
                \draw[->] (c) to [bend right=10] (d);
                \draw[<-] (a) to [bend left=10] (b);
                \draw[<-] (a) to [bend left=90] (c);
                \draw[<-] (a) to [bend left=10] (d);
                \draw[<-] (b) to [bend left=10] (c);
                \draw[<-] (b) to [bend left=10] (d);
                \draw[<-] (c) to [bend left=10] (d);
            \end{tikzpicture}
            \caption{完全有向图}
            \label{subfig:demo_complete_digraph}
        \end{subfigure}
        \caption{完全图}
        \label{fig:complete_graphs}
    \end{figure}
\end{fragile}

\begin{fragile}
    \frametitle{\insertsectionhead}
    \begin{block}{带权图(Weighted Graph)}
        \begin{itemize}
            \item 为每条边指定权重,又名\textbf{带权网络(network)}
            \item 用于表示顶点关系的细节,如长度、流量、成本等
            \item 普通图可看作所有边权重均为$1$的带权图
        \end{itemize}
    \end{block}
    \begin{figure}
        \centering
        \begin{subfigure}[T]{0.3\textwidth}
            \centering
            \begin{tikzpicture}[ >=stealth, thick, black!50, %
                    list item/.style={draw=gray, circle, thick}, %
                ]
                \node (a) [terminal] at (-1.5,0) {$v_{0}$};
                \node (b) [terminal] at (0,-.5) {$v_{1}$};
                \node (c) [terminal] at (1.5,0) {$v_{2}$};
                \node (d) [terminal] at (0,.5) {$v_{3}$};
                \draw[->] (a) to (b);
                \draw[<-] (a) to (d);
                \draw[->] (b) to (c);
                \draw[<-] (d) to (c);
                \draw[->] (a) to (c);
            \end{tikzpicture}
            \caption{普通图}
            \label{subfig:non_weighted_graph}
        \end{subfigure}
        ~
        \begin{subfigure}[T]{0.3\textwidth}
            \centering
            \begin{tikzpicture}[ >=stealth, thick, black!50, %
                    list item/.style={draw=gray, circle, thick}, %
                ]
                \node (a) [terminal] at (-1.5,0) {$v_{0}$};
                \node (b) [terminal] at (0,-.5) {$v_{1}$};
                \node (c) [terminal] at (1.5,0) {$v_{2}$};
                \node (d) [terminal] at (0,.5) {$v_{3}$};
                \draw[->] (a) to node [pos=0.5,below,sloped] {$w_{0}$} (b);
                \draw[<-] (a) to node [pos=0.5,above,sloped] {$w_{1}$} (d);
                \draw[->] (b) to node [pos=0.5,below,sloped] {$w_{2}$} (c);
                \draw[<-] (d) to node [pos=0.5,above,sloped] {$w_{3}$} (c);
                \draw[->] (a) to node [pos=0.5,sloped] {$w_{4}$} (c);
            \end{tikzpicture}
            \caption{带权图}
            \label{subfig:demo_weighted_graph}
        \end{subfigure}
        \caption{普通图与带权图}
        \label{fig:weighted_graphs}
    \end{figure}
\end{fragile}