\subsection{深度优先搜索}

\begin{frame}
    \frametitle{\insertsubsectionhead}
    \begin{block}{深度优先搜索(Depth-First Search, DFS)}
        \begin{itemize}
            \item 基本原则:不撞南墙不回头
            \item 类似树的先序遍历或后序遍历
            \item 结果不唯一,与遍历邻接顶点的顺序有关
        \end{itemize}
    \end{block}
    \begin{alertblock}{算法步骤}
        \begin{enumerate}
            \item 标记所有顶点为\alert{未被访问}状态
            \item 按序号依次取出\alert{未被访问}顶点并执行
                  \begin{enumerate}[a.]
                      \item 访问该顶点并将其标记为\alert{已被访问}
                      \item 对其所有\alert{未被访问}的\alert{邻接}顶点递归执行深
                            度优先搜索
                      \item 返回步骤2
                  \end{enumerate}
            \item 结束
        \end{enumerate}
    \end{alertblock}
\end{frame}

\begin{fragile}
    \frametitle{\insertsubsectionhead}
    \begin{figure}
        \centering
        \begin{tikzpicture}[ >=stealth, thick, black!50, %
                list item/.style={draw=gray, circle, thick}, %
                level distance=5ex, %
                level/.style={sibling distance=16ex/#1}
            ]
            \node [terminal] (a) {\alert<2->{$A$}}
            child {node [terminal] (b) {\alert<3->{$B$}}
                    child {node [terminal] (d) {\alert<4->{$D$}}
                            child [missing]
                            child {node [terminal] (h) {\alert<6->{$H$}}}
                        }
                    child {node [terminal] (e) {\alert<5->{$E$}}}
                }
            child {node [terminal] (c) {\alert<7->{$C$}}
                    child {node [terminal] (f) {\alert<8->{$F$}}}
                    child {node [terminal] (g) {\alert<9->{$G$}}}
                };
            \draw (d) to (e);
            \draw (f) to (g);
            \draw<3-> [red] (a) to (b);
            \draw<7-> [red] (a) to (c);
            \draw<4-> [red] (b) to (d);
            % \draw [red] (b) to (e);
            \draw<5-> [red] (d) to (e);
            \draw<8-> [red] (c) to (f);
            % \draw [red] (c) to (g);
            \draw<9-> [red] (f) to (g);
            \draw<6-> [red] (d) to (h);
        \end{tikzpicture}
        \caption{图的深度优先遍历过程:$A,B,D,E,H,C,F,G$}
        \label{fig:demo_dfs}
    \end{figure}
\end{fragile}

\begin{fragile}
    \frametitle{\insertsubsectionhead}
    \bicolumns[0.5]{
        \begin{block}{练习:从$A$开始的深度优先搜索可能结果为}
            \begin{enumerate}[I.]
                \item $A,B,E,C,D,F$
                \item $A,C,F,E,B,D$
                \item $A,E,B,C,F,D$
                \item \alert<2->{$A,E,D,F,C,B$}
            \end{enumerate}
        \end{block}
    }{
        \begin{figure}
            \centering
            \begin{tikzpicture}[ >=stealth, thick, black!50, %
                    list item/.style={draw=gray, circle, thick}, %
                    level distance=5ex, %
                    level/.style={sibling distance=8ex/#1}
                ]
                \node [terminal] (a) {$A$}
                child {node [terminal] (b) {$B$}}
                child {node [terminal] (e) {$E$}
                        child {node [terminal] (d) {$D$}}
                    }
                child {node [terminal] (c) {$C$}
                        child {node [terminal] (f) {$F$}}
                    };
                \draw (d) to (f);
            \end{tikzpicture}
            \caption{图的深度优先遍历测试}
            \label{fig:test_dfs}
        \end{figure}
    }
\end{fragile}

\begin{fragile}
    \frametitle{\insertsubsectionhead}
    \begin{block}{深度优先搜索的一种\alert{邻接矩阵}结构递归实现}
        \vspace{2ex}
        \bicolumns[0.5]{
            \begin{itemize}
                \item 多次调用以便找出可能的多个连通域
                        \begin{minted}[highlightlines={7,8}]{c}
                            void depth_first_search_matrix(
                                    AdjacencyMatrix *m, VisitType v) {
                                assert(m);
                                clear_matrix_states(m); // 清除顶点状态
                                for (int i = 0; i < m->size; ++i) {
                                    if (!vertex_at(m, i)->visited) {
                                        depth_first_search_matrix_core(
                                                m, i, v);
                                    }
                                }
                            }
                        \end{minted}
            \end{itemize}
        }{
            \begin{itemize}
                \item 对特定顶点递归执行深度优先搜索
                        \begin{minted}[highlightlines={9,10}]{c}
                            static void depth_first_search_matrix_core(
                                    AdjacencyMatrix *m, int k, VisitType v) {
                                Vertex *vertex = vertex_at(m, k);
                                v(vertex->data); // 访问顶点数据
                                vertex->visited = true; // 标记已被访问
                                for (int i = 0; i < m->size; ++i) {
                                    if (is_adjacent(m, k, i) &&
                                        !vertex_at(m, i)->visited) {
                                        depth_first_search_matrix_core(
                                                m, i, v);
                                    }
                                }
                            }
                        \end{minted}
            \end{itemize}
        }
    \end{block}
\end{fragile}

\begin{fragile}
    \frametitle{\insertsubsectionhead}
    \begin{block}{深度优先搜索的一种\alert{邻接表}结构递归实现}
        \vspace{2ex}
        \bicolumns[0.5]{
            \begin{itemize}
                \item 多次调用以便找出可能的多个连通域
                        \begin{minted}[highlightlines={11,12}]{c}
                            void depth_first_search_list(
                                    AdjacencyList *l, VisitType v) {
                                assert(l);
                                clear_list_states(l); // 清除顶点状态
                                int n_vertices = 
                                    size_of_vector(l->vertices);
                                for (int i = 0; i < n_vertices; ++i) {
                                    Vertex *x = 
                                        get_vertex_at(l->vertices, i);
                                    if (x && !x->visited) {
                                        depth_first_search_list_core(
                                                l, x, v);
                                    }
                                }
                            }
                        \end{minted}
            \end{itemize}
        }{
            \begin{itemize}
                \item 对特定顶点递归执行深度优先搜索
                        \begin{minted}[highlightlines={11,12}]{c}
                            static void depth_first_search_list_core(
                                    AdjacencyList *l, Vertex *x, VisitType v) {
                                v(x->data); // 访问顶点数据
                                x->visited = true; // 标记已被访问
                                int n_neighbors = 
                                    size_of_vector(x->neighbors);
                                for (int i = 0; i < n_neighbors; ++i) {
                                    Vertex *to = 
                                        get_edge_at(x->neighbors, i)->to;
                                    if (!to->visited) {
                                        depth_first_search_list_core(
                                                l, to, v);
                                    }
                                }
                            }
                        \end{minted}
            \end{itemize}
        }
    \end{block}
\end{fragile}