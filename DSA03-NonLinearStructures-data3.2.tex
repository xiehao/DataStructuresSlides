\subsection{广度优先搜索}

\begin{frame}
    \frametitle{\insertsubsectionhead}
    \begin{block}{广度优先搜索(Breadth-First Search, BFS)}
        \begin{itemize}
            \item 基本原则:
            \item 类似树的层次遍历
            \item 结果不唯一,与遍历邻接顶点的顺序有关
        \end{itemize}
    \end{block}
    \begin{alertblock}{思路}
        \begin{enumerate}
            \item 标记所有顶点为\alert{未被访问}状态
            \item 按序号依次取出\alert{未被访问}顶点并执行
                  \begin{enumerate}[a.]
                      \item 初始化顶点队列
                      \item 访问该顶点、将其标记为\alert{已被访问}并入队
                      \item 若队列不空,则执行
                            \begin{enumerate}[i.]
                                \item 将队首顶点出队
                                \item 对该顶点的所有未被访问邻接顶点执行:访问、
                                      标记为\alert{已被访问}并入队
                            \end{enumerate}
                      \item 销毁队列并返回步骤2
                  \end{enumerate}
            \item 结束
        \end{enumerate}
    \end{alertblock}
\end{frame}

\begin{fragile}
    \frametitle{\insertsubsectionhead}
    \begin{figure}
        \centering
        \begin{tikzpicture}[ >=stealth, thick, black!50, %
                list item/.style={draw=gray, circle, thick}, %
                level distance=5ex, %
                level/.style={sibling distance=16ex/#1}
            ]
            \node [terminal] (a) {\alert<2->{$A$}}
            child {node [terminal] (b) {\alert<3->{$B$}}
                    child {node [terminal] (d) {\alert<5->{$D$}}
                            child [missing]
                            child {node [terminal] (h) {\alert<9->{$H$}}}
                        }
                    child {node [terminal] (e) {\alert<6->{$E$}}}
                }
            child {node [terminal] (c) {\alert<4->{$C$}}
                    child {node [terminal] (f) {\alert<7->{$F$}}}
                    child {node [terminal] (g) {\alert<8->{$G$}}}
                };
            \draw (d) to (e);
            \draw (f) to (g);
            \draw<3-> [red] (a) to (b);
            \draw<4-> [red] (a) to (c);
            \draw<5-> [red] (b) to (d);
            \draw<6-> [red] (b) to (e);
            % \draw [red] (d) to (e);
            \draw<7-> [red] (c) to (f);
            \draw<8-> [red] (c) to (g);
            % \draw [red] (f) to (g);
            \draw<9-> [red] (d) to (h);
        \end{tikzpicture}
        \caption{图的一种广度优先遍历过程:$A,B,C,D,E,F,G,H$}
        \label{fig:demo_bfs}
    \end{figure}
\end{fragile}