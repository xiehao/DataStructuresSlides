\subsection{广度优先搜索}

\begin{frame}
    \frametitle{\insertsubsectionhead}
    \begin{block}{广度优先搜索(Breadth-First Search, BFS)}
        \begin{itemize}
            \item 基本原则:远交近攻
            \item 类似树的层次遍历
            \item 结果不唯一,与遍历邻接顶点的顺序有关
        \end{itemize}
    \end{block}
    \begin{alertblock}{算法步骤}
        \begin{enumerate}
            \item 标记所有顶点为\alert{未被访问}状态
            \item 按序号依次取出\alert{未被访问}顶点并执行
                  \begin{enumerate}[a.]
                      \item 初始化顶点队列
                      \item 访问该顶点、将其标记为\alert{已被访问}并入队
                      \item 若队列不空,则执行
                            \begin{enumerate}[i.]
                                \item 将队首顶点出队
                                \item 对该顶点的所有未被访问邻接顶点执行:访问、
                                      标记为\alert{已被访问}并入队
                            \end{enumerate}
                      \item 销毁队列并返回步骤2
                  \end{enumerate}
            \item 结束
        \end{enumerate}
    \end{alertblock}
\end{frame}

\begin{fragile}
    \frametitle{\insertsubsectionhead}
    \begin{figure}
        \centering
        \begin{tikzpicture}[ >=stealth, thick, black!50, %
                list item/.style={draw=gray, circle, thick}, %
                level distance=5ex, %
                level/.style={sibling distance=16ex/#1}
            ]
            \node [terminal] (a) {\alert<2->{$A$}}
            child {node [terminal] (b) {\alert<3->{$B$}}
                    child {node [terminal] (d) {\alert<5->{$D$}}
                            child [missing]
                            child {node [terminal] (h) {\alert<9->{$H$}}}
                        }
                    child {node [terminal] (e) {\alert<6->{$E$}}}
                }
            child {node [terminal] (c) {\alert<4->{$C$}}
                    child {node [terminal] (f) {\alert<7->{$F$}}}
                    child {node [terminal] (g) {\alert<8->{$G$}}}
                };
            \draw (d) to (e);
            \draw (f) to (g);
            \draw<3-> [red] (a) to (b);
            \draw<4-> [red] (a) to (c);
            \draw<5-> [red] (b) to (d);
            \draw<6-> [red] (b) to (e);
            % \draw [red] (d) to (e);
            \draw<7-> [red] (c) to (f);
            \draw<8-> [red] (c) to (g);
            % \draw [red] (f) to (g);
            \draw<9-> [red] (d) to (h);
        \end{tikzpicture}
        \caption{图的一种广度优先遍历过程:$A,B,C,D,E,F,G,H$}
        \label{fig:demo_bfs}
    \end{figure}
\end{fragile}

\begin{fragile}
    \frametitle{\insertsubsectionhead}
    \begin{block}{广度优先搜索的一种\alert{邻接矩阵}结构非递归实现}
        \begin{itemize}
            \item 多次调用以便找出可能的多个连通域
                    \begin{minted}[highlightlines={6}]{c}
                        void breadth_first_search_matrix(AdjacencyMatrix *m, VisitType v) {
                            assert(m);
                            clear_matrix_states(m);
                            for (int i = 0; i < m->size; ++i) {
                                if (!vertex_at(m, i)->visited) {
                                    breadth_first_search_matrix_core(m, i, v);
                                }
                            }
                        }
                    \end{minted}
        \end{itemize}
    \end{block}
\end{fragile}

\begin{fragile}
    \frametitle{\insertsubsectionhead}
    \begin{block}{广度优先搜索的一种\alert{邻接矩阵}结构非递归实现}
        \begin{itemize}
            \item 对特定顶点执行广度优先搜索
                    \begin{minted}[highlightlines={5,7,11}]{c}
                        static void breadth_first_search_matrix_core(AdjacencyMatrix *m, int k, VisitType v) {
                            LinkedQueue *buffer = create_linked_queue(); // 准备辅助队列
                            if (!buffer) { return; }
                            Vertex *from = vertex_at(m, k); // 取出当前顶点
                            visit_vertex_and_push_queue(v, buffer, from); // 访问之并入队
                            while (!empty_linked_queue(buffer)) { // 若队列非空,则循环执行
                                pop_linked_queue(buffer, (DataType *)&from); // 出队
                                for (int i = 0; i < m->size; ++i) { // 依次处理出队顶点的未被访问邻接顶点
                                    Vertex *to = vertex_at(m, i);
                                    if (is_adjacent(m, from->index, i) && !to->visited) {
                                        visit_vertex_and_push_queue(v, buffer, to); // 访问之并入队
                                    }
                                }
                            }
                            destroy_linked_queue(buffer); // 销毁队列
                        }
                    \end{minted}
        \end{itemize}
    \end{block}
\end{fragile}

\begin{fragile}
    \frametitle{\insertsubsectionhead}
    \begin{block}{广度优先搜索的一种\alert{邻接表}结构非递归实现}
        \begin{itemize}
            \item 多次调用以便找出可能的多个连通域
                    \begin{minted}[highlightlines={8}]{c}
                        void breadth_first_search_list(AdjacencyList *l, VisitType v) {
                            assert(l);
                            clear_list_states(l);
                            int n_vertices = size_of_vector(l->vertices);
                            for (int i = 0; i < n_vertices; ++i) {
                                Vertex *x = get_vertex_at(l->vertices, i);
                                if (!x->visited) {
                                    breadth_first_search_list_core(l, x, v);
                                }
                            }
                        }
                    \end{minted}
        \end{itemize}
    \end{block}
\end{fragile}

\begin{fragile}
    \frametitle{\insertsubsectionhead}
    \begin{block}{广度优先搜索的一种\alert{邻接表}结构非递归实现}
        \begin{itemize}
            \item 对特定顶点执行广度优先搜索
                    \begin{minted}[highlightlines={4,6,11}]{c}
                        static void breadth_first_search_list_core(AdjacencyList *l, Vertex *x, VisitType v) {
                            LinkedQueue *buffer = create_linked_queue();
                            if (!buffer) { return; }
                            visit_vertex_and_push_queue(v, buffer, x);
                            while (!empty_linked_queue(buffer)) {
                                pop_linked_queue(buffer, (DataType *)&x);
                                int n_neighbors = size_of_vector(x->neighbors);
                                for (int i = 0; i < n_neighbors; ++i) {
                                    Vertex *to = get_edge_at(x->neighbors, i)->to;
                                    if (!to->visited) {
                                        visit_vertex_and_push_queue(v, buffer, to);
                                    }
                                }
                            }
                            destroy_linked_queue(buffer);
                        }
                    \end{minted}
        \end{itemize}
    \end{block}
\end{fragile}