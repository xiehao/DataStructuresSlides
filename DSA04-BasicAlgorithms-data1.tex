\section{排序}

\begin{frame}
    \frametitle{\insertsectionhead}
    \begin{block}{基本概念}
        \begin{itemize}
            \item 已知数据元素序列$\{a_{i}\}_{i=0}^{n}$,为每个元素$a_{i}$均关联
                  一个\textbf{关键码(key)}\footnote{简称\textbf{键}}$k_{i}$
            \item 为关键码定义一种\textbf{顺序(order)},使得关键码之间可互相比较
                  大小
            \item 对下标序列$\{i\}_{i=0}^{n}$重\alert{排列}生成新序列
                  $\{p_{i}\}_{i=0}^{n}$
            \item 若满足$k_{p_{0}}\leq{}k_{p_{1}}\leq\cdots\leq{}k_{p_{n}}$,则
                  称$\{a_{p_{i}}\}_{i=0}^{n}$为\textbf{有序序列(sorted
                      sequence)}
            \item 称生成有序序列的过程为\textbf{排序(sorting)}
        \end{itemize}
    \end{block}
\end{frame}

\begin{frame}
    \frametitle{\insertsectionhead}
    \begin{exampleblock}{例}
        \begin{itemize}
            \item 已知整数序列:$21, 25, 22, 10, 25, 18$
            \item \alert{关键码}为元素值本身
            \item 约定\alert{顺序}为升序
            \item 排序结果为:$10, 18, 21, 22, 25, 25$
        \end{itemize}
    \end{exampleblock}
\end{frame}

\begin{fragile}
    \frametitle{\insertsectionhead}
    \begin{block}{排序的分类方法}
        \vspace*{2ex}
        \bicolumns[0.5]{
            \begin{itemize}
                \item 根据\alert{数据规模}与\alert{存储特点}分:
                      \begin{itemize}
                          \item \textbf{内部排序}:规模较小,内存足矣
                                \only<2>{$\checkmark$}
                          \item \textbf{外部排序}:规模庞大,须借助外存
                                \footnotemark
                      \end{itemize}
                \item 根据\alert{输入形式}分:
                      \begin{itemize}
                          \item \textbf{离线排序}:一次性给出全部数据
                                \only<2>{$\checkmark$}
                          \item \textbf{在线排序}:实时生成数据,分批给出
                      \end{itemize}
            \end{itemize}
        }{
            \begin{itemize}
                \item 根据\alert{体系结构}分:
                      \begin{itemize}
                          \item \textbf{串行排序}:按顺序依次处理
                                \only<2>{$\checkmark$}
                          \item \textbf{并行排序}:部分同时处理
                      \end{itemize}
                \item 根据\alert{算法策略}分:
                      \begin{itemize}
                          \item \textbf{确定排序}:不采用随机策略
                                \only<2>{$\checkmark$}
                          \item \textbf{随机排序}:采用随机策略
                      \end{itemize}
            \end{itemize}
        }
    \end{block}

    \footnotetext{如硬盘、分布式设备等}
\end{fragile}

\begin{frame}
    \frametitle{\insertsectionhead}
    \begin{block}{排序的种类}
        \begin{table}
            \centering
            \smaller
            \begin{tabular}{l|l}
                \toprule
                \textbf{算法}                   & \textbf{特点}              \\
                \midrule
                插入类:\alert{直接插入}              & \multirow{2}{*}{顺序比较、简单} \\
                选择类:\alert{直接选择}、\alert{冒泡}、堆 &                          \\
                \midrule
                \alert{快速}、归并                 & 分治策略                     \\
                \midrule
                希尔、桶、基数                       & 其他                       \\
                \bottomrule
            \end{tabular}
        \end{table}
    \end{block}
\end{frame}

\begin{fragile}
    \frametitle{\insertsectionhead}
    \begin{block}{插入排序:整理一手乱序扑克牌}
        \begin{itemize}
            \item 将序列$s$分为\alert{无序}与\alert{有序}两个子序列$s_{u}=s$与
                  $s_{o}=\emptyset$
            \item 反复从$s_{u}$中取出元素并插入$s_{o}$使之依然有序,直至
                  $s_{u}=\emptyset$,则$s_{o}$即为所求
        \end{itemize}
    \end{block}
    \bicolumns[0.5]{
        \begin{block}{算法描述}
            \begin{algorithmx}[alg:insertion]{插入排序}
                \Procedure{InsersionSort}{$s,f,t$}
                \State \Comment{对序列$s$中下标区间$[f,t)$内的元素进行插入排序}
                \State $i\gets f$
                \While{$i<t$}
                \State $s\gets$ InsertElement($s,f,i$)\Comment{插入元素子算法}
                \State $i\gets i+1$
                \EndWhile\label{insertionendwhile}
                \State \textbf{return} $s$\Comment{返回排序后的序列$s$}
                \EndProcedure
            \end{algorithmx}
        \end{block}
    }{
        \begin{exampleblock}{实例}
            \begin{itemize}
                \item 原序列:$()\alert{21}, 25, 22, 10, 25, 18$
                \item 第一趟:$(21), \alert{25}, 22, 10, 25, 18$
                \item 第二趟:$(21, 25), \alert{22}, 10, 25, 18$
                \item 第三趟:$(21, 22, 25), \alert{10}, 25, 18$
                \item 第四趟:$(10, 21, 22, 25), \alert{25}, 18$
                \item 第五趟:$(10, 21, 22, 25, 25), \alert{18}$
                \item 第六趟:$(10, 18, 21, 22, 25, 25)$
            \end{itemize}
        \end{exampleblock}
    }
\end{fragile}

\begin{frame}
    \frametitle{\insertsectionhead}
    \begin{block}{插入元素}
        \begin{itemize}
            \item 在\alert{有序序列}中\alert{查找}并\alert{插入}指定元素
        \end{itemize}
    \end{block}
    \begin{block}{可选思路}
        \begin{itemize}
            \item 顺序查找\footnote{称对应算法为\textbf{直接插入排序}}:按前驱后
                  继顺序依次遍历子序列元素,找出目标元素的合理插入位置
            \item 一些改进:二分/折半查找、Fibonacci查找、散列查找等
        \end{itemize}
    \end{block}
    \begin{block}{存储方式}
        \begin{itemize}
            \item 连续存储:须移动大量元素
            \item 链式存储:只需移动局部元素,较方便
        \end{itemize}
    \end{block}
\end{frame}

\begin{frame}
    \frametitle{\insertsectionhead}
    \begin{alertblock}{排序算法的稳定性}
        \begin{itemize}
            \item 若键相同的元素在排序前后的相对顺序保持不变,则称该排序算法为
                  \textbf{稳定的}
            \item 稳定排序算法适用于\alert{多键排序}\footnote{先按主键排序,在主
                      键重复的序列中按次键排序,以此类推。}
        \end{itemize}
    \end{alertblock}
\end{frame}

\begin{frame}
    \frametitle{\insertsectionhead}
    \begin{block}{直接插入排序算法性能分析}
        \begin{itemize}
            \item 时间复杂度:$O(n^{2})$
                  \begin{itemize}
                      \item 平均比较次
                            数:$\sum_{k=2}^{n}\frac{k}{2}=\frac{(n+2)(n-1)}{4}$
                  \end{itemize}
            \item 空间复杂度:$O(n)$
                  \begin{itemize}
                      \item 须额外一个元素的空间缓存待交换元素
                  \end{itemize}
            \item 稳定性
                  \begin{itemize}
                      \item 在插入元素时若遇键相同的情况可小心处理顺序,故该算法
                            \alert{稳定}
                  \end{itemize}
        \end{itemize}
    \end{block}
\end{frame}

\begin{fragile}
    \frametitle{\insertsectionhead}
    \begin{block}{选择排序:择优录取}
        \begin{itemize}
            \item 将序列$s$分为\alert{无序}与\alert{有序}两个子序列$s_{u}=s$与
                  $s_{o}=\emptyset$
            \item 反复从$s_{u}$中取出键最小的元素并放至$s_{o}$末尾,直至
                  $s_{u}=\emptyset$,则$s_{o}$即为所求
        \end{itemize}
    \end{block}
    \bicolumns[0.5]{
        \begin{block}{算法描述}
            \begin{algorithmx}[alg:selection]{选择排序}
                \Procedure{SelectionSort}{$s,f,t$}
                \State \Comment{对序列$s$中下标区间$[f,t)$内的元素进行选择排序}
                \State $i\gets f$
                \While{$i<t$}
                \State $k\gets$ SelectElement($s,i,t$)\Comment{选择元素子算法}
                \State Swap($s_{i},s_{k}$)\Comment{交换元素}
                \State $i\gets i+1$
                \EndWhile
                \State \textbf{return} $s$\Comment{返回排序后的序列$s$}
                \EndProcedure
            \end{algorithmx}
        \end{block}
    }{
        \begin{exampleblock}{实例}
            \begin{itemize}
                \item 原序列:$()21, 25, 22, \alert{10}, 25, 18$
                \item 第一趟:$(10), 25, 22, 21, 25, \alert{18}$
                \item 第二趟:$(10, 18), 25, 22, \alert{21}, 25$
                \item 第三趟:$(10, 18, 21), 25, \alert{22}, 25$
                \item 第四趟:$(10, 18, 21, 22), \alert{25}, 25$
                \item 第五趟:$(10, 18, 21, 22, 25), \alert{25}$
                \item 第六趟:$(10, 18, 21, 22, 25, 25)$
            \end{itemize}
        \end{exampleblock}
    }
\end{fragile}

\begin{frame}
    \frametitle{\insertsectionhead}
    \begin{block}{选择元素}
        \begin{itemize}
            \item 在\alert{无序序列}中\alert{查找}键最小元素
        \end{itemize}
    \end{block}
    \begin{block}{可选思路}
        \begin{itemize}
            \item \textbf{冒泡排序}:通过逐个交换相邻元素的方式\alert{选出}键最
                  小元素
            \item \textbf{直接选择排序}:按前驱后继顺序依次遍历子序列元
                  素,\alert{找出}键最小元素
            \item \textbf{堆排序}:采用\alert{优先队列}组织无序子序列
        \end{itemize}
    \end{block}
\end{frame}

\begin{frame}
    \frametitle{\insertsectionhead}
    \begin{block}{直接选择排序的性能分析}
        \begin{itemize}
            \item 与直接插入排序类似,略
        \end{itemize}
    \end{block}
\end{frame}

\begin{fragile}
    \frametitle{\insertsectionhead}
    \begin{block}{快速排序:分治策略的应用}
        \begin{itemize}
            \item 将序列\alert{划分}为三部分:任意元素$s_{m}$作为\alert{支点}、
                  所有元素分别比$s_{m}$小与大的两个子序列
            \item 对后两个子序列递归执行上述\alert{划分}策略,直至所有子序列为空
        \end{itemize}
    \end{block}
    \bicolumns[0.5]{
        \begin{block}{算法描述}
            \begin{algorithmx}[alg:quick]{快速排序}
                \Procedure{QuickSort}{$s,l,h$}
                \State \Comment{对序列$s$中下标区间$[l,h]$内的元素进行选择排序}
                \While{$l<h$}
                \State $m\gets$ Partition($s,l,h$)\Comment{序列划分子算法}
                \State QuickSort($s,l,m-1$)\Comment{左侧递归}
                \State QuickSort($s,m+1,h$)\Comment{右侧递归}
                \EndWhile
                \State \textbf{return} $s$\Comment{返回排序后的序列$s$}
                \EndProcedure
            \end{algorithmx}
        \end{block}
    }{
        \begin{exampleblock}{实例}
            \begin{itemize}
                \item 原序列:$21, 25, 22, 10, 25, 18$
                \item 第一趟:$18, 10, \alert{21}, 22, 25, 25$
                \item 第二趟:$10, \alert{18}, \alert{21}, \alert{22}, 25, 25$
                \item 第三趟:$\alert{10}, \alert{18}, \alert{21}, \alert{22}, \alert{25}, 25$
                \item 第四趟:$\alert{10}, \alert{18}, \alert{21}, \alert{22}, \alert{25}, \alert{25}$
            \end{itemize}
        \end{exampleblock}
    }
\end{fragile}

\begin{fragile}
    \frametitle{\insertsectionhead}
    \bicolumns[0.5]{
        \begin{block}{划分算法}
            \begin{minted}{c}
                int partition(DataType *s, int l, int h,
                              CompareType c) {
                    DataType pivot = s[l]; // 缓存支点元素
                    for (;;) {
                        for (; l < h && c(&pivot, s + h) < 0; --h)
                            ; // 左移h指针至须交换处
                        if (l >= h) break;
                        s[l++] = s[h]; // 交换元素并右移l指针
                        for (; l < h && c(s + l, &pivot) < 0; ++l)
                            ; // 右移l指针至须交换处
                        if (l >= h) break;
                        s[h--] = s[l]; // 交换元素并左移h指针
                    } // 此时l与h二指针重合,结束循环
                    s[l] = pivot; // 放回支点元素
                    return l; // 返回支点元素新下标
                }
            \end{minted}
        \end{block}
    }{
        \begin{exampleblock}{实例:一次划分}
            \begin{itemize}
                \item 原始序列:$21_{l}, 25, 22, 10, 25, 18_{h}, (\,\,\,\,\,\,)$
                \item 缓存支点:$\alert{\,\,\,\,\,\,_{l}}, 25, 22, 10, 25, \alert{18_{h}},(21_{m})$
                \item \alert{交换元素}:$\alert{18_{l}}, 25, 22, 10, 25, \alert{\,\,\,\,\,\,_{h}},(21_{m})$
                \item 移动指针:$18, \alert{25_{l}}, 22, 10, 25, \alert{\,\,\,\,\,\,_{h}},(21_{m})$
                \item \alert{交换元素}:$18, \alert{\,\,\,\,\,\,_{l}}, 22, 10, 25, \alert{25_{h}},(21_{m})$
                \item 移动指针:$18, \alert{\,\,\,\,\,\,_{l}}, 22, 10, \alert{25_{h}}, 25,(21_{m})$
                \item 移动指针:$18, \alert{\,\,\,\,\,\,_{l}}, 22, \alert{10_{h}}, 25, 25,(21_{m})$
                \item \alert{交换元素}:$18, \alert{10_{l}}, 22, \alert{\,\,\,\,\,\,_{h}}, 25, 25,(21_{m})$
                \item 移动指针:$18, 10, \alert{22_{l}}, \alert{\,\,\,\,\,\,_{h}}, 25, 25,(21_{m})$
                \item \alert{交换元素}:$18, 10, \alert{\,\,\,\,\,\,_{l}}, \alert{22_{h}}, 25, 25,(21_{m})$
                \item 移动指针:$18, 10, \alert{\,\,\,\,\,\,_{lh}}, 22, 25, 25,(21_{m})$
                \item 放回支点:$18, 10, \alert{21_{m}}, 22, 25, 25,(\,\,\,\,\,\,)$
            \end{itemize}
        \end{exampleblock}
    }
\end{fragile}

\begin{frame}
    \frametitle{\insertsectionhead}
    \begin{block}{快速排序的性能分析}
        \begin{itemize}
            \item 时间复杂度
                  \begin{itemize}
                      \item 最好情况:$O(n\log n)$
                      \item 最坏情况:$O(n^{2})$
                      \item 平均情况:$O(n\log n)$
                  \end{itemize}
            \item 空间复杂度:$O(n)$
            \item 稳定性
                  \begin{itemize}
                      \item 由于交换过程会越过支点元素且无法控制,故该算法
                            \alert{不稳定}
                  \end{itemize}
        \end{itemize}
    \end{block}
\end{frame}

\begin{frame}
    \frametitle{\insertsectionhead}
    \begin{alertblock}{各种排序算法的取舍}
        \begin{itemize}
            \item 问题规模较小\footnote{一般认为待排序序列元素不超过$23$个}:直
                  接选择排序、直接插入排序
            \item 问题规模略大:快速排序、归并排序、堆排序
            \item 当需要多键排序时,宜选用稳定算法:直接选择排序、直接插入排序、
                  堆排序
        \end{itemize}
    \end{alertblock}
\end{frame}