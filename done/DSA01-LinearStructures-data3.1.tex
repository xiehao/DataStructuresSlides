\subsection{栈}

\begin{fragile}
    \frametitle{\insertsubsectionhead}
    \bicolumns[0.6]{
        \begin{block}{栈(Stack)}
            \begin{itemize}
                \item 一种操作\alert{受限}的序列
                \item \alert{仅}允许在一端插入删除元素
                \item 可操作一端为\textbf{栈顶(top)},另一端为\textbf{栈底(bottom)}
                \item \textbf{后进先出(Last In First Out, LIFO)}
            \end{itemize}
        \end{block}
        \uncover<2>{
            \begin{exampleblock}{日常生活中的栈}
                \begin{itemize}
                    \item 一摞碗、碟、凳子等
                    \item 糖葫芦、牛羊肉串等
                    \item 子弹夹
                    \item $\dots$
                \end{itemize}
            \end{exampleblock}
        }
    }{
        \begin{figure}
            \centering
            \begin{bytefield}{6}
                \stackbox{4.5}{6}{$\vdots$\\\vspace*{1ex}$\uparrow\downarrow$} \\
                \begin{leftwordgroup}{$n$}
                    \begin{rightwordgroup}{\mintinline[fontsize=\smaller]{c}{栈顶}}
                        \stackbox{1.5}{6}{$a_{n}$}
                    \end{rightwordgroup} \\
                    \stackbox{2.5}{6}{$\vdots$\vspace{1.5ex}} \\
                    \stackbox{1.5}{6}{$a_{2}$} \\
                    \begin{rightwordgroup}{\mintinline[fontsize=\smaller]{c}{栈底}}
                        \stackbox{1.5}{6}{$a_{1}$}
                    \end{rightwordgroup}
                \end{leftwordgroup}
            \end{bytefield}
            \caption{栈的示意图}
            \label{fig:demo_stack}
        \end{figure}
    }
\end{fragile}

\begin{fragile}
    \frametitle{\insertsectionhead}
    \begin{block}{栈的抽象数据类型}
        \begin{minted}[linenos=false,escapeinside=@@]{c}
            ADT Stack {
            数据:
                数据对象: @$\mathcal{D} = \{a_{k} | a_{k}\in\text{数据元素集合}, k\in\mathbb{Z}\cap[1,n]\}$@
                逻辑关系: @$\mathcal{R} = \{\langle{}a_{k-1},a_{k}\rangle | k\in\mathbb{Z}\cap[2,n]\}$@
            操作:
                create_stack()
                    构造并初始化一个空栈@$s$@
                is_empty_stack()
                    判断栈是否为空
                push_stack(s, x)
                    若栈@$s$@存在且未满, 则在其栈顶插入值为@$x$@的新元素
                pop_stack(s)
                    若栈@$s$@存在且非空, 则删除其栈顶元素
                top_stack(s)
                    若栈@$s$@存在且非空, 则返回其栈顶元素
            }
        \end{minted}
    \end{block}
\end{fragile}

\begin{fragile}
    \frametitle{\insertsubsectionhead}
    \bicolumns[0.5]{
        \begin{block}{栈的类型说明}
            \begin{itemize}
                \item 可利用已有结构的再封装
                \item \alert{顺序栈}可采用序列
                \item \alert{链式栈}可采用链表
            \end{itemize}
        \end{block}
    }{
        \begin{minted}[highlightlines={}]{c}
            typedef struct {
                SequenceList *_;
            } SequenceStack;
        \end{minted}
        \begin{minted}[highlightlines={}]{c}
            typedef struct {
                LinkedList *_;
            } LinkedStack;
        \end{minted}
    }
\end{fragile}

\begin{fragile}
    \frametitle{\insertsubsectionhead}
    \bicolumns[0.5]{
        \begin{block}{栈的初始化}
            \begin{itemize}
                \item 可利用序列/链表已有的初始化方法
                \item 注意空指针的处理
            \end{itemize}
        \end{block}
    }{
        \begin{minted}[highlightlines={}]{c}
            SequenceStack *create_sequence_stack() {
                SequenceStack *s = malloc(sizeof(SequenceStack));
                if (s) {
                    s->_ = create_sequence_list();
                }
                return s;
            }
        \end{minted}
        \begin{minted}[highlightlines={}]{c}
            LinkedStack *create_linked_stack() {
                LinkedStack *s = malloc(sizeof(LinkedStack));
                if (s) {
                    s->_ = create_linked_list();
                }
                return s;
            }
        \end{minted}
    }
\end{fragile}

\begin{fragile}
    \frametitle{\insertsubsectionhead}
    \bicolumns[0.5]{
        \begin{block}{判断栈是否为空}
            \begin{itemize}
                \item 顺序栈可利用序列已有的判空方法
                \item 链式栈需检测首结点的下一个结点
                \item 此处利用了短路求值原则,下同
            \end{itemize}
        \end{block}
    }{
        \begin{minted}[highlightlines={}]{c}
            bool empty_sequence_stack(SequenceStack *s) {
                return !s || empty_sequence_list(s->_);
            }
        \end{minted}
        \begin{minted}[highlightlines={}]{c}
            bool empty_linked_stack(LinkedStack *s) {
                return !s || !s->_->head->next;
            }
        \end{minted}
    }
\end{fragile}

\begin{fragile}
    \frametitle{\insertsubsectionhead}
    \bicolumns[0.5]{
        \begin{block}{入栈}
            \begin{itemize}
                \item 将新元素\alert{压入}栈中
                \item 指将新元素放入栈顶并更新栈顶
                \item 顺序栈栈顶序号为\mintinline[fontsize=\smaller]{c}{last}
                \item 链式栈栈顶为首结点的直接后继
            \end{itemize}
        \end{block}
    }{
        \begin{minted}[highlightlines={}]{c}
            bool push_sequence_stack(
                    SequenceStack *s, DataType d) {
                return s && insert_sequence_list(
                        s->_, s->_->last + 1, d);
            }
        \end{minted}
        \begin{minted}[highlightlines={}]{c}
            bool push_linked_stack(
                    LinkedStack *s, DataType d) {
                return s && insert_after_linked_by_index(
                        s->_, 0, d);
            }
        \end{minted}
    }
\end{fragile}

\begin{fragile}
    \frametitle{\insertsubsectionhead}
    \bicolumns[0.5]{
        \begin{block}{出栈}
            \begin{itemize}
                \item 将栈顶元素从栈中\alert{弹出}
                \item 指将栈顶元素取出并更新栈顶
            \end{itemize}
        \end{block}
    }{
        \begin{minted}[highlightlines={}]{c}
            bool pop_sequence_stack(
                    SequenceStack *s, DataType *p) {
                return s && remove_sequence_list(
                        s->_, s->_->last, p);
            }
        \end{minted}
        \begin{minted}[highlightlines={}]{c}
            bool pop_linked_stack(
                    LinkedStack *s, DataType *p) {
                return s && remove_linked_by_index(
                        s->_, 1, p);
            }
        \end{minted}
    }
\end{fragile}

\begin{fragile}
    \frametitle{\insertsubsectionhead}
    \bicolumns[0.5]{
        \begin{block}{取栈顶元素}
            \begin{itemize}
                \item 将栈顶元素\alert{复制}一份
                \item 不改变栈
                \item 首先判断栈是否为空
            \end{itemize}
        \end{block}
    }{
        \begin{minted}[highlightlines={}]{c}
            bool top_sequence_stack(
                    SequenceStack *s, DataType *p) {
                if (empty_sequence_stack(s) || !p) {
                    return false;
                }
                *p = s->_->data[s->_->last];
                return true;
            }
        \end{minted}
        \begin{minted}[highlightlines={}]{c}
            bool top_linked_stack(
                    LinkedStack *s, DataType *p) {
                if (empty_linked_stack(s) || !p) {
                    return false;
                }
                *p = s->_->head->next->data;
                return true;
            }
        \end{minted}
    }
\end{fragile}

\begin{fragile}
    \frametitle{\insertsubsectionhead}
    \begin{exampleblock}{栈与递归}
        \begin{itemize}
            \item \textbf{递归(recursion)}:函数直接或间接调用自身的过程
            \item 递归\textbf{要素}:终止条件与递归体
            \item 递归\textbf{实现}:在\alert{栈}中记录每次调用的有用信息
        \end{itemize}
    \end{exampleblock}
    \pause
    \begin{exampleblock}{例:求阶乘}
        \vspace{2ex}
        \bicolumns[0.5]{
            \vspace*{1.5ex}
            \[
                n!=\begin{cases}
                    1            & n=0 \\
                    n\cdot(n-1)! & n>0
                \end{cases}
            \]
        }{
            \begin{minted}{c}
                int factorial(int n) {
                    return 0 == n ? 1 : n * factorial(n - 1);
                }
            \end{minted}
        }
    \end{exampleblock}
\end{fragile}

\begin{fragile}
    \frametitle{\insertsubsectionhead}
    \bicolumns[0.5]{
        \begin{figure}
            \centering
            \begin{bytefield}{2}
                \stackbox{3}{12}{\only<5>{
                        \mintinline[fontsize=\smaller]{console}{factorial()} \\
                        \mintinline[fontsize=\smaller]{console}{return address = 1002} \\
                        \mintinline[fontsize=\smaller]{console}{n = 0, ...}
                    }
                } \\
                \stackbox{3}{12}{\only<4-6>{
                        \mintinline[fontsize=\smaller]{console}{factorial()} \\
                        \mintinline[fontsize=\smaller]{console}{return address = 1002} \\
                        \mintinline[fontsize=\smaller]{console}{n = 1, ...}
                    }
                } \\
                \stackbox{3}{12}{\only<3-7>{
                        \mintinline[fontsize=\smaller]{console}{factorial()} \\
                        \mintinline[fontsize=\smaller]{console}{return address = 2003} \\
                        \mintinline[fontsize=\smaller]{console}{n = 2, ...}
                    }
                } \\
                \stackbox{2}{12}{\only<2-8>{
                        \mintinline[fontsize=\smaller]{console}{main()}\\
                        \mintinline[fontsize=\smaller]{console}{n = 2, ...}
                    }
                }
            \end{bytefield}
            \caption{函数调用栈:\only<2-5>{递进调用}\only<6-9>{回归求值}}
            \label{fig:function_call_stack}
        \end{figure}
    }{
        \begin{minted}[firstnumber=1001,highlightlines={1002}]{c}
            int factorial(int n) {
                return 0 == n ? 1 : n * factorial(n - 1);
            }
        \end{minted}
        \begin{minted}[firstnumber=2001,highlightlines={2003}]{c}
            int main(void) {
                int n = 2;
                int f = factorial(n);
                return 0;
            }
        \end{minted}
    }
\end{fragile}

\begin{frame}
    \frametitle{\insertsubsectionhead}
    \begin{exampleblock}{递归应用:汉诺塔(Hanoi)问题}
        \begin{itemize}
            \item 相传世界之初有钻石宝塔甲,其上有$64$金碟,由大到小依次向上摆放
            \item 附近另有二空塔:乙、丙,与甲类似
            \item 婆罗门牧师试图将甲塔金碟移至丙塔,但需借乙塔中转
            \item 每次只移最上一碟,且不可将大碟置于小碟之上
            \item 牧师完成之日便是世界末日
        \end{itemize}
    \end{exampleblock}
\end{frame}

\begin{fragile}
    \frametitle{\insertsubsectionhead}
    \bicolumns[0.5]{
        \begin{exampleblock}{递归解题思路}
            \begin{enumerate}
                \item 若剩余碟数$n=0$,则结束
                \item 否则执行:
                      \begin{itemize}
                          \item 将$n-1$碟从甲借丙移至乙
                          \item 将甲中剩余一碟从甲移至丙
                          \item 将$n-1$碟从乙借甲移至丙
                      \end{itemize}
            \end{enumerate}
        \end{exampleblock}
    }{
        \begin{minted}{c}
            void move(char a, char b) {
                printf("%c -> %c\n", a, b);
            }
        \end{minted}
        \begin{minted}{c}
            void hanoi(int n, char a, char b, char c) {
                if (!n) {
                    return; // 递归出口
                }
                hanoi(n - 1, a, c, b);
                move(a, c);
                hanoi(n - 1, b, a, c);
            }
        \end{minted}
    }
\end{fragile}

\begin{frame}
    \frametitle{\insertsubsectionhead}
    \begin{alertblock}{注意}
        \begin{itemize}
            \item 递归可在抽象层面帮助快速理清思路
            \item 递归需借助栈结构存储额外信息,故效率较迭代/循环\alert{低}
            \item 在实现中,栈容量有限,故过多层递归极易\alert{爆栈}
        \end{itemize}
    \end{alertblock}
\end{frame}
% \begin{fragile}
    \frametitle{\insertsubsectionhead}
    \bicolumns[0.45]{
        \begin{exampleblock}{应用实例:迷宫寻路}
            \begin{itemize}
                \item 输入:迷宫
                \begin{itemize}
                    \item 由$\mathbb{R}^{m\times{}n}$上的矩阵表示
                    \item 矩阵元素可为\alert{墙壁}或\alert{地面}
                \end{itemize}
                \item 输出:自\alert{始}至\alert{终}的一条\alert{通路}
                \begin{itemize}
                    \item 通路由\alert{相邻}的地面序列组成
                    \item 相邻指\alert{八邻域}
                \end{itemize}
            \end{itemize}
        \end{exampleblock}
    }{
        \begin{figure}
            \centering
            \begin{bytefield}{2}
                \bitboxes{2}{{X}{\alert{始}}{X}{X}{X}{X}{X}{X}} \\
                \bitboxes{2}{{X}{\only<2>{$\checkmark$}}{X}{\only<2>{$\checkmark$}}{ }{ }{X}{X}} \\
                \bitboxes{2}{{X}{X}{\only<2>{$\checkmark$}}{X}{\only<2>{$\checkmark$}}{ }{ }{X}} \\
                \bitboxes{2}{{X}{X}{ }{X}{\only<2>{$\checkmark$}}{X}{X}{X}} \\
                \bitboxes{2}{{X}{X}{ }{X}{\only<2>{$\checkmark$}}{ }{ }{X}} \\
                \bitboxes{2}{{X}{ }{X}{X}{X}{\only<2>{$\checkmark$}}{X}{X}} \\
                \bitboxes{2}{{X}{X}{ }{X}{ }{\only<2>{$\checkmark$}}{X}{X}} \\
                \bitboxes{2}{{X}{X}{X}{X}{X}{X}{\alert{终}}{X}}
            \end{bytefield}
            \caption{迷宫示意图(X表示墙壁)}
            \label{fig:demo_maze}
        \end{figure}
    }
\end{fragile}

\begin{fragile}
    \frametitle{\insertsubsectionhead}
    \bicolumns[0.45]{
        \begin{exampleblock}{回溯法(Backtracking)思想}
            \begin{itemize}
                \item 一种试探纠错的搜索方法
            \end{itemize}
            \begin{enumerate}
                \item 自初始点出发
                \item 初始化搜索方向
                \item 按搜索方向向前搜索
                \item 判断搜索结果
                \begin{itemize}
                    \item 若到达终止点,则成功停止
                    \item 否则切换新搜索方向并转向步骤3
                \end{itemize}
                \item \alert{沿原路返回}上一结点并转向步骤2
                \item 失败停止
            \end{enumerate}
        \end{exampleblock}
    }{
        \begin{figure}
            \centering
            \begin{bytefield}{2}
                \bitboxes{2}{{X}{\alert{始}}{X}{X}{X}{X}{X}{X}} \\
                \bitboxes{2}{{X}{\only<2->{\alert{$\checkmark$}}}{X}{\only<12->{\alert{$\checkmark$}}}{ }{ }{X}{X}} \\
                \bitboxes{2}{{X}{X}{\only<3->{\alert{$\checkmark$}}}{X}{\only<13->{\alert{$\checkmark$}}}{ }{ }{X}} \\
                \bitboxes{2}{{X}{X}{\only<4->{\alert<4-10>{$\checkmark$}}}{X}{\only<14->{\alert{$\checkmark$}}}{X}{X}{X}} \\
                \bitboxes{2}{{X}{X}{\only<5->{\alert<5-9>{$\checkmark$}}}{X}{\only<15->{\alert{$\checkmark$}}}{ }{ }{X}} \\
                \bitboxes{2}{{X}{\only<6->{\alert<6-8>{$\checkmark$}}}{X}{X}{X}{\only<16->{\alert{$\checkmark$}}}{X}{X}} \\
                \bitboxes{2}{{X}{X}{\only<7->{\alert<7>{$\checkmark$}}}{X}{ }{\only<17->{\alert{$\checkmark$}}}{X}{X}} \\
                \bitboxes{2}{{X}{X}{X}{X}{X}{X}{\alert<18>{终}}{X}}
            \end{bytefield}
            \caption{回溯法一种可能的过程(X表示墙壁)}
            \label{fig:demo_backtracking}
        \end{figure}
    }
\end{fragile}

\begin{frame}
    \frametitle{\insertsubsectionhead}
    \begin{alertblock}{思考}
        \begin{itemize}
            \item 以何种数据结构表达迷宫?
            \item 如何表达寻路试探方向?
            \item 如何记录已试探结点以便原路返回?
            \item 如何避免重复试探相同结点?
        \end{itemize}
    \end{alertblock}
\end{frame}

\begin{fragile}
    \frametitle{\insertsubsectionhead}
    \bicolumns[0.45]{
        \begin{exampleblock}{表达迷宫的数据结构}
            \begin{itemize}
                \item 可采用二维整型数组
                \begin{itemize}
                    \item 0:地面
                    \item 1:障碍
                \end{itemize}
                \item 为实现方便可外加一圈障碍
            \end{itemize}
        \end{exampleblock}
        \begin{minted}{c}
            #define WIDTH 8
            #define HEIGHT 8

            static int map[HEIGHT + 2][WIDTH + 2];
        \end{minted}
    }{
        \begin{figure}
            \centering
            \begin{bytefield}{2}
                \bitboxes{2}{{1}{1}{1}{1}{1}{1}{1}{1}{1}{1}} \\
                \bitboxes{2}{{1}{1}{\alert{0}}{1}{1}{1}{1}{1}{1}{1}} \\
                \bitboxes{2}{{1}{1}{0}{1}{0}{0}{0}{1}{1}{1}} \\
                \bitboxes{2}{{1}{1}{1}{0}{1}{0}{0}{0}{1}{1}} \\
                \bitboxes{2}{{1}{1}{1}{0}{1}{0}{1}{1}{1}{1}} \\
                \bitboxes{2}{{1}{1}{1}{0}{1}{0}{0}{0}{1}{1}} \\
                \bitboxes{2}{{1}{1}{0}{1}{1}{1}{0}{1}{1}{1}} \\
                \bitboxes{2}{{1}{1}{1}{0}{1}{0}{0}{1}{1}{1}} \\
                \bitboxes{2}{{1}{1}{1}{1}{1}{1}{1}{\alert{0}}{1}{1}} \\
                \bitboxes{2}{{1}{1}{1}{1}{1}{1}{1}{1}{1}{1}}
            \end{bytefield}
            \caption{迷宫数据结构}
            \label{fig:data_structure_maze}
        \end{figure}
    }
\end{fragile}

\begin{fragile}
    \frametitle{\insertsubsectionhead}
    \bicolumns[0.55]{
        \begin{exampleblock}{寻路的试探方向}
            \begin{itemize}
                \item 可采用二维向量
            \end{itemize}
        \end{exampleblock}
        \begin{figure}
            \centering
            \begin{bytefield}{2}
                \bitboxes{8}{{$(x-1,y-1)$}{$(x+0,y-1)$}{$(x+1,y-1)$}} \\
                \bitboxes{8}{{$(x-1,y+0)$}{$(x+0,y+0)$}{$(x+1,y+0)$}} \\
                \bitboxes{8}{{$(x-1,y+1)$}{$(x+0,y+1)$}{$(x+1,y+1)$}}
            \end{bytefield}
            \caption{试探方向}
            \label{fig:searching_directions}
        \end{figure}
    }{
        \begin{minted}{c}
            #define N_DIRECTIONS 8

            typedef struct {
                int x, y;
            } Direction;

            static const Direction steps[N_DIRECTIONS] = {
                {1, 0}, {1, 1}, {0, 1}, {-1, 1},
                {-1, 0}, {-1, -1}, {0, -1}, {1, -1},
            };
        \end{minted}
    }
\end{fragile}

\begin{fragile}
    \frametitle{\insertsubsectionhead}
    \bicolumns[0.5]{
        \begin{exampleblock}{结点记录与原路返回}
            \begin{itemize}
                \item 可采用\alert{栈}记录已访问结点
                \item 结点信息须包含\alert{位置}与\alert{方向}
            \end{itemize}
        \end{exampleblock}
    }{
        \begin{minted}{c}
            typedef struct {
                int x; // 横坐标
                int y; // 纵坐标
                int d; // 方向数组下标
            } Cell; // 结点类型
        \end{minted}
    }
\end{fragile}