\subsection{几种特殊的二叉树}

\begin{fragile}
    \frametitle{\insertsubsectionhead}
    \bicolumns[0.5]{
        \begin{block}{左斜树\footnotemark}
            \begin{itemize}
                \item 所有非叶结点均有且仅有$1$个左子树
                \item 每层均有且仅有$1$个结点
                \item 已退化为线性结构
            \end{itemize}
        \end{block}
    }{
        \begin{figure}
            \centering
            \begin{subfigure}[b]{0.3\textwidth}
                \centering
                \begin{tikzpicture}[ >=stealth, thick, black!50, %
                        list item/.style={draw=gray, circle, thick}, %
                        level distance=4ex, sibling distance=4ex, %
                    ]
                    \node [terminal] {$A$}
                        child {node [terminal] {$B$}
                            child {node [terminal] {$C$}
                                child {node [terminal] {$D$}}
                                child [missing]
                            }
                            child [missing]
                        }
                        child [missing];
                \end{tikzpicture}
                \caption{左斜树}
                \label{subfig:left_slanted_tree}
            \end{subfigure}
            ~
            \begin{subfigure}[b]{0.3\textwidth}
                \centering
                \begin{tikzpicture}[ >=stealth, thick, black!50, %
                        list item/.style={draw=gray, circle, thick}, %
                        level distance=4ex, sibling distance=4ex, %
                    ]
                    \node [terminal] {$A$}
                        child [missing]
                        child {node [terminal] {$B$}
                            child [missing]
                            child {node [terminal] {$C$}
                                child [missing]
                                child {node [terminal] {$D$}}
                            }
                        };
                \end{tikzpicture}
                \caption{右斜树}
                \label{subfig:right_slanted_tree}
            \end{subfigure}
            \caption{斜树}
            \label{fig:demo_slanted_tree}
        \end{figure}
    }

    \footnotetext{右斜树与之类似,只需将左改作右}
\end{fragile}

\begin{fragile}
    \frametitle{\insertsubsectionhead}
    \bicolumns[0.5]{
        \begin{block}{满二叉树(full binary tree)}
            \begin{itemize}
                \item 所有叶结点均在最后一层
                \item 所有非叶结点度均为$2$
            \end{itemize}
        \end{block}
        \begin{exampleblock}{性质}
            \begin{itemize}
                \item 同深度的二叉树中满二叉树结点最多
                \item 同深度的二叉树中满二叉树叶结点最多
            \end{itemize}
        \end{exampleblock}
    }{
        \begin{figure}
            \centering
                \begin{tikzpicture}[ >=stealth, thick, black!50, %
                        list item/.style={draw=gray, circle, thick}, %
                        level distance=4ex, %
                        level 1/.style={sibling distance=16ex}, %
                        level 2/.style={sibling distance=8ex}, %
                        level 3/.style={sibling distance=4ex}, %
                    ]
                    \node [terminal] {$A$}
                        child {node [terminal] {$B$}
                            child {node [terminal] {$D$}
                                child {node [terminal] {$H$}}
                                child {node [terminal] {$I$}}
                            }
                            child {node [terminal] {$E$}
                                child {node [terminal] {$J$}}
                                child {node [terminal] {$K$}}
                            }
                        }
                        child {node [terminal] {$C$}
                            child {node [terminal] {$F$}
                                child {node [terminal] {$L$}}
                                child {node [terminal] {$M$}}
                            }
                            child {node [terminal] {$G$}
                                child {node [terminal] {$N$}}
                                child {node [terminal] {$O$}}
                            }
                        };
                \end{tikzpicture}
            \caption{满二叉树}
            \label{fig:demo_full_binary_tree}
        \end{figure}
    }
\end{fragile}

\begin{fragile}
    \frametitle{\insertsubsectionhead}
        \begin{figure}
            \centering
            \begin{subfigure}[b]{0.4\textwidth}
                \centering
                \begin{tikzpicture}[ >=stealth, thick, black!50, %
                        list item/.style={draw=gray, circle, thick}, %
                        level distance=4ex, %
                        level 1/.style={sibling distance=16ex}, %
                        level 2/.style={sibling distance=8ex}, %
                        level 3/.style={sibling distance=4ex}, %
                    ]
                    \node [terminal] {$A$}
                        child {node [terminal] {$B$}
                            child {node [terminal] {$D$}
                                child {node [terminal] {$H$}}
                                child {node [terminal] {$I$}}
                            }
                            child {node [terminal] {$E$}
                                child {node [terminal] {$J$}}
                                child {node [terminal] {$K$}}
                            }
                        }
                        child {node [terminal] {$C$}
                            child {node [terminal] {$F$}
                                child {node [terminal] {$L$}}
                                child {node [terminal] {$M$}}
                            }
                            child {node [terminal] {\alert{$G$}}}
                        };
                \end{tikzpicture}
                \caption{叶结点不在同层}
                \label{subfig:non_full_binary_tree_a}
            \end{subfigure}
            ~
            \begin{subfigure}[b]{0.4\textwidth}
                \centering
                \begin{tikzpicture}[ >=stealth, thick, black!50, %
                        list item/.style={draw=gray, circle, thick}, %
                        level distance=4ex, %
                        level 1/.style={sibling distance=16ex}, %
                        level 2/.style={sibling distance=8ex}, %
                        level 3/.style={sibling distance=4ex}, %
                    ]
                    \node [terminal] {$A$}
                        child {node [terminal] {$B$}
                            child {node [terminal] {$D$}
                                child {node [terminal] {$H$}}
                                child {node [terminal] {$I$}}
                            }
                            child {node [terminal] {$E$}
                                child {node [terminal] {$J$}}
                                child {node [terminal] {$K$}}
                            }
                        }
                        child {node [terminal] {$C$}
                            child {node [terminal] {$F$}
                                child {node [terminal] {$L$}}
                                child {node [terminal] {$M$}}
                            }
                            child {node [terminal] {\alert{$G$}}
                                child {node [terminal] {$N$}}
                                child [missing]
                            }
                        };
                \end{tikzpicture}
                \caption{个别非叶结点度不为$2$}
                \label{subfig:non_full_binary_tree_b}
            \end{subfigure}
            \caption{非满二叉树}
            \label{fig:demo_non_full_binary_tree}
        \end{figure}
\end{fragile}

\begin{fragile}
    \frametitle{\insertsubsectionhead}
    \bicolumns[0.5]{
        \begin{block}{完全二叉树(proper binary tree)}
            \begin{itemize}
                \item 若去除最后一层结点,则为满二叉树
                \item 最后一层结点自左至右连续\footnotemark{}排列
            \end{itemize}
        \end{block}
        \begin{exampleblock}{性质}
            \begin{itemize}
                \item 同结点数的二叉树中完全二叉树最矮
                \item 满二叉树亦为完全二叉树的一种
            \end{itemize}
        \end{exampleblock}
    }{
        \begin{figure}
            \centering
                \begin{tikzpicture}[ >=stealth, thick, black!50, %
                        list item/.style={draw=gray, circle, thick}, %
                        level distance=4ex, %
                        level 1/.style={sibling distance=16ex}, %
                        level 2/.style={sibling distance=8ex}, %
                        level 3/.style={sibling distance=4ex}, %
                    ]
                    \node [terminal] {$A$}
                        child {node [terminal] {$B$}
                            child {node [terminal] {$D$}
                                child {node [terminal] {$H$}}
                                child {node [terminal] {$I$}}
                            }
                            child {node [terminal] {$E$}
                                child {node [terminal] {$J$}}
                                child [missing]
                            }
                        }
                        child {node [terminal] {$C$}
                            child {node [terminal] {$F$}}
                            child {node [terminal] {$G$}}
                        };
                \end{tikzpicture}
            \caption{完全二叉树}
            \label{fig:demo_proper_binary_tree}
        \end{figure}
    }

    \footnotetext{指中间无空结点}
\end{fragile}