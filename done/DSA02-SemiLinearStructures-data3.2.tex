\subsection{二叉树的性质}

\begin{frame}
    \frametitle{\insertsubsectionhead}
    \begin{block}{性质甲}
        \begin{itemize}
            \item 令$N$层二叉树第$k$层结点数为$n_{l}(k)$,则有
            \[
                1\leq{}n_{l}(k)\leq2^{k},\;k\in\mathbb{Z}\cap[0,N)
            \]
        \end{itemize}
    \end{block}
    \pause
    \begin{exampleblock}{证明}
        \begin{enumerate}
            \item 左侧不等式显然成立
            \item 右侧不等式当$k=0$时,第$0$层仅有根结点,故$n_{l}(0)=1=2^{0}$显
                  然成立
            \item 假设当$k=n<N-1$时成立,即$n_{l}(n)\leq2^{n}$,因所有结点的度均
                  不大于$2$,故
                  \[
                        n_{l}(n+1)\leq2\cdot{}n_{l}(n)\leq2\cdot2^{n}=2^{n+1}
                  \]
                  于是当$k=n+1$时,归纳假设成立\hfill$\qed$
        \end{enumerate}
    \end{exampleblock}
\end{frame}

\begin{frame}
    \frametitle{\insertsubsectionhead}
    \begin{block}{性质乙}
        \begin{itemize}
            \item 令$k$层二叉树的结点总数为$n(k)$,则有
                  \[
                        k\leq{}n(k)\leq2^{k}-1
                  \]
        \end{itemize}
    \end{block}
    \pause
    \begin{exampleblock}{证明}
        \begin{enumerate}
            \item 由性质甲,$1\leq{}n_{l}(i)\leq2^{i},\;i\in\mathbb{Z}\cap[0,k)$
            \item 经累加后,有
                  \[
                        k=\sum_{i=0}^{k-1}{1}\leq{}n(k)=\sum_{i=0}^{k-1}{n_{l}(i)}\leq\sum_{i=0}^{k-1}{2^{i}}=2^{k}-1
                  \]
                  \hfill$\qed$
        \end{enumerate}
    \end{exampleblock}
\end{frame}

\begin{frame}
    \frametitle{\insertsubsectionhead}
    \begin{alertblock}{思考}
        \begin{itemize}
            \item 满足$n(k)=k$的二叉树一定是斜树么?
            \item 满足$n(k)=2^{k}-1$的二叉树一定是满二叉树么?
        \end{itemize}
    \end{alertblock}
\end{frame}

\begin{frame}
    \frametitle{\insertsubsectionhead}
    \begin{block}{性质丙}
        \begin{itemize}
            \item 令二叉树中度为$d$的结点数为$n_{d},\;(d\in\{0,1,2\})$,则有
                  \[
                        n_{0}=n_{2}+1
                  \]
        \end{itemize}
    \end{block}
    \pause
    \begin{exampleblock}{证明}
        \begin{enumerate}
            \item 一方面,结点总数由各种度的结点构成,故总结点数$n$满足
                  \[
                        n=\sum_{d=0}^{2}{n_{d}}=n_{0}+n_{1}+n_{2}
                  \]
            \item 另一方面,每个非根结点均由$1$个结点生成,故度为$d$的结点可生成
                  $d$个结点,即
                  \[
                        n=\sum_{d=0}^{2}{d\cdot{}n_{d}}+1=n_{1}+2n_{2}+1
                  \]
            \item 综上得证\hfill$\qed$
        \end{enumerate}
    \end{exampleblock}
\end{frame}

\begin{frame}
    \frametitle{\insertsubsectionhead}
    \begin{alertblock}{思考}
        \begin{itemize}
            \item 有$n$个结点的完全二叉树有多少叶结点?
        \end{itemize}
    \end{alertblock}
    \pause
    \begin{exampleblock}{提示}
        \begin{itemize}
            \item 在完全二叉树中,度为$1$的结点数不多于$1$
            \item 当$n$为偶数时,$n_{1}=1$,$n_{0}=n/2$,$n_{2}=n/2-1$
            \item 当$n$为奇数时,$n_{1}=0$,$n_{0}=(n+1)/2$,$n_{2}=(n-1)/2$
            \item 故$n_{0}=\lceil{n/2}\rceil,\;n_{2}=\lfloor{(n-1)/2}\rfloor$
        \end{itemize}
    \end{exampleblock}
\end{frame}

\begin{frame}
    \frametitle{\insertsubsectionhead}
    \begin{block}{性质丁}
        \begin{itemize}
            \item 具有$n$个结点的完全二叉树层数$k=\lfloor\log_{2}{n}\rfloor+1$
        \end{itemize}
    \end{block}
    \pause
    \begin{exampleblock}{证明}
        \begin{enumerate}
            \item 由完全二叉树性质与性质乙可得,$k$层完全二叉树结点数$n$满足
                \[
                    2^{k-1}-1<n\leq2^{k}-1\;\text{或}\;2^{k-1}\leq{}n<2^{k}
                \]
            \item 取对数
                \[
                    k-1\leq\log_{2}{n}<k\;\text{或}\;\log_{2}{n}<k\leq\log_{2}{n}+1
                \]
            \item 注意到$k\in\mathbb{Z}$,于是得证\hfill$\qed$
        \end{enumerate}
    \end{exampleblock}
\end{frame}

\begin{fragile}
    \frametitle{\insertsubsectionhead}
    \bicolumns[0.5]{
        \begin{exampleblock}{完全二叉树按层序编号}
            \begin{itemize}
                \item 可为完全二叉树结点按层序依次编号
                \item 约定根编号为$1$,依次递增
                \item 称如此编号为$k$的结点为\textbf{结点$k$}
            \end{itemize}
        \end{exampleblock}
    }{
        \begin{figure}
            \centering
                \begin{tikzpicture}[ >=stealth, thick, black!50, %
                        list item/.style={draw=gray, circle, thick}, %
                        level distance=4ex, %
                        level 1/.style={sibling distance=16ex}, %
                        level 2/.style={sibling distance=8ex}, %
                        level 3/.style={sibling distance=4ex}, %
                    ]
                    \node [terminal] {$1$}
                        child {node [terminal] {$2$}
                            child {node [terminal] {$4$}
                                child {node [terminal] {$8$}}
                                child {node [terminal] {$9$}}
                            }
                            child {node [terminal] {$5$}
                                child {node [terminal] {$10$}}
                                child [missing]
                            }
                        }
                        child {node [terminal] {$3$}
                            child {node [terminal] {$6$}}
                            child {node [terminal] {$7$}}
                        };
                \end{tikzpicture}
            \caption{为完全二叉树按层序编号}
            \label{fig:demo_number_proper_binary_tree_by_layer_order}
        \end{figure}
    }
\end{fragile}

\begin{frame}
    \frametitle{\insertsubsectionhead}
    \begin{block}{性质戊}
        \begin{itemize}
            \item 为含有$n$个结点的完全二叉树按层序对结点编号\footnote{根结点为
                  $1$},则有
                \begin{enumerate}
                    \item 结点$k$的左右子结点序号分别为$2k$与
                        $2k+1,\;k\in\mathbb{Z}\cap[1,\lfloor{}n/2\rfloor]$
                    \item 结点$k$的父结点序号为
                        $\lfloor{}k/2\rfloor,\;k\in\mathbb{Z}\cap(1,n]$
                \end{enumerate}
        \end{itemize}
    \end{block}
    \pause
    \begin{exampleblock}{证明}
        \begin{enumerate}
            \item 考察结论$1$,当$k=1$时,显然其左右子结点序号分别为$2$与$3$,成
                  立
            \item 假设当$k=m$时成立,即结点$m$的左右子结点序号分别为$2m$与
                  $2m+1$,
                  \begin{itemize}
                    \item 因结点$m+1$的左子结点必为结点$m$的右子结点的后继
                    \item 故结点$m+1$的左子结点序号为$(2m+1)+1=2(m+1)$
                    \item 且结点$m+1$的右子结点序号为$2(m+1)+1$
                  \end{itemize}
                  则当$k=m+1$时,假设成立,故结论$1$成立
            \item 由结论$1$知结论$2$成立\hfill\qed
        \end{enumerate}
    \end{exampleblock}
\end{frame}