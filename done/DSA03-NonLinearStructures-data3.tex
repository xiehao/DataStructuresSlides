\section{图的遍历}

\begin{fragile}
    \frametitle{\insertsectionhead}
    \begin{block}{图的连通性}
        \begin{itemize}
            \item 图中任意两顶点间均有\alert{路径}相连\footnote{满足单向相连即
                      可,亦称为\textbf{可达(reachable)};约定只有单个顶点的图为
                      连通图},则称该图为\textbf{连通图(connected graph)}
            \item 若图$G=(V,E)$与$G'=(V',E')$满足$V'\subseteq{V}$且
                  $E'\subseteq{E}$,则称$G'$为$G$的\textbf{子图(subgraph)}
            \item 称非连通图的\alert{极大连通子图}为\textbf{连通域(connected
                      component)}
        \end{itemize}
    \end{block}
    \begin{figure}
        \centering
        \begin{subfigure}[T]{0.3\textwidth}
            \centering
            \begin{tikzpicture}[ >=stealth, thick, black!50, %
                    list item/.style={draw=gray, circle, thick}, %
                ]
                \node (a) [terminal] at (-1.5,0) {$v_{0}$};
                \node (b) [terminal] at (0,-.5) {$v_{1}$};
                \node (c) [terminal] at (1.5,0) {$v_{2}$};
                \node (d) [terminal] at (0,.5) {$v_{3}$};
                \draw[->] (a) to (b);
                \draw[<-] (a) to (d);
                \draw[->] (b) to (c);
                \draw[<-] (d) to (c);
                \draw[->] (a) to (c);
            \end{tikzpicture}
            \caption{连通图}
            \label{subfig:connected_graph}
        \end{subfigure}
        ~
        \begin{subfigure}[T]{0.3\textwidth}
            \centering
            \begin{tikzpicture}[ >=stealth, thick, black!50, %
                    list item/.style={draw=gray, circle, thick}, %
                ]
                \node (a) [terminal] at (-1.5,0) {$v_{0}$};
                \node (b) [terminal] at (0,-.5) {$v_{1}$};
                \node (c) [terminal] at (1.5,0) {$v_{2}$};
                \node (d) [terminal] at (0,.5) {$v_{3}$};
                \draw[->] (a) to (b);
                % \draw[<-] (a) to (d);
                \draw[->] (b) to (c);
                % \draw[<-] (d) to (c);
                \draw[->] (a) to (c);
            \end{tikzpicture}
            \caption{非连通图}
            \label{subfig:disconnected_graph}
        \end{subfigure}
        \caption{连通图与非连通图}
        \label{fig:connected_and_disconnected_graphs}
    \end{figure}
\end{fragile}

\begin{frame}
    \frametitle{\insertsectionhead}
    \begin{block}{遍历(Traversal)\footnote{在图中亦称\textbf{搜索(search)}}}
        \begin{itemize}
            \item 按某种约定顺序访问非线性结构中的所有顶点与边
            \item 每个顶点与边均\alert{被且仅被}访问$1$次
            \item \alert{意义}:使非线性结构转化为半线性结构
            \item 遍历的产物\footnote{此处\alert{生成}亦可替换为\textbf{支撑
                          (support)}或\textbf{遍历}}
                  \begin{itemize}
                      \item 称对连通图遍历生成的\alert{树}为\textbf{生成树
                                (spanning tree)}
                      \item 称对非连通图遍历生成的\alert{森林}为\textbf{生成森林
                                (spanning forest)}
                  \end{itemize}
        \end{itemize}
    \end{block}
\end{frame}

\begin{frame}
    \frametitle{\insertsectionhead}
    \begin{alertblock}{几个关键问题}
        \begin{itemize}
            \item 遍历过程从何处出发?
                  \begin{itemize}
                      \item<2-> 原则上约定从编号最小的顶点出发
                  \end{itemize}
            \item 如何避免遗漏顶点或边?
                  \begin{itemize}
                      \item<3-> 多发生于非连通图中,可对不同连通子图分别执行
                  \end{itemize}
            \item 如何避免重复访问顶点?
                  \begin{itemize}
                      \item<4-> 在每个顶点上设置被访问状态,并在访问后更新
                  \end{itemize}
            \item 如何选择下一个顶点?
                  \begin{itemize}
                      \item<5-> 常见思路:\alert{深度}优先搜索与\alert{广度}优先
                          搜索
                  \end{itemize}
        \end{itemize}
    \end{alertblock}
\end{frame}

\subsection{深度优先搜索}

\begin{frame}
    \frametitle{\insertsubsectionhead}
    \begin{block}{深度优先搜索(Depth-First Search, DFS)}
        \begin{itemize}
            \item 基本原则:不撞南墙不回头
            \item 类似树的先序遍历或后序遍历
            \item 结果不唯一,与遍历邻接顶点的顺序有关
        \end{itemize}
    \end{block}
    \begin{alertblock}{思路}
        \begin{enumerate}
            \item 标记所有顶点为\alert{未被访问}状态
            \item 按序号依次取出\alert{未被访问}顶点并执行
                  \begin{enumerate}[a.]
                      \item 访问该顶点并将其标记为\alert{已被访问}
                      \item 对其所有\alert{未被访问}的\alert{邻接}顶点递归执行深
                            度优先搜索
                      \item 返回步骤2
                  \end{enumerate}
            \item 结束
        \end{enumerate}
    \end{alertblock}
\end{frame}

\begin{fragile}
    \frametitle{\insertsubsectionhead}
    \begin{figure}
        \centering
        \begin{tikzpicture}[ >=stealth, thick, black!50, %
                list item/.style={draw=gray, circle, thick}, %
                level distance=5ex, %
                level/.style={sibling distance=16ex/#1}
            ]
            \node [terminal] (a) {\alert<2->{$A$}}
            child {node [terminal] (b) {\alert<3->{$B$}}
                    child {node [terminal] (d) {\alert<4->{$D$}}
                            child [missing]
                            child {node [terminal] (h) {\alert<6->{$H$}}}
                        }
                    child {node [terminal] (e) {\alert<5->{$E$}}}
                }
            child {node [terminal] (c) {\alert<7->{$C$}}
                    child {node [terminal] (f) {\alert<8->{$F$}}}
                    child {node [terminal] (g) {\alert<9->{$G$}}}
                };
            \draw (d) to (e);
            \draw (f) to (g);
            \draw<3-> [red] (a) to (b);
            \draw<7-> [red] (a) to (c);
            \draw<4-> [red] (b) to (d);
            % \draw [red] (b) to (e);
            \draw<5-> [red] (d) to (e);
            \draw<8-> [red] (c) to (f);
            % \draw [red] (c) to (g);
            \draw<9-> [red] (f) to (g);
            \draw<6-> [red] (d) to (h);
        \end{tikzpicture}
        \caption{图的深度优先遍历过程:$A,B,D,E,H,C,F,G$}
        \label{fig:demo_dfs}
    \end{figure}
\end{fragile}

\begin{fragile}
    \frametitle{\insertsubsectionhead}
    \bicolumns[0.5]{
        \begin{block}{练习:从$A$开始的深度优先搜索可能结果为}
            \begin{enumerate}[I.]
                \item $A,B,E,C,D,F$
                \item $A,C,F,E,B,D$
                \item $A,E,B,C,F,D$
                \item \alert<2->{$A,E,D,F,C,B$}
            \end{enumerate}
        \end{block}
    }{
        \begin{figure}
            \centering
            \begin{tikzpicture}[ >=stealth, thick, black!50, %
                    list item/.style={draw=gray, circle, thick}, %
                    level distance=5ex, %
                    level/.style={sibling distance=8ex/#1}
                ]
                \node [terminal] (a) {$A$}
                child {node [terminal] (b) {$B$}}
                child {node [terminal] (e) {$E$}
                        child {node [terminal] (d) {$D$}}
                    }
                child {node [terminal] (c) {$C$}
                        child {node [terminal] (f) {$F$}}
                    };
                \draw (d) to (f);
            \end{tikzpicture}
            \caption{图的深度优先遍历测试}
            \label{fig:test_dfs}
        \end{figure}
    }
\end{fragile}
\subsection{广度优先搜索}

\begin{frame}
    \frametitle{\insertsubsectionhead}
    \begin{block}{广度优先搜索(Breadth-First Search, BFS)}
        \begin{itemize}
            \item 基本原则:
            \item 类似树的层次遍历
            \item 结果不唯一,与遍历邻接顶点的顺序有关
        \end{itemize}
    \end{block}
    \begin{alertblock}{思路}
        \begin{enumerate}
            \item 标记所有顶点为\alert{未被访问}状态
            \item 按序号依次取出\alert{未被访问}顶点并执行
                  \begin{enumerate}[a.]
                      \item 初始化顶点队列
                      \item 访问该顶点、将其标记为\alert{已被访问}并入队
                      \item 若队列不空,则执行
                            \begin{enumerate}[i.]
                                \item 将队首顶点出队
                                \item 对该顶点的所有未被访问邻接顶点执行:访问、
                                      标记为\alert{已被访问}并入队
                            \end{enumerate}
                      \item 销毁队列并返回步骤2
                  \end{enumerate}
            \item 结束
        \end{enumerate}
    \end{alertblock}
\end{frame}

\begin{fragile}
    \frametitle{\insertsubsectionhead}
    \begin{figure}
        \centering
        \begin{tikzpicture}[ >=stealth, thick, black!50, %
                list item/.style={draw=gray, circle, thick}, %
                level distance=5ex, %
                level/.style={sibling distance=16ex/#1}
            ]
            \node [terminal] (a) {\alert<2->{$A$}}
            child {node [terminal] (b) {\alert<3->{$B$}}
                    child {node [terminal] (d) {\alert<5->{$D$}}
                            child [missing]
                            child {node [terminal] (h) {\alert<9->{$H$}}}
                        }
                    child {node [terminal] (e) {\alert<6->{$E$}}}
                }
            child {node [terminal] (c) {\alert<4->{$C$}}
                    child {node [terminal] (f) {\alert<7->{$F$}}}
                    child {node [terminal] (g) {\alert<8->{$G$}}}
                };
            \draw (d) to (e);
            \draw (f) to (g);
            \draw<3-> [red] (a) to (b);
            \draw<4-> [red] (a) to (c);
            \draw<5-> [red] (b) to (d);
            \draw<6-> [red] (b) to (e);
            % \draw [red] (d) to (e);
            \draw<7-> [red] (c) to (f);
            \draw<8-> [red] (c) to (g);
            % \draw [red] (f) to (g);
            \draw<9-> [red] (d) to (h);
        \end{tikzpicture}
        \caption{图的一种广度优先遍历过程:$A,B,C,D,E,F,G,H$}
        \label{fig:demo_bfs}
    \end{figure}
\end{fragile}

\begin{frame}
    \frametitle{\insertsectionhead}
    \begin{table}
        \centering
        \small
        \setlength{\leftmargini}{0.4cm}
        \caption{深度优先搜索与广度优先搜索的比较}
        \label{tab:dfs_vs_bfs}
        \begin{tabular}{rll}
            \toprule
                                           & \textbf{深度优先搜索} & \textbf{广度优先搜索} \\
            \midrule
            \textbf{其余邻接顶点}                & 后遍历             & 先遍历             \\
            \midrule
            \textbf{辅助结构}                  & 栈               & 队列              \\
            \midrule
            \multirow{2}{*}{\textbf{典型用途}} & 有向无环图顶点的拓扑排序    & 无权图的单源最短路径      \\
                                           & 有向图的强连通域        & 无向图的连通域         \\
            \bottomrule
        \end{tabular}
    \end{table}
\end{frame}