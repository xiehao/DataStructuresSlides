\begin{fragile}
    \frametitle{\insertsubsectionhead}
    \begin{exampleblock}{栈与递归}
        \begin{itemize}
            \item \textbf{递归(recursion)}:函数直接或间接调用自身的过程
            \item 递归\textbf{要素}:终止条件与递归体
            \item 递归\textbf{实现}:在\alert{栈}中记录每次调用的有用信息
        \end{itemize}
    \end{exampleblock}
    \pause
    \begin{exampleblock}{例:求阶乘}
        \vspace{2ex}
        \bicolumns[0.5]{
            \vspace*{1.5ex}
            \[
                n!=\begin{cases}
                    1            & n=0 \\
                    n\cdot(n-1)! & n>0
                \end{cases}
            \]
        }{
            \begin{minted}{c}
                int factorial(int n) {
                    return 0 == n ? 1 : n * factorial(n - 1);
                }
            \end{minted}
        }
    \end{exampleblock}
\end{fragile}

\begin{fragile}
    \frametitle{\insertsubsectionhead}
    \bicolumns[0.5]{
        \begin{figure}
            \centering
            \begin{bytefield}{2}
                \stackbox{3}{12}{\only<5>{
                        \mintinline[fontsize=\smaller]{console}{factorial()} \\
                        \mintinline[fontsize=\smaller]{console}{return address = 1002} \\
                        \mintinline[fontsize=\smaller]{console}{n = 0, ...}
                    }
                } \\
                \stackbox{3}{12}{\only<4-6>{
                        \mintinline[fontsize=\smaller]{console}{factorial()} \\
                        \mintinline[fontsize=\smaller]{console}{return address = 1002} \\
                        \mintinline[fontsize=\smaller]{console}{n = 1, ...}
                    }
                } \\
                \stackbox{3}{12}{\only<3-7>{
                        \mintinline[fontsize=\smaller]{console}{factorial()} \\
                        \mintinline[fontsize=\smaller]{console}{return address = 2003} \\
                        \mintinline[fontsize=\smaller]{console}{n = 2, ...}
                    }
                } \\
                \stackbox{2}{12}{\only<2-8>{
                        \mintinline[fontsize=\smaller]{console}{main()}\\
                        \mintinline[fontsize=\smaller]{console}{n = 2, ...}
                    }
                }
            \end{bytefield}
            \caption{函数调用栈:\only<2-5>{递进调用}\only<6-9>{回归求值}}
            \label{fig:function_call_stack}
        \end{figure}
    }{
        \begin{minted}[firstnumber=1001,highlightlines={1002}]{c}
            int factorial(int n) {
                return 0 == n ? 1 : n * factorial(n - 1);
            }
        \end{minted}
        \begin{minted}[firstnumber=2001,highlightlines={2003}]{c}
            int main(void) {
                int n = 2;
                int f = factorial(n);
                return 0;
            }
        \end{minted}
    }
\end{fragile}

\begin{frame}
    \frametitle{\insertsubsectionhead}
    \begin{exampleblock}{递归应用:汉诺塔(Hanoi)问题}
        \begin{itemize}
            \item 相传世界之初有钻石宝塔甲,其上有$64$金碟,由大到小依次向上摆放
            \item 附近另有二空塔:乙、丙,与甲类似
            \item 婆罗门牧师试图将甲塔金碟移至丙塔,但需借乙塔中转
            \item 每次只移最上一碟,且不可将大碟置于小碟之上
            \item 牧师完成之日便是世界末日
        \end{itemize}
    \end{exampleblock}
\end{frame}

\begin{fragile}
    \frametitle{\insertsubsectionhead}
    \bicolumns[0.5]{
        \begin{exampleblock}{递归解题思路}
            \begin{enumerate}
                \item 若剩余碟数$n=0$,则结束
                \item 否则执行:
                      \begin{itemize}
                          \item 将$n-1$碟从甲借丙移至乙
                          \item 将甲中剩余一碟从甲移至丙
                          \item 将$n-1$碟从乙借甲移至丙
                      \end{itemize}
            \end{enumerate}
        \end{exampleblock}
    }{
        \begin{minted}{c}
            void move(char a, char b) {
                printf("%c -> %c\n", a, b);
            }
        \end{minted}
        \begin{minted}{c}
            void hanoi(int n, char a, char b, char c) {
                if (!n) {
                    return; // 递归出口
                }
                hanoi(n - 1, a, c, b);
                move(a, c);
                hanoi(n - 1, b, a, c);
            }
        \end{minted}
    }
\end{fragile}

\begin{frame}
    \frametitle{\insertsubsectionhead}
    \begin{alertblock}{注意}
        \begin{itemize}
            \item 递归可在抽象层面帮助快速理清思路
            \item 递归需借助栈结构存储额外信息,故效率较迭代/循环\alert{低}
            \item 在实现中,栈容量有限,故过多层递归极易\alert{爆栈}
        \end{itemize}
    \end{alertblock}
\end{frame}